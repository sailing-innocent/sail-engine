\begin{frame}
    \frametitle{遗传算法}
    \begin{enumerate}
        \item 经典的遗传算法(Genetic Algorithm)指的是将数值程序的解看做一长串字符,并通过字符种群的随机杂交(crossover)突变(mutation),并通过目标函数来进行筛选(selection)来最终启发式地优化到最优解。
        \item 遗传算法有跳出局部最优达到全局最优的能力
        \item 遗传算法受限于初始化的水平和参数选择
    \end{enumerate}
\end{frame}

\begin{frame}
    \frametitle{抽象语法树}
    \begin{figure}
        \centering
        \includegraphics[width=0.4\textwidth]{fig_ast_example.png}
        \caption{The Example of Abstract Syntax Tree $(2.2-\frac{x}{11})+(7\cos{y})$}
        \label{fig:ast_example}
    \end{figure}
    \begin{quote}    
抽象语法树是一种编程表达方法,可以把文本转化为结构化统一的树形,如图\ref{fig:ast_example}所示,程序$(2.2-\frac{x}{11})+(7\cos{y})$ 可以展现为四个二元操作符(Binary Operator)$+,-,*,/$和一个函数操作符(Call Operator)$\cos$和字面量符号(literal variable)的组合
    \end{quote}
\end{frame}

\begin{frame}
    \frametitle{遗传编程}
    \begin{enumerate}
        \item 遗传编程(Genetic Programming)就是作用在抽象语法树上的遗传算法
        \item 交叉crossover: 从两个个体中各自选择一个节点,并将个体A中自节点往下的子树删除,并将个体B的子树复制到个体A的节点下。
        \item 复制replication: 部分个体会简单的复制到下一个种群
        \item 突变mutation: 突变策略非常多样,可能是随机替换选择某一个叶子节点,也可能是对某一条支路随机裁剪。
        \item 选择Selection: 对于每一个种群,执行得到目标函数,设定选择标准,只有达到标准的个体可以进行下一代的繁殖操作。
    \end{enumerate}
\end{frame}