遗传编程(Genetic Programming)是一种将进化算法和编程结合的技术,通过将程序表示为一个抽象语法树(Abstract Semantic Tree),并对树的部分节点进行一定的修建,复制,转移,拼贴等进化策略,最终实现最优的程序实现。

\subsection{抽象语法树}
抽象语法树是一种编程表达方法,可以把文本转化为结构化统一的树形,进而方便后续的优化操作。抽象语法树广泛用于编译器的构建过程,解析文本构建抽象语法树往往是所有编译过程的第一步。

如图\ref{fig:ast_example}所示,程序$(2.2-\frac{x}{11})+(7\cos{y})$ 可以展现为四个二元操作符(Binary Operator)$+,-,*,/$和一个函数操作符(Call Operator)$\cos$和字面量符号(literal variable)的组合

\begin{figure}[H]
    \centering
        \includegraphics[width=0.8\textwidth]{fig_ast_example.png}
        \caption{The Example of Abstract Syntax Tree $(2.2-\frac{x}{11})+(7\cos{y})$}
    \label{fig:ast_example}
\end{figure}


\subsection{遗传算法}

经典的遗传算法(Genetic Programming)是为了解决最优化问题而提出的一种启发式算法,参考了生物种群进化的过程。遗传算法将适合的解编码为一段二进制字符,称为个体,而所有可行解的集合称为种群。在每一次迭代中,都对当前种群内的个体进行交叉操作(crossover)和突变(mutation)产生新的可行解,再对新种群内的所有个体进行选择(selection),最终得到更靠近最优解的种群分布。

遗传算法有跳出局部最优达到全局最优的能力,允许使用非常复杂的目标函数。但是参数选择不当也可能无法跳出局部最优,如果初始种群数量选择不当,则可能会无法搜索覆盖到全局最优,或者耗费大量计算资源。

\subsection{遗传编程}

遗传编程则是经典的遗传算法在优化程序结构上的应用,通过将经典遗传算法的操作对象由二进制字符串改为抽象语法树,遗传编程可以继承遗传算法的优点。虽然在计算量上更加复杂,但是能解决遗传算法无法解决的问题。

遗传编程的种群操作如下:

\begin{enumerate}
    \item 交叉crossover: 从两个个体中各自选择一个节点,并将个体A中自节点往下的子树删除,并将个体B的子树复制到个体A的节点下。
    \item 复制replication: 部分个体会简单的复制到下一个种群
    \item 突变mutation: 突变策略非常多样,可能是随机替换选择某一个叶子节点,也可能是对某一条支路随机裁剪。
    \item 选择Selection: 对于每一个种群,执行得到目标函数,设定选择标准,只有达到标准的个体可以进行下一代的繁殖操作。
\end{enumerate}

