

\begin{equation}
I_\lambda(\hat{\mathbf{x}})=
\int_0^L c_\lambda(\hat{\mathbf{x}}, \xi)
g(\hat{\mathbf{x}},\xi)
e^{-\int_0^\xi g(\hat{\mathbf{x}},\mu)d\mu}
d\xi
\label{eq:volume_render_equation}
\end{equation}

where $g(\mathbf{x})$ is the \textit{extinction function}
that defines the rate of light occlusion, 
and $c_\lambda(\mathbf{x})$ is an emission coefficient.

The exponential term is \textit{attenuation factor}, 
and the production of $c_\lambda(x)g(x)$ is also called source term, 
describing the light intensity scattered in the direction of light of ray 
$\hat{x}$ at point x.




Now we assume the extinction function is the weighted 
sum of coefficients and reconstruction kernels
\begin{equation}
g(x)=\sum\limits_{k}\limits^{} g_kr_k(\mathbf{x})
\label{eq:extinction_function}
\end{equation}


This reconstruction kernel reflects the position and shape for individual particles.

We substitude the equation (\ref{eq:extinction_function}) into the volume rendering equation (\ref{eq:volume_render_equation}):

$$I_{\lambda}(\hat{\mathbf{x}})=
\sum\limits_{k}\limits^{} 
(
    \int_0^L c_\lambda(\hat{\mathbf{x}},\xi)g_kr_k(\hat{\mathbf{x}},\xi)
    \prod\limits_{j }\limits^{} e^{
        g_j \int_0^\xi r_j(\hat{\mathbf{x}},\mu)d\mu
    } d\xi
)$$

To compute this function mathematically, we often use assumptions

\begin{enumerate}
    \item use simplified reconstruction kernel, e.g. Gaussian Kernel.
    \item the local support of each reconstruction won't overlap with each other along the ray.
    \item the reconstruction kernels can be ordered front to back
    \item the emission coefficient is constant 
\end{enumerate}

Further more, we use the Taylor expansion for the exponential function, and we can get the final equation for Splatting:


\begin{equation}
I_{\lambda}(\hat{\mathbf{x}})=
\sum\limits_{k}\limits^{} 
(
    c_{\lambda k}(\hat{\mathbf{x}})g_kq_k(\hat{\mathbf{x}})\prod\limits_{j=0}\limits^{k-1} (1-g_jq_j(\hat{\mathbf{x}}))
)\label{eq:splatting_equation}
\end{equation}

Where $q_k(\hat{\mathbf{x}}) = \int_{\mathbb{R}}r_k(\hat{\mathbf{x}},x_2)dx_2$ 

