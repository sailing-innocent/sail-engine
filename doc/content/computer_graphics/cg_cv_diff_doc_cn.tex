图形学经常与视觉搞混,两者都是利用计算机算法对物理世界的建模,但图形学建模的是图片形成之前的阶段,绝大部分是物理的范畴,主要是光线与物体表面交互,光在空气中传输,光最终映射到人眼形成图像,而计算机视觉建模的是图像形成之后人脑如何对图像进行理解的过程,从图像中提取特征,识别特定的对象和理解抽象的概念等等。换言之,计算机图形学的兴趣在于物理世界,而计算机视觉的兴趣在于人的视觉理解系统,两者虽然互有涉猎,且在色彩模型和相机模型上有共通知识,但归根到底是不同的学科范畴。