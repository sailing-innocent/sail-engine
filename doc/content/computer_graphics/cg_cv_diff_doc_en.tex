Graphics is often confused with computer vision. Both use computer algorithms to model the physical world, but graphics models the stages before an image is formed, which mostly falls within the realm of physics. It mainly involves the interaction of light with the surface of objects, the transmission of light through air, and the final mapping of light to the human eye to form images. Computer vision, on the other hand, models the process of how the brain understands images after they are formed, extracting features from images, recognizing specific objects, and understanding abstract concepts, etc. In other words, the interest of computer graphics lies in the physical world, while the interest of computer vision lies in the human visual understanding system. Although the two fields overlap and share common knowledge in color models and camera models, they are fundamentally different academic disciplines.

