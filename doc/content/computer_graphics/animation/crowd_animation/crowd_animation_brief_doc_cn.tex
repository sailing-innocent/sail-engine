\paragraph{人群动画} 在游戏,电影等内容生产的过程中,为了给存在许多人的环境,比如地铁,广场,马路等更真实的体验,我们往往希望能给环境添加具有真实感的人群动画。人群动画的任务是给定一个环境描述,生成一段有真实感的人群运动控制率,进而在底层的动画引擎中驱动人物网格体来生成逼真的人群动画。

\paragraph{人群动画的挑战} 我们希望人群动画尽可能真实,这就要求人群动画需要

\begin{enumerate}
    \item 避免碰撞,避免人物之间的重叠
    \item 避免不自然的效果,比如人物之间的穿越,或者恶意挤压某个人物轨迹导致不自然折返
    \item 按照正确的路径行走,每一个人物的目的和其轨迹需要符合逻辑。
\end{enumerate}

但是在模拟的过程中,我们会遇到一些问题:

\begin{enumerate}
    \item 环境的语义多样,不同的环境描述会导致不同的人群行为
    \item 人群和环境的交互是多模态的,不同的人群行为会导致不同的环境变化
    \item 物理正确的模拟是计算密集型的,需要大量的计算资源
\end{enumerate}

\paragraph{人群动画的主要方法} 人群动画的主要方法有三类:

\begin{figure}
    \centering
        \begin{subfigure}{0.3\linewidth}
            \includegraphics[width=\textwidth]{fig_navigation_field.png}
            \caption{Flow-Based Method}
            \label{fig:flow_based_method}
        \end{subfigure}
        \begin{subfigure}{0.3\linewidth}
            \includegraphics[width=\textwidth]{fig_maslow_hierarchy_of_needs.png}
            \caption{Entity-Based Method}
            \label{fig:entity_based_method}
        \end{subfigure}
        \begin{subfigure}{0.3\linewidth}
            \includegraphics[width=\textwidth]{fig_agent_based_crowd.png}
            \caption{Agent-Based Method}
            \label{fig:agent_based_method}
        \end{subfigure}
\end{figure}

\begin{enumerate}
    \item 基于流的方法:如图\ref{fig:flow_based_method}通过流场来控制人群的运动,比如\cite{patilDirectingCrowdSimulations2011}
    \item 基于实体的方法:如图\ref{fig:entity_based_method}通过实体之间的关系来控制人群的运动,往往会涉及依稀心理学中的知识,构建基于规则的人工智能
    \item 基于智能体的方法:如图\ref{fig:agent_based_method}通过智能体之间的交互来控制人群的运动,比如\cite{tanakaGuidanceFieldVector2016}
\end{enumerate}

