\begin{frame}
    \frametitle{体渲染器}
    \begin{figure}
        \centering
        \includegraphics[width=\linewidth]{fig_volume_rendering.png}
        \caption[short]{体渲染器原理}
    \end{figure}
\end{frame}
\begin{frame}
    \frametitle{光线步进算法(Ray March)}
    \begin{itemize}
        \item 对于每一根射线,从初始距离$t_n$开始,到最远距离$t_f$为止,步步采样每个位置的颜色$c(t)$和密度$\sigma(t)$
        \item 最终像素是射线上的样本加权和:$C(\mathbf{r})=\int_{t_n}^{t_f}T(t)\sigma(\mathbf{r}(t))\mathbf{c}(\mathbf{r}(t),\mathbf{d})dt$
        \item 此处$T(t)=exp(-\int_{t_n}^{t}\sigma(\mathbf{r}(s))ds)$代表t之后的颜色依然可以残留多少
    \end{itemize}
    \begin{figure}
        \begin{subfigure}{0.48\textwidth}
            \includegraphics[width=\linewidth]{fig_ray_march.png}
            \caption[short]{光线步进}
        \end{subfigure}
        \begin{subfigure}{0.48\textwidth}
            \includegraphics[width=\linewidth]{fig_ray_march_shade.png}
            \caption[short]{传统光估计}
        \end{subfigure}
    \end{figure}
\end{frame}