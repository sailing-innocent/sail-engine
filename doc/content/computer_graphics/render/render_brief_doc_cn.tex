“渲染”这一过程,简单来说,就是将构成模型的基本几何单元(三角面,点,体素等)经过变换映射到虚拟的相机平面上的过程,其输入是场景的描述,包括几何模型,材质,光照,相机参数等,输出是一张固定分辨率的2D图片。

在“渲染”过程中,场景信息非常庞杂(复杂场景可能会达到百万甚至更高级别的三角面),同时最终渲染出图的分辨率也是极高的,比如(1024x1024)的图中有百万个的像素。而更不幸的是,每一个像素的结果都与每一个场景信息都有关联,如果简单地串行遍历,那么这个mxn(m,n都在百万级别)的计算复杂度是我们无法接受的。

所以,我们必须把其中一些步骤改为并行,那么要么就并行地处理场景,让每一个场景元素都独立地投影到屏幕上,然后再并行地在像素屏幕上混合;要么从一开始就并行地处理像素,从每个像素发出一条光线在场景中进行采样,最后采样的结果直接就是像素的颜色。前者就是所谓的光栅方法,后者就是所谓的光追(光线追踪)方法。光线追踪方法更符合人的自然想法,而光栅方法更符合机器的原型。所以在机器性能不佳的场合,光栅方法占据了绝对主流,而近年来随着计算能力的提升,部分显卡也支持光线追踪核心(Ray Tracing Core)来加速光线追踪方法的计算。
