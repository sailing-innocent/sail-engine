% require \usetikzlibrary{3d}
\input{fig_flip_y_camera}

如图\ref{fig:flip_y_camera}所示,一个正常的右手系世界坐标系$\bf{x_w,y_w,z_w}$和原点$\bf{O}$下,
相机视角的局部坐标系$\bf{x_c,y_c,z_c}$和相机原点在世界坐标系下的坐标 $\bf{c}$

$$\left[\begin{matrix} 
    \vec{x}_c \\ \vec{y}_c \\ \vec{z}_c 
\end{matrix}\right]=
\left[\begin{matrix} 
    x_{c0} & x_{c1} & x_{c2} \\ 
    y_{c0} & y_{c1} & y_{c2} \\ 
    z_{c0} & z_{c1} & z_{c2} 
\end{matrix}\right]
\left[\begin{matrix} 
    \vec{x} \\ \vec{y} \\ \vec{z} 
\end{matrix}\right]=
R\left[\begin{matrix} 
    \vec{x} \\ \vec{y} \\ \vec{z} 
\end{matrix}\right]$$

\subparagraph{视角变换 View Transform}

在相机确定之后,任意一个世界坐标$(p_{wx},p_{wy},p_{wz})$都可以经过一个旋转
和一个平移变换到相机坐标系$(p_{cx},p_{cy},p_{cz})$之中,我们不妨引入齐次坐标的概念

来确定一个$4\times 4$的变换矩阵$\bf{M}$,使得

$[p_{cx},p_{cy},p_{cz},1]=[p_{wx},p_{wy},p_{wz},1]\bf{M}$

并且可以推导出$\bf{M}$的表达式

$$[p_{wx},p_{wy},p_{wz},1]
\left[\begin{matrix} 
    x_{c0} & y_{c0} & z_{c0} & 0 \\ 
    x_{c1} & y_{c1} & z_{c1} & 0 \\ 
    x_{c2} & y_{c2} & z_{c2} & 0 \\ 
    -\vec{c}\cdot\vec{x}_c &-\vec{c}\cdot\vec{y}_c &-\vec{c}\cdot\vec{z}_c & 1 
\end{matrix}\right]=[p_{cx},p_{cy},p_{cz}, 1]$$


这里的M被称为View Matrix视角矩阵。
它的作用就是把一个世界坐标系下的点变换到相机局部坐标系(NDC)下。

\subparagraph{透视投影变换 Perspective Transform}

在变换到相机局部坐标系之后,我们还希望进行一个投影变换,

把相机所包括的锥体变换为一个长方体,具体来说,就是把点 $(x_c,y_c,z_c)$

变换为 $x_c/z_c, y_c/z_c, z_c$ 这个过程叫做透视投影变换perspective transform

人眼所看到的物体大多是经过透视投影变换的结果。

透视投影变换是一种非仿射变换,导致的结果只保形,不保角,两条平行线在经过透视投影变换
之后会相交(正如我们观察铁轨时候经常发现的那样)

透视投影变换的实现方式并不唯一,在实际中与每一个相机的内部参数密切相关。

