自从2020年开始,大语言模型逐渐展现出其强大的能力,成功跨越了图灵测试,在聊天机器人,文本语义理解与提取,图像理解多模态等很多领域取得了巨大的进展。

如今基于大语言模型的研究也深度改变了过往的科研范式。在自然语言处理领域,大语言模型的出现使得很多传统的任务都可以通过预训练模型来解决,如文本分类,文本生成,文本相似度计算等。在计算机视觉领域,大语言模型也可以通过文本描述来生成图像,实现多模态的交互。在语音识别领域,大语言模型也可以通过文本生成语音,实现多模态的交互。在自然语言处理领域,大语言模型的出现使得很多传统的任务都可以通过预训练模型来解决,如文本分类,文本生成,文本相似度计算等。在计算机视觉领域,大语言模型也可以通过文本描述来生成图像,实现多模态的交互。在语音识别领域,大语言模型也可以通过文本生成语音,实现多模态的交互。

\begin{figure}[H]
    \includegraphics[width=\textwidth]{fig_llm_history.png}
    \caption{Overview of LLM \cite{naveedComprehensiveOverviewLarge2024}}
    \label{fig:llm_history}
\end{figure}

如图\ref{fig:llm_history}所示,大语言模型的发展经历了几个阶段,从GPT-3到GPT-4,再到GPT-4o,每一代的模型都在模型规模,模型性能,模型应用等方面有了很大的提升。其中GPT-4o是最新的一代大语言模型,它是一个多模态的大语言模型,可以同时处理文本,图像,语音等多种模态的信息。