整理于2023年11月17日

\paragraph{写入ppm}

ppm是一种用纯文本来描述图片的格式,没有经过压缩,所以一般不方便用来储存和转移,但是比较适合用来作为测试和教学的工具。首先我们尝试写入一个256x256的ppm文件:

\begin{figure}[H]
    \includegraphics[width=\textwidth]{"fig_demo_rtow_01_write_image.png"}
    \caption{image.ppm}
\end{figure}

\paragraph{定义向量和光线}

之后我们定义vector和ray的类,得到每一个像素对应的光线方向,并按照归一化后光线方向的y坐标插值,最终得到如图所示的图片。这样我们就将最终图片的每一个像素点都看做若干光线采样的组合。

\begin{figure}[H]
    \includegraphics[width=\textwidth]{"fig_demo_rtow_02_simple_ray.png"}
    \caption{ray color}
\end{figure}


\paragraph{光线与物体相交}

通过数学推导,我们可以写出一条射线和一个球是否相交的判别式,并将相交的射线颜色置为红色

\begin{figure}[H]
    \includegraphics[width=\textwidth]{"fig_demo_rtow_03_hit_sphere.png"}
    \caption{Hit Sphere}
\end{figure}

\paragraph{法向量}

% 我们可以通过判别式得到最近点的t,进而根据球的几何描述得到对应交点的坐标和表面法向量,并将其可视化如图。

\begin{figure}[H]
    \includegraphics[width=\textwidth]{"fig_demo_rtow_05_sphere_normal.png"}
    \caption{Sphere Normal}
\end{figure}

\paragraph{碰撞类抽象}

我们可以使用碰撞记录和可碰撞物体对于场景进行抽象,此处我们只有两个实例球和可碰撞列表,之后可以添加更多的实例。

\begin{figure}[H]
    \includegraphics[width=\textwidth]{"fig_demo_rtow_06_hittable_world.png"}
    \caption{Multiple Spheres}
\end{figure}

\paragraph{消除锯齿}

这里我们对于每一个pixel,都通过随机数任意在周围选取若干射线,并将最终结果累加平均,通过多次采样(Multiple Sampling)来抗锯齿。此处使用spp为10。

\begin{figure}[H]
    \centering
        \begin{subfigure}{0.48\linewidth}
            \includegraphics[width=\textwidth,trim={150 100 250 130},clip]{"fig_demo_rtow_06_hittable_world.png"}
            \caption{Not Antialiased}
        \end{subfigure}
        \begin{subfigure}{0.48\linewidth}
            \includegraphics[width=\textwidth,trim={150 100 250 130},clip]{"fig_demo_rtow_04_antialiasing.png"}
            \caption{Antialiased}
        \end{subfigure}
\end{figure}

\paragraph{不同反射深度}

之后我们可以选择在表面的半球进行随机采样下一根线条,并且按照2的幂次等比下降重要性进行加权求和。



\begin{figure}[H]
    \centering
        \begin{subfigure}{0.33\linewidth}
            \includegraphics[width=\textwidth]{"fig_demo_rtow_07_matte_1.png"}
            \caption{Max Depth = 1}
        \end{subfigure}
        \begin{subfigure}{0.33\linewidth}
            \includegraphics[width=\textwidth]{"fig_demo_rtow_07_matte_10.png"}
            \caption{Max Depth = 10}
        \end{subfigure}
        \begin{subfigure}{0.33\linewidth}
            \includegraphics[width=\textwidth]{"fig_demo_rtow_07_matte_50.png"}
            \caption{Max Depth = 50}
        \end{subfigure}
\end{figure}

\paragraph{Lambertian}

使用Lambertian材质




\begin{figure}[H]
    \centering
    \begin{subfigure}{0.25\linewidth}
        \includegraphics[width=\textwidth]{"fig_demo_rtow_08_lambertian_10_10.png"}
        \caption{depth=10,spp=10}
    \end{subfigure}
    \begin{subfigure}{0.25\linewidth}
        \includegraphics[width=\textwidth]{"fig_demo_rtow_08_lambertian_10_100.png"}
        \caption{depth=10,spp=100}
    \end{subfigure}
\end{figure}
\begin{figure}[H]
    \centering
    \begin{subfigure}{0.25\linewidth}
        \includegraphics[width=\textwidth]{"fig_demo_rtow_08_lambertian_50_10.png"}
        \caption{depth=50,spp=10}
    \end{subfigure}
    \begin{subfigure}{0.25\linewidth}
        \includegraphics[width=\textwidth]{"fig_demo_rtow_08_lambertian_50_100.png"}
        \caption{depth=50,spp=100}
    \end{subfigure}
\end{figure}

这里我们比较Lambertian和Matte材质:Lambertian的阴影会更细腻一点,当然这里因为场景过于简单,差距不会太明显。

\begin{figure}[H]
    \centering
    \begin{subfigure}{0.48\linewidth}
        \includegraphics[width=\textwidth]{"fig_demo_rtow_07_matte_50.png"}
        \caption{Matte}
    \end{subfigure}
    \begin{subfigure}{0.48\linewidth}
        \includegraphics[width=\textwidth]{"fig_demo_rtow_08_lambertian_50_10.png"}
        \caption{Lambertian}
    \end{subfigure}
\end{figure}

\paragraph{其他材质}

金属材质

\begin{figure}[H]
    \includegraphics[width=\textwidth]{"fig_demo_rtow_09_materials_50_100.png"}
    \caption{Metal Materials}
\end{figure}


% 玻璃材质


% \begin{figure}[H]
%     \includegraphics[width=\textwidth]{"fig_dielectric_rtow_50_100.png"}
%     \caption{Glass Material}
% \end{figure}

% \paragraph{高级相机}

% \begin{figure}[H]
%     \centering
%     \begin{subfigure}{0.48\linewidth}
%         \includegraphics[width=\textwidth]{"fig_advanced_camera_rtow_20.png"}
%         \caption{vfov=20}
%     \end{subfigure}
%     \begin{subfigure}{0.48\linewidth}
%         \includegraphics[width=\textwidth]{"fig_advanced_camera_rtow_90.png"}
%         \caption{vfov=90}
%     \end{subfigure}
% \end{figure}

% \paragraph{最终渲染效果}

% \begin{figure}[H]
%     \includegraphics[width=\textwidth]{"fig_final_rtow.png"}
%     \caption{Final Ray Tracer}
% \end{figure}