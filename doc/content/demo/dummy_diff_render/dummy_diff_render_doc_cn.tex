用ing框架构建了一个最基础的可微渲染器,并将其bind到pytorch框架中使用

实现于2023年12月17日初始化如下

\begin{figure}[H]
    \centering
        \begin{subfigure}{0.3\linewidth}
            \includegraphics[width=\textwidth]{fig_result_ddr_target_dummy_diff_raster}
            \caption{Target}
        \end{subfigure}
        \begin{subfigure}{0.3\linewidth}
            \includegraphics[width=\textwidth]{fig_result_ddr_source_0_dummy_diff_raster}
            \caption{Initial Source}
        \end{subfigure}
        \begin{subfigure}{0.3\linewidth}
            \includegraphics[width=\textwidth]{fig_result_ddr_target_0_dummy_diff_raster}
            \caption{Initial Target}
        \end{subfigure}
\end{figure}

进行迭代后的最终效果

\begin{figure}[H]
    \centering
        \begin{subfigure}{0.3\linewidth}
            \includegraphics[width=\textwidth]{fig_result_ddr_target_dummy_diff_raster}
            \caption{Target}
        \end{subfigure}
        \begin{subfigure}{0.3\linewidth}
            \includegraphics[width=\textwidth]{fig_result_ddr_source_final_dummy_diff_raster}
            \caption{Final Source}
        \end{subfigure}
        \begin{subfigure}{0.3\linewidth}
            \includegraphics[width=\textwidth]{fig_result_ddr_target_final_dummy_diff_raster}
            \caption{Final Target}
        \end{subfigure}
\end{figure}

全过程loss下降曲线

\begin{figure}[H]
    \input{pgf_ing_dummy_diff_raster}
    \caption{Loss Function}
\end{figure}