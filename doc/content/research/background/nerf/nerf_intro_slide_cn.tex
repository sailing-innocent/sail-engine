\begin{frame}
    \frametitle{体渲染器}
    \begin{figure}
        \centering
        \includegraphics[width=\linewidth]{fig_volume_rendering.png}
        \caption[short]{体渲染器原理}
    \end{figure}
\end{frame}
\begin{frame}
    \frametitle{光线步进算法(Ray March)}
    \begin{itemize}
        \item 对于每一根射线,从初始距离$t_n$开始,到最远距离$t_f$为止,步步采样每个位置的颜色$c(t)$和密度$\sigma(t)$
        \item 最终像素是射线上的样本加权和:$C(\mathbf{r})=\int_{t_n}^{t_f}T(t)\sigma(\mathbf{r}(t))\mathbf{c}(\mathbf{r}(t),\mathbf{d})dt$
        \item 此处$T(t)=exp(-\int_{t_n}^{t}\sigma(\mathbf{r}(s))ds)$代表t之后的颜色依然可以残留多少
    \end{itemize}
    \begin{figure}
        \begin{subfigure}{0.48\textwidth}
            \includegraphics[width=\linewidth]{fig_ray_march.png}
            \caption[short]{光线步进}
        \end{subfigure}
        \begin{subfigure}{0.48\textwidth}
            \includegraphics[width=\linewidth]{fig_ray_march_shade.png}
            \caption[short]{传统光估计}
        \end{subfigure}
    \end{figure}
\end{frame}
\begin{frame}
    \frametitle{用神经网络估计$c(t)$和$\sigma(t)$}
    \begin{figure}[H]
        \centering
        \includegraphics[width=0.9\linewidth,keepaspectratio]{fig_nerf_principle.jpg}
        \caption{神经辐射场原理}
        \label{fig:nerf_principle}
    \end{figure}
    $c(t),\sigma(t)=MLP(\mathbf{x}(t),\mathbf{d}(t))$
    \begin{itemize}
        \item 通过引入MLP,将整个渲染过程变成了一种可微的过程,从而实现了一个可微的体渲染器
        \item 与通常的神经网络不同,此处MLP的角色更类似点云,网格等存储格式
        \item 最终预测效果达到大幅度的提升,在集成数据集800x800的图片上PSNR达到了30+
    \end{itemize}
\end{frame}

\begin{frame}
    \frametitle{新视角生成问题数据集}
    \begin{figure}[H]
        \includegraphics[width=0.9\linewidth]{fig_demo_nerf_dataset.png}
        \caption[short]{渲染得到不同视角图片}
    \end{figure}
\end{frame}

\begin{frame}
    \frametitle{NeRF之后的开放问题}
    \begin{itemize}
        \item 更高的精度
        \item 更快的训练与推理速度
        \item 支持特殊场景(动态,反射,暗光,大场景)
        \item 支持编辑(材质,光照,前后景分割等)与泛化
        \item 拓展任务(与Diffusion, SAM等结合)与应用
    \end{itemize}
\end{frame}

\begin{frame}
    \frametitle{近三年NeRF的发展 -- 更高的精度}
    \begin{itemize}
        \item MipNeRF \cite{barronMipNeRFMultiscaleRepresentation2021} 降低模糊
        \item 2022 CVPR Mip NeRF 360 \cite{barronMipNeRF360Unbounded2022} 降低模糊并支持360度全景
        \item 2023 CVPR Zip NeRF \cite{barronZipNeRFAntiAliasedGridBased2023} 进一步引入了NGP的hash encoding
        \item 2023 CVPR NeRFLix \cite{zhouNeRFLiXHighQualityNeural2023} 高清NeRF
    \end{itemize}
\end{frame}

\begin{frame}
    \frametitle{近三年NeRF的发展 -- 更快的速度}
    \begin{itemize}
        \item 2022 CVPR: PointNeRF\cite{xuPointNeRFPointbasedNeurala} 将NeRF变换为特征点云
        \item 2022 CVPR: Direct Voxel Grid Optimization Super Fast Convergence for Radiance  Fields Reconstruction \cite{sunDirectVoxelGrid2022} 使用体素混合MLP
        \item 2022 CVPR: Plenoxels \cite{yuPlenoxelsRadianceFields2021} 完全使用稀疏体素而放弃MLP进行新视角识别
        \item 2022 SIGGRAPH: Instant NGP \cite{mullerInstantNeuralGraphics2022} 使用多精度空间哈希编码将大网络转化为小网络,从而将速度优化到了实时
        \item 2023 MeRF\cite{reiserMERFMemoryEfficientRadiance2023} 减少内存占用
        \item 2023 SIGGRAPH: 3D Gaussian Splatting \cite{kerbl3DGaussianSplatting2023} 使用3D高斯,放弃MLP进行机器学习优化,获得了当前SOTA的精度和效率
    \end{itemize}
    \begin{quote}
        绝大部分的加速措施都需要修改原本的网络,将其中神经网络的部分尽可能地减少甚至消除
    \end{quote}
\end{frame}

\begin{frame}
    \frametitle{近三年NeRF的发展 -- 特殊的场景}
    \begin{itemize}
        \item 2021 CVPR NeRF in the wild \cite{martin-bruallaNeRFWildNeural2021} 不受限制的室外场景
        \item 2023 CVPR Deblur NeRF \cite{leeDPNeRFDeblurredNeural2023} 去镜头模糊
        \item 2023 CVPR Large Urban Scene \cite{xuGridguidedNeuralRadiance2023} 大规模城市场景(复现困难?)
        \item 2022 CVPR D-NeRF \cite{pumarolaDNeRFNeuralRadiance2021} 基于基准场景和变换的动态场景
        \item 2023 CVPR k-planes \cite{KPlanesExplicitRadiance2023} 使用正交投影来支持4D动态场景
    \end{itemize}
\end{frame}

\begin{frame}
    \frametitle{近三年NeRF的发展 -- 编辑与泛化}
    \begin{itemize}
        \item 2023 CVPR Nope NeRF \cite{bianNoPeNeRFOptimisingNeural2023} 不需要相机位姿的NeRF
        \item 四面体网格 tetrahedra\cite{kulhanekTetraNeRFRepresentingNeural2023} 方便后续编辑
    \end{itemize}
\end{frame}

\begin{frame}
    \frametitle{近三年NeRF的发展 -- 拓展与应用}
    \begin{itemize}
        \item 2022 CVPR CLIP NeRF\cite{wangCLIPNeRFTextandImageDriven2022} 尝试了使用文本到场景的转换
        \item 2021 CVPR GIRAFFE \cite{niemeyerGIRAFFERepresentingScenes2021} 使用NeRF作为GAN的后处理生成场景
        \item 2022 CVPR GIRAFFE HD \cite{xueGIRAFFEHDHighResolution2022} 更高清的GIRAFFE
        \item 2023 CVPR Lift3D \cite{liLift3DSynthesize3D2023} 使用3D GAN进行后续生成
        \item Dream Fusion \cite{pooleDreamFusionTextto3DUsing2022} 将 Diffusion Model和NeRF结合,利用SDS loss实现文本控制3D模型生成
    \end{itemize}
\end{frame}