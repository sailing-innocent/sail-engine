A Radiance Field is a complex, 
multi-variable function, denoted as $f_c(\vec{x},\vec{d})$, 
which represents the intensity of radiance in a scene. 
This function considers each color $c$, 
every point $\vec{x}$, and all directions $\vec{d}$. 
Due to its complexity, it's often impractical to 
express this function explicitly as a mathematical equation. 
Instead, researchers employ various approximation methods 
such as Neural Networks, Grids, Hashes, or 3D Gaussians. 
Once the Radiance Field is generated, it enables the synthesis of novel scene views. 
This is achieved by integrating the radiance intensity along the rays extending from the camera to the scene, 
or by projecting the radiance intensity onto the image plane.