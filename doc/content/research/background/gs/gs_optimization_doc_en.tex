Here we introduce the overall optimization process of 
the Gaussian Splatting Algorithm. 

After initializing the Gaussian Parameters $\{\mathcal{G}\}$, repeatedly perform the following steps:

\begin{enumerate}
    \item randomly choose a view $\mathcal{P}$ from dataset and its corresponding ground truth image $\mathcal{I}_{gt}$
    \item Calculate the output image $\mathcal{I}_{w\times h}$ using the Forward Pass
    \item Calculate the loss function $\mathcal{L}$ between the output image $\mathcal{I}_{w\times h}$ and the ground truth image $\mathcal{I}_{gt}$
    \item Calculate the derivative of the loss function with respect to the output image $\frac{\partial \mathcal{L}}{\partial \mathcal{I}}$
    \item Calculate the derivative of the output image with respect to the input Gaussian Parameters $\{\mathcal{G}\}$: $\frac{\partial \mathcal{I}}{\partial \mathcal{G}}$ and back-propagate the derivatives from the output image to the input Gaussian Parameters $\{\mathcal{G}\}$: $\frac{\partial \mathcal{L}}{\partial \mathcal{G}} = \frac{\partial \mathcal{L}}{\partial \mathcal{I}}\frac{\partial \mathcal{I}}{\partial \mathcal{G}}$ using the Backward Pass
    \item Update the Gaussian Parameters $\{\mathcal{G}\}$ using gradient descent
\end{enumerate}

The optimization process is repeated until the Gaussian Parameters $\{\mathcal{G}\}$ converge to the optimal solution, 
where the output image $\mathcal{I}$ is close enough to the ground truth image $\mathcal{I}_{gt}$. 
We will save the optimal Gaussian Parameters $\hat{\{\mathcal{G}\}}$ for the next stage of evaluation.