The 3D Gaussian Splatting method interprets a scene 
as a collection of $N$ 3D Gaussians, forming a characteristic point cloud. 
Each Gaussian is characterized by its mean $\mathbf{m}_i \in R^3$, 
covariance matrix $\mathbf{C}_i$, density $\rho_i\in R$, and color $\mathbf{c}_i\in R^3$.

The covariance matrix $\mathbf{C}_i$, a $3\times 3$ matrix, 
determines the Gaussian's shape. 
It can be computed from a standard Gaussian through a combination of scaling and rotation transformations. 
The scaling transformation, represented by $s_i\in R^3$, indicates the scaling factor along each axis. 
The rotation transformation, represented by a normalized quaternion $q_i\in R^4$, signifies the rotation.

In practice, the gaussian color is computed using Spherical Harmonics (SH) coefficients. 
This approach effectively captures the anisotropic properties of the radiance field across different directions.
For D degree SH, the color $\mathbf{c}_i$ is represented as a vector of $(D+1)^2$ coefficients.