\begin{frame}
    \frametitle{LuisaCompute}
    \begin{figure}
        \includegraphics[width=0.9\linewidth]{fig_diagram_lc.jpg}
        \caption[short]{LuisaCompute Architecture} \cite{zhengLuisaRenderHighPerformanceRendering2022}
    \end{figure}
\end{frame}

\begin{frame}
    \frametitle{Functions in LC}
    \begin{enumerate}
        \item \textbf{Kernel}: The entry function of device.
        \item \textbf{Callable}: The function that can be called by Kernel.
        \item \textbf{Inline Function}: The function that can be called by Kernel, but will be inlined in Kernel.
    \end{enumerate}
\end{frame}

\begin{frame}
    \frametitle{Capture}
    \begin{itemize}
        \item  \textbf{Variable Capture}: Capture the value of host variable. For basic types, the value is the runtime value of the variable. For complex types, the value is the runtime value of the pointer.
        \item  \textbf{Resource Capture}: Capture the handle of host resource. LC will pass the handle of the resource to device implicitly.
    \end{itemize}


\end{frame}

\begin{frame}
    \frametitle{Multi-stage Code Generation}
    \begin{enumerate}
        \item C++ Preprocessor
        \item C++ Template
        \item C++ Runtime AST Generation
        \item AST to Device Code Generation
    \end{enumerate}
    \begin{quote}
        Stage 1 and 2 are common C++ code generation stages. Stage 3 and 4 are unique to LC. With LC's \textbf{C++ embedded DSL} (C++ Embedded Domain Specific Language), SPHerePackage could increse its ability of code reuse and modularity.
    \end{quote}
\end{frame}

\begin{frame}
    \frametitle{Just In Time Compilation}
    \begin{enumerate}
        \item Runtime Compilation
        \item Cut-off Branches
        \item Optimize Register Allocation
        \item Reduce Memory Barrier without decrease code reuse
    \end{enumerate}
\end{frame}

\begin{frame}
    \frametitle{Command Reordering}
    \begin{enumerate}
        \item Stream Agnostic
        \item Reduce Synchronization
        \item Let User Decide When to Execute
    \end{enumerate}
\end{frame}

\begin{frame}
    \frametitle{LC helps the optimize your code on the backend}
    These principles are quoted from \textit{https://gpuopen.com/learn/rdna-performance-guide/}
    \begin{enumerate}
        \item assure the number of allocators greator than the thread recoding
        \item try to reuse the same allocator
        \item minimize the Amount of Command Buffer Submission to GPU
        \item use pipeline cache to reuse PSO 
        \item create static large memory heaps and sub-allocated from the heap
        \item ...
    \end{enumerate}

\end{frame}