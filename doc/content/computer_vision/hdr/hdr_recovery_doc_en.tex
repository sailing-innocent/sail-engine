Classic HDR Recovery Methods directly accumulates 
multiple LDR images taken from a fixed camera pose 
to recover HDR image, which are prone to ghosting artifacts
because of dynamic scenes or camera motion.

To cope with this limitation, several methods suggests
to use a set of LDR images taken from different camera poses
to recover HDR image using image alignment techniques, 
such as image wraping and optimal flows.

However, they still suffer from large camera motion and 
occlusion due to the imperfect alignment.

Learning-based approach required ground truth.

Recent stutides adopt the differential rendering framework, suggests 
to recover HDR image by calibrating the entire HDR radiance field.