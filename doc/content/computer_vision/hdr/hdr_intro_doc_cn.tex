\paragraph{高动态范围 HDR 简介}

一般的图片被保存为 $w\times h\times 3$个0-255之间的整数,这被称为低动态范围(LDR),原因在于这种表示方式,对于每一个颜色通道
只有256种可能的状态,最亮的部分和最暗的部分最大差距也只在255倍,但在真实的世界中,
尤其是涉及到天光的自然风景中,一张照片的亮处和暗处的对比可能达到上千倍,
这就导致如果我们用LDR来表示这种图片,就会出现亮处细节丢失,暗处细节丢失的问题,这就是所谓的动态范围不足。

在摄影过程中,感光底片上的光强由光圈大小和曝光时间决定。
当曝光时间较长时,感光底片上的光强较大,这时候我们可以看到更多亮处的细节,但是如果光强过大,
就会导致感光底片上的光强超过了感光底片的容量,这时候就会出现过曝现象,即亮处细节丢失。
相反,当曝光时间较短时,感光底片上的光强较小,这时候我们可以看到更多暗处的细节,但是如果光强过小,
就会导致感光底片上的光强低于感光底片的容量,这时候就会出现欠曝现象,即暗处细节丢失。
为了解决这个问题,我们可以通过合成多张曝光时间不同的照片,然后将这些照片合成一张高动态范围的照片,这就是所谓的高动态范围(High Dynamic Range, HDR) 

\paragraph{经典的高动态范围图像合成}