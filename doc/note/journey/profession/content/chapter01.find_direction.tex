高中之前的课业基本没有对我后来的专业有太多直接的帮助。而来自于小时候订阅的《少年电脑世界》,对网文的痴迷,一点点的绘画基础,高中时候每期都看的《科学美国人》,让我在高中毕业时候给自己的专业定下的方向是增强现实AR, 人工智能AI和机器人Robotics,严格来说这些方向选的其实还不错,都是很有前景的方向。

但是同样因为对于专业的不了解,且无人指点,导致后来走了很大的一段弯路。2017年我通过高考被浙江大学竺可桢学院录取为交叉创新平台控制和机电双学位。算是我的专业开端,虽然分数不算低,但是因为合肥总体上还算是个二线城市,我的家庭也都在医疗行业,对于我的未来无法提出比较有效的建议。但无论如何,我的专业旅程都可以算作在这一年开始。

当时选择的时候我并不了解这个专业,也不知道自己真正喜欢的是什么。以至于我本科阶段的专业旅程并不顺利。大三时候一度对于专业课感到绝望。正因为没有兴趣,也没有钻研的目标,所以最终的成果自然也可以想见。

2019年春夏C++面向对象课程中,组队的计算机学院学长用learnopengl开发了一款小游戏。但是当时我并没有意识到这背后是什么,只觉得learnopengl特别难啃,图形编程的思路和之前学习的内容似乎截然不同(这主要归功于我大一的C语言没有学懂)。于是C++也没有学会,一度放弃。

本专业的主要软件都是matlab,mathematica,solidworks以及ansys等,编程的基础只能依靠自己去点,当然这个过程并不好受,大二时候才开始学会python,大三下时候看见了图形学,才开始重新尝试C,本科的毕业设计也是尝试用taichi来写,整个本科时间似乎也没有写过超过千行的代码,也没有写过一个完整的项目,这些短板在后来会耗费我大量的时间。

寻找方向的道路是极其痛苦的,因为我并不知道自己真正喜欢的是什么,也不知道自己真正擅长的是什么,更不知道想要走上这条路需要准备什么,所谓的聪明在这种程度的信息差之前,微不足道。直到2020年的春天,因为疫情在家的我在B站上看到了GAMES101,才突然意识到,对,我要做这个。

但是当时只有一个模糊的想法,虽然这个想法在后来被不断地强化与验证。比如秋招寻找工作的时候看到米哈游招聘的物理模拟研究员,后悔充斥着内心,发觉自己应该在这个方向早做准备。

本科的最后阶段就是在挣扎和纠结中度过,期间虽然做了一些小项目,比如使用python调用yolo识别电梯上的摔倒,使用python+javascript编写自动复制粘贴的程序。终于开始入门了编程,可惜一切看起来有点太迟了,

2020年秋冬我在迷茫中试图考研计算机学院,自学408,不甚了了。考研自然是没有通过,尤其是专业课分数特别低,2021年7月我理应毕业,但是因为之前乱修学分带来的后果,导致我最终没有能够正常毕业。这一切毫无疑问切断了我的人生进程。从此我仿佛成了一叶小舟,没有准备地冲向风浪滔天的大海。

很多时候我在想,小地方出来的人和我大学期间遇到的那些游刃有余的同学,区别究竟在哪,最后意识到其实这就跟开放世界游戏的一周目和二周目的区别,有前辈经验的加持,很多人的游戏其实是二周目,知道主线在哪,知道资源在哪,知道坑在哪。而一周目的时候,世界是一片迷雾,小小的困难也好像难过了天,就好像雷兽山上那只红皮人马一样,按理说是最菜的人马,但是却是最难打的。看似相同的禀赋,相同的起点,但要达到同样地步,花费的时间却是指数级的差距,甚至到时机错过,终身无法挽回。

