2021年的夏天我在紫金港旁边租了个房子,经过多轮煎熬之后得知自己只需要补选一门机电控制课程,再通过一场考研本专业即可(结业生不能跨专业,只能在本专业中)。一切显得那么讽刺,我之前有多么讨厌本专业,现在被迫重新考本专业的研究生就有多痛苦。

更何况,就算把眼前的关卡度过去之后呢,我难道继续在控制这块混吗?绝望之中我还报名了一个字节跳动举办的前端青训营,当时也算是没有办法的办法。

2021年的十月我青训营拿到了优秀和面试的推荐,但是十一月很快又被面试狠狠教做人。十一月终于把课程实验和期末考试都搞定了,返回合肥准备考研。我依然还记得那个寒冷的冬天,清晨出门时候路上还有冰面。午饭买了考场旁边的红薯,答题时手都在颤抖。

考完之后感觉发挥得也不是很好,一月在原神中度过,二月查分时候对比往年录取成绩更是无望。于是继续抱着前端来找工作。所幸字节跳动的竟然连过了两轮面试。三月初返回学校提交结业换证申请,当时的人已经处在一种麻木的状态,一轮一轮地被波折追赶。在紫金港提交材料发现有一个材料需要去玉泉找老师盖章,当时下午四点,我硬生生骑了一个多小时的车,在下班前去堵机械学院的老师,因为当时甚至都很难承受再次申请疫情同行码的不确定性,老师看到我时候眼神都是震惊的,她大概觉得正常人都应该明早再来。

后来意外发现居然有扩招,自己还是有复试的资格。于是临时取消了后续旅行的行程,返回家中准备最后的字节三面和考研复试。因为结业的缘故,我还需要加试两门专业课,事前完全不知道会考什么科目,
一周前才拿到科目名字,也没有任何参考资料,我记得是一门信号与系统,一门人工智能。加试考完之后的第二天紧接着就是面试。

我已经不记得那段日子是怎么度过的,回忆时候只剩下永远灰暗的天色。清明之后和母亲去杭州周边的一座小山旅行,因为当时还是疫情,人很少,在旅馆里突然看到了复试结果名单,当时母亲比我还要激动,仿佛终于透出了一点天光。

2022年的四月我入职字节跳动,在那里写了三个月的前端,算是彻底掌握了编程的方法。五月中旬有一段时间因为产品节奏的问题,导致我有一段时间空闲,我趁机用研究WebGL的借口,再次开启了图形学的尝试。六月偶然看到学校通知,知道自己竟然还可以申请竺可桢学院的荣誉证书,颠簸了一年,这份我原本以为永远不会有的荣誉竟然又回到了我的手上,五味杂陈。
再次回到学校时候,感觉自己的内心有一部分被点燃了。

我觉得好像还是应该做一些自己的事情,去看图形学,去学C++,去做那些以为自己不可能做到的事情。于是我提了离职。七月离职之后,便开始了自己的学习和积累。

等到2022年九月入学了南大之后,觉得自己终于有了一段很长的时间,于是下定决心,重新学习C++,Vulkan, CUDA,OpenGL,自己在那个小阁楼里一周一周地啃,11月又侥幸获得了D5渲染器的实习机会,认识了一些对我道路有深刻影响的人,参加了CSIG的流体模拟比赛,终于有了一点实际的C++编程体会。2023年末把自己的开题定为“基于可微渲染的新视角生成方法”,对着Gaussian Splatting死磕了大半年,2024年4月,终于觉得自己的基础开始变得牢固。

回过头去,这与发轫之初,2020年看到GAMES的那一刻,已经过去了整整四年时光,疫情,毕业,结业,考试,互联网大厂,一切回过头去,都好像梦一样。

\section{GAMES}
GAMES很大程度上改变了我的人生。

\subsection{GAMES101}

主要介绍了计算机图形学的基础概念,着重讲了渲染

计算机图形学是在研究使用计算机显示图像过程中逐渐发展起来的学科,其核心是渲染,传统上指如何得到用三角面定义的场景模型在指定相机参数下生成的真实感图片。随着学科的发展,也逐步引入动画(基于关键帧的人物动画和基于物理的仿真模拟)几何处理,人工智能等领域的知识。

\paragraph{计算机图形学的发展历史}

\paragraph{计算机图形学的应用}

\paragraph{计算机图形学与计算机视觉}
图形学经常与视觉搞混,两者都是利用计算机算法对物理世界的建模。

但图形学(渲染)建模的是图片形成之前的阶段,主要是光线与物体表面交互,光在空气中传输,光最终映射到人眼形成图像。此外图形学还有很多和图像无关的部分,比如角色/物理动画,几何处理等。

而计算机视觉(图像理解,重建)建模的是图像形成之后人脑如何对图像进行理解的过程,从图像中提取特征,识别特定的对象和理解抽象的概念等等。同时,计算机视觉还涉及很多单纯图像的变换,比如图形的滤波,降噪,超分辨率,形变等。

换言之,计算机图形学的兴趣在于物理世界,而计算机视觉的兴趣在于人的视觉理解系统,两者虽然互有涉猎,且在色彩模型和相机模型上有共通知识,但归根到底是不同的学科范畴。

\begin{figure}[H]
    \centering
    \input{tikz_diff_cg_cv}
    \caption{Difference of CG and CV}
    \label{fig:diff_cg_cv}
\end{figure}




\section{Interactive Computer Graphics}
这是我2022年5月在字节开始,借着WebGL学习图形学 

\begin{figure}[H]
    \centering
        \includegraphics[width=\textwidth]{2022-05-16-interactive-computer-graphics.png}
        \caption{Interactive Computer Graphics Demo: the 3D Gasket}
    \label{fig:interactive_computer_graphics}
\end{figure}


\section{Learn OpenGL}

