\paragraph{第一步2017~2023:选择方向}

2017年我通过高考被浙江大学竺可桢学院录取为交叉创新平台控制和机电双学位。算是我的专业开端,虽然分数不算低,但是因为合肥总体上还算是个二线城市,我的家庭也都在医疗行业,对于我的未来无法提出比较有效的建议。但无论如何,我的专业旅程都可以算作在这一年开始。

当时选择的时候我并不了解这个专业,也不知道自己真正喜欢的是什么。以至于我本科阶段的专业旅程并不顺利。大三时候一度对于专业课感到绝望。正因为没有兴趣,也没有钻研的目标,所以最终的成果自然也可以想见。

2019年春夏C++面向对象课程中,组队的计算机学院学长用learnopengl开发了一款小游戏。但是当时我并没有意识到这背后是什么,只觉得learnopengl特别难啃,图形编程的思路和之前学习的内容似乎截然不同(这主要归功于我大一的C语言没有学懂)。于是C++也没有学会,一度放弃。

2020年寒假疫情在家期间,我又重新通过GAMES101接触了计算机图形学,我惊觉这其实就是我一直想要找的领域。但是此时我无奈发现自己已经积重难返,本科就剩下一年的时间,无论是编程,数学,杂事,专业上都远远来不及。更何况还有疫情封控的影响,寻找未来出路的压力,当时一度陷入崩溃和迷茫。

本科的最后阶段就是在挣扎和纠结中度过,期间虽然做了一些小项目,比如使用python调用yolo识别电梯上的摔倒,使用python+javascript编写自动复制粘贴的程序。终于开始入门了编程,可惜一切看起来有点太迟了,

2020年秋冬我在迷茫中试图考研计算机学院,自学408,不甚了了。考研自然是没有通过,尤其是专业课分数特别低,2021年7月我理应毕业,但是因为之前乱修学分带来的后果,导致我最终没有能够正常毕业。这一切毫无疑问切断了我的人生进程。从此我仿佛成了一叶小舟,没有准备地冲向风浪滔天的大海。

2021年的夏天我在紫金港旁边租了个房子,经过多轮煎熬之后得知自己只需要补选一门机电控制课程,再通过一场考研本专业即可。一切显得那么讽刺,我之前有多么讨厌本专业,现在就有多痛苦,更何况就算把眼前的关卡度过去之后呢,我难道继续在控制这块混吗?绝望之中我还报名了一个字节跳动举办的前端青训营,当时也算是没有办法的办法。

2021年的十月我青训营拿到了优秀和面试的推荐,但是十一月很快又被面试狠狠教做人。十一月终于把论文和期末考试都搞定了,返回合肥准备考研。我依然还记得那个寒冷的冬天,清晨出门时候路上还有冰面。午饭买了一中旁边的红薯,答题时手都在颤抖。

考完之后感觉发挥得也不是很好,一月在原神中度过,二月查分时候对比往年录取成绩更是无望。于是继续抱着前端来找工作。所幸字节跳动的竟然连过了两轮面试。三月初返回学校提交结业换证申请,意外发现居然有扩招,自己还是有复试的资格。于是临时取消了旅行的行程,返回家中准备最后的字节三面和考研复试。因为结业的缘故,我还需要加试两门专业课,事前完全不知道会考什么科目,一周前拿到科目名字也没有任何参考,我记得是一门信号与系统,一门人工智能。加时考完之后的第二天紧接着就是面试。我已经不记得那段日子是怎么度过的。回忆时候只剩下了灰暗的天色,以及我在卧室和书房前的情景。

2022年的四月我入职字节跳动,在那里写了三个月的前端。算是彻底掌握了编程的方法。五月中旬有一段时间因为产品节奏的问题,导致我有一段时间空闲,我趁机用webgl再次开启了图形学的尝试。七月离职之后,便开始了自己的学习和积累。

等到2022年九月入学了南大之后,觉得自己终于有了一段很长的时间,于是下定决心,重新学习C++,11月又侥幸获得了D5渲染器的实习机会,认识了一些对我道路有深刻影响的人,2023年末把自己的开题定为“基于可微渲染的新视角生成方法”,终于才逐渐驶入正轨。

\paragraph{程序的开始}
这一阶段主要对应我从2021年八月开始到2022年七月结束的前端之旅。这个过程让我真正成为一名职业的程序员。当然回过头来会发现好像其实没有剩下太多值得记录的东西,因为很多比如vscode, git, github, 命令行,工程经验之类的东西是没有办法留下系统记录的,这些已经成为了我能力的一部分。就好像一般人也很难说清它到底如何学会了呼吸,学会了走路。

\paragraph{第二步2024~2030:积累基础}

积累基础ING 