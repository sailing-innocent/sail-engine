
\section{模块化}
\subsection{SPHerePackage}
Luisa Compute的Device端代码编译非常灵活(Multi Stage),但对于大型协作项目而言我们仍需要特定的规则来规范封装,提高代码可读性、复用性和接口统一性。
目前为止\textbf{SPHerePackage}包含以下内容:
\paragraph{包环境}
\begin{itemize}
	\item PackageGlobal,一个thread\_local静态类,是当前线程的包环境上下文。
	\item PackageEnvScope,一个Guard类,他的构造与析构确定了一个特定的包环境。构造时,向PackageGlobal push当前环境,析构时,恢复PackageGlobal原有环境,若不使用PackageEnvScope 则始终处于默认的环境。
	\item Package基类,用于规范行为封装和提供辅助代码,所有的包开发均继承自此基类。
\end{itemize}

\paragraph{包类型}
\begin{enumerate}
	\item Method Package:为用户提供Callable的包,本身无需Compile,只有AST Gen阶段。
	\item Routine Package:为用户提供Kernel的包,用户提供符合要求的输入数据并调用此类包提供的Kernel以获取结果(如本项目实现的并行原语库),此类包需要进行Compile才可使用。
	\item Module Package:为用户提供Kernel/Callable的包,与1/2中不同的是,此类包自己维护一份私有的数据结构(占有Device端资源),用户通过此包提供的接口来与此包的算法进行交互,SPHere中典型的Module Package有:BVHCollisionDetection (带资源的Kernel)、SDF(带资源的Callable)。
	\item Inline Package: 为用户提供内联展开代码段的包,用户直接调用对应的C++函数,在Callable、Kernel中展开对应代码,SPHere中典型的Inline Package有:并行语义库ParallelSemantic$^2$,此类包本身无需Compile 和AST Gen,他的AST Gen阶段与调用他的Callable、Kernel的AST Gen时机一致。
\end{enumerate}
注$^2$:其中grid\_stride\_loop就为典型的内联展开代码段,他所在的Package并没有完全遵循SPHere Package实现,没有继承自Package基类,但行为和上述设计吻合。为了书写便利,我们将他实现为一个全局的Inline包。

\paragraph{包规范}
\begin{itemize}
	\item \textbf{config}:每个包必须有config阶段,用于决定包内代码在AST Gen阶段的捕获行为。
	\item \textbf{astgen}:Callable,Kernel一般可以通过懒生成的方式来完成AST Gen,Package基类提供的对应函数为\hl{lazy\_astgen()}和\hl{lazy\_call()}
	      \item\textbf{ compile}:对于Routine Package与Module Package,存在compile阶段,可以使用懒编译(当用户调用特定接口时再进行编译),Package基类提供的对应函数为\hl{lazy\_compile()}。一般而言,Routine Package与含有Callable的Module Package可以将astgen和compile阶段同时进行,直接生成最终Shader。
\end{itemize}
值得注意的是,所有可能导致捕获值改变的行为都通过config来实现,例如,要求用户使用如下接口来改变捕获行为:
\begin{lstlisting}[language=C++]
	Option option{
		...
	};
	package.config(option);
\end{lstlisting}
\paragraph{包管理}
为了提高包在系统中的复用率,需要部分包能够交叉引用,而不是树状引用。
\begin{itemize}
	\item Callable Package、Routine Package可以被管理和复用,Module Package不可以被复用,因为后者带有特定资源。
	\item \textbf{SPHerePackage}的包管理维护了一个\hl{hash\_map\textless Description, Package\textgreater}。 任意一个包需要能被Description唯一描述,符合描述的包将在\hl{PackageManager::require(Description)}时返回给用户,若没有满足描述的包则创建对应的包,并返回给用户。
	\item 包之间可以相互依赖,这种依赖也通过\hl{require}实现,调用\hl{require}的时机为\textbf{config}阶段。
\end{itemize}

\subsection{调用形式}
\paragraph{Callable Package}可以分为两类:
\begin{enumerate}
	\item 【Module】带资源的,即此类包需要向device申请如Buffer、Image等资源,并将他作为包内部的数据结构,允许用户通过包给出的接口对包内部数据进行访问。对应图\ref{fig:CallablePackage}中的Package A。
	\item 【Method】不带资源的,即此类包不需要向device申请任何资源,此包提供的所有函数均在用户提供的资源上进行操作,或仅为工具函数。对应图\ref{fig:CallablePackage}中的Package B。
\end{enumerate}
对于Callable Package,用户会首先对包进行config和astgen,随后在他的Kernel中使用。如图\ref{fig:CallablePackage}所示:

\begin{figure}[H]
	\centering
	\includegraphics[keepaspectratio,width=0.4\linewidth]{fig_callable_package_csig2023.png}
	\caption{Callable Package}
	\label{fig:CallablePackage}
\end{figure}

\paragraph{Kernel Package} 可以分为两类:
\begin{enumerate}
	\item 【Module】带资源的,即此类包需要向device申请如Buffer、Image等资源,并将他作为包内部的数据结构。对应下图中,Input Buffer$\rightarrow$Viewer路径。
	\item 【Routine】不带资源的,即此类包不需要向device申请任何资源,所有资源由用户申请并提供给此包。对应下图中,Input Buffer$\rightarrow$Output Buffer路径。
\end{enumerate}

\begin{figure}[H]
	\centering
	\includegraphics[keepaspectratio,width=0.4\linewidth]{fig_module_package_csig2023.png}
	\caption{Module Package}
	\label{fig:Module Package}
\end{figure}