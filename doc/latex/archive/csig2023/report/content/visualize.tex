\section{可视化}

可视化与仿真引擎的共同纽带为粒子位置,可视化引擎会维护深度信息,必要的模型重建信息和光照等其他渲染信息。
而模拟引擎则只会关心粒子本身的性质和约束。SPHere的可视化方案需要支持平时迭代的调试任务和最终的演示任务,
同时对于流体模拟粒子的仿真渲染,尤其是基于GPU的并行渲染,在业界并没有一个成熟统一的方案,
因此我们需要一个尽可能通用的架构来支持各种可能的可视化工具,方便持续实验和迭代。

\subsection{可视化架构}

\begin{figure}[H]
	\centering
	\includegraphics[width=\linewidth,keepaspectratio]{fig_visualizer_framework_csig2023.png}
\end{figure}

如图所示,整体的可视化架构分为GUI, Visualizer和Painter三层:

\begin{itemize}
	\item GUI负责外部展示仿真画面,接受处理用户输入并转换为配置信息,控制整个更新流程。
	\item Visualizer内部维护一个swapchain,将注册的Painter维护为一个painter list并在渲染更新过程中依次调用每个painter的paint方法在display buffer上渲染图像,并最终将display buffer同步到前台
	\item Painter负责具体的可视化算法,从模拟引擎中获得粒子位置信息,从场景配置中获得光源等场景信息,并从GUI获得相机位置和方向的更新信息。
\end{itemize}

得益于Painter可拆卸的架构,我们可以从最简单的sprite particle方案逐步迭代到最终兼顾灯光,场景,美术效果的可视化方案上。
这样就既可以支持模拟算法的同步开发与debug,也可以支持最终形成一个完整的演示。

Visualizer只有两类,BaseVisualizer和SceneVisualizer,前者是一个简单的2D绘制框架,
后者与前者的区别在于增加了相机信息,从而可以支持3D绘制方法。具体可以参考src/packages/visualizer/

Painter则相对多样,主要是各种算法的实验。我们也希望模拟算法的开发者可以实现自己的painter。在最终的演示中,我们展示了三种可视化方案:

\begin{itemize}
	\item 光栅化方案 Raster
	\item 屏幕空间流体 Screen Surface Fluid, SSF
	\item 体渲染方案 Volume Rendering
\end{itemize}

\subsection{光栅化方案}

我们按照SPHere项目的模式封装了LuisaCompute自带的Raster功能,使用Point模式来绘制流体位置,
将流体的位置buffer和手动生成的index buffer一起注册为mesh对象每次drawcall同步。

利用共用深度检测的方法来同时光栅化流体,布料,SDF等对象,并最终实现了如图所示的效果:

\begin{figure}[H]
	\centering
	\includegraphics[width=\linewidth,keepaspectratio]{fig_raster_csig2023.png}
\end{figure}

不难发现,这种方法的优点在于可以同时光栅化支持多种对象,适合在开发过程中进行快速地展示预期效果,帮助调试。

但是缺点在于重建流体表面的方法过于昂贵,无法实时更新大量流体的表面,因此无法拓展到更加复杂的光影渲染。

\subsection{屏幕空间流体}

基于屏幕空间的流体渲染方案(Screen Surface Fluid)主要流程如下

\begin{itemize}
	\item 通过相机位姿标定最终会投影到屏幕上的粒子,取最小值和最大值的方法得到流体的深度(depth)和厚度(thickness)
	\item 将获得的深度和厚度值存储在一个和屏幕分辨率相同大小的G-buffer中,称为depth buffer和thickness buffer
	\item 此时获得的深度噪声很严重,我们需要对其进行一个双边滤波(bilateral filter)来去除噪声
	\item 最后根据滤波后的depth buffer来重建normal,具体来说,对于任意一个点,我们得到其周围的点,$\vec{n}=\vec{ddx} \times \vec{ddy}$
	\item 通过normal我们可以指定反射和折射率,从而最终得到该像素的颜色
\end{itemize}

屏幕空间方法的优点在于计算快捷方便,能够快速得到结果,如图所示:

\begin{figure}[H]
	\centering
	\includegraphics[width=\linewidth,keepaspectratio]{fig_ssf_method_csig2023.png}
\end{figure}

但是不难发现,这种方法只能处理凸包的流体,不能处理存在凹陷,重叠区域的流体,比如瀑布的粒子。

同时因为经过NDC映射的depth buffer本身不是一个正交空间,在这个空间上进行双边滤波和法线重建的结果并非物理正确。

\subsection{体渲染方案}

体渲染方法可以认为是从传统基于物理的渲染方法(physically based rendering)中对于介质(Tranmittance)的处理单独抽出形成的算法。

经典的体渲染策略比如光线步进(ray marching)方法主要流程如下:

\begin{itemize}
	\item 从相机中心向每个像素发射一条光线
	\item 得到光线与介质的入射点和出射点
	\item 从出射点向入射点,取某个很小的距离步,每一步计算出介质的透射率,累积得到最终的透射率
	\item 在整个光线上对积累的样本做积分,得到最终的像素颜色
\end{itemize}

其中对于最终透射率的计算基于Beer Lambert's Law: $T=\exp(-d\sigma)$

同时,为了积累光线的影响,在每个采样点,需要额外向光源发射一根射线,用同样的方法计算其从光源积累到该采样点的光源本身颜色。

在具体的实现上,为了减少复杂度,我们同样参考屏幕空间方法,将整个渲染过程分为预渲染和渲染两步。

预渲染过程中,我们分别从相机中心和光源位置发射光线,得到从相机中心与流体的入射点和出射点,从光源与流体的入射点与出射点。

为了构建简单的焦散效果,我们在处理地面的时候,如果判断其在光源G-Buffer所覆盖的区域中,则会根据流体介质影响的程度来对背景着色。

最终,进入渲染过程时,我们会重新从相机中心向每一个像素点发射光线,从depth buffer中得到入射点,从far depth buffer中得到出射点,而对于中间介质进行采样和积分。采样的过程会利用背景色和从光源重建的light depth buffer。

最终的效果如下:

\begin{figure}[H]
	\centering
	\includegraphics[width=\linewidth,keepaspectratio]{fig_volume_method_csig2023.png}
\end{figure}

\subsection{GUI方案}
LuisaCompute内部虽然用glfw封装了一个简单的窗口用于展示,
可是并没有可以支持各种控制信号的GUI实现。而因为后端并不固定,所以使用imgui也比较繁琐。
因此经过调研和尝试,我们最终采取QT作为GUI窗口实现的方案。

QT本身提供了类型反射,窗口绑定等功能。我们只需要将QT-widget自带的winId
(一个unsigned long long句柄),提供给LuisaCompute用来创建swapchain,
就可以完全按照一般的模式来渲染场景,并且利用qt的信号机制来传递消息,
侵入性降低到了最小。

\begin{figure}[H]
	\centering
	\includegraphics[width=\linewidth,keepaspectratio]{fig_gui_csig2023.png}
\end{figure}
上图展示了我们的GUI设计,具体展示为:
\begin{itemize}
	\item MainWindowQT: 继承自QMainWindow对象,GUI的主要窗口
	\item GUIWidget: 继承自QWidget,GUI的中心widget,有下面两个主要的子模块
	      \begin{enumerate}
		      \item Canvas: 用于展示仿真画面的可视化部分
		      \item Control Panel: 用于控制参数,重复查看效果
	      \end{enumerate}
\end{itemize}
同时我们实现了如下操控功能:
\begin{itemize}
	\item 仿真流程控制,利用空格暂停/继续,鼠标右键暂停/继续,pause按钮暂停/继续,r重置仿真,鼠标左键重置仿真,reset按钮重置仿真
	\item 相机控制,使用WASD移动,QE上下,方向键上下左右转动视角
	\item 控制sph仿真参数,如dx, dt, alpha, stiffB等
	\item 切换边界约束和可视化:除了标准的题目场景外,我们还支持了多种不同的场景,比如瀑布场景,与布料的耦合,基于体渲染的可视化方案等,这些都可以通过GUI进行切换展示。
\end{itemize}
同时为了能够让多个方法进行同一个控制流的对比,我们还实现了对比视图:
\begin{figure}[H]
	\centering
	\includegraphics[width=\linewidth,keepaspectratio]{fig_two_window_csig2023.png}
\end{figure}
如图上所示,左边展示的是原始的WCSPH,右侧是我们使用的PCISPH,两者共用一套参数和控制流,可以非常方便地对比。这个例子也展示了我们GUI方案良好的可扩展性。
可视化模块的实现源码在src/utils/qgui中。