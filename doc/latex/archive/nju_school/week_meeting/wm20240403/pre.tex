\documentclass{njupre/njupre}
\title[组会报告]{ 组会报告 }
\author[朱子航]{\texorpdfstring{朱子航 \\ \smallskip \textit{522022150087@smail.nju.edu.cn}}{}}
\date[\today]{\texorpdfstring{2023-03-28}{}}
\begin{document}
\begin{frame}
    \titlepage
\end{frame}

\begin{frame}
    \frametitle{目录}
    \tableofcontents
\end{frame}

\begin{frame}[allowframebreaks]{Reference}
    \bibliography{ref}
    \bibliographystyle{plain}
\end{frame}

\begin{frame}
    \frametitle{解决了一个困扰了很久的复现bug}
    \begin{figure}
        \centering
            \begin{subfigure}{0.48\linewidth}
                \includegraphics[width=\textwidth]{fig_debug_before_20240403.png}
                \caption{}
            \end{subfigure}
            \begin{subfigure}{0.48\linewidth}
                \includegraphics[width=\textwidth]{fig_debug_after_20240403.png}
                \caption{}
            \end{subfigure}
    \end{figure}
\begin{quote}
    原因是在使用GPU编程Shared Memory加速时,需要在加载SharedMemory前后进行同步,否则会造成线程竞争导致数据错误
\end{quote}
\end{frame}

\begin{frame}
    \frametitle{Best Reproduction}
    \begin{figure}
        \includegraphics[width=\textwidth]{fig_best_reprod_20240403.png}
        \caption{The Reproduction Result}
    \end{figure}
\end{frame}

\begin{frame}
    \frametitle{对不同参数组合探索}
    重启了对不同参数组合的3DGS训练效果的实验
    对比不同的参数组合对训练效果的影响
\begin{table}
    \input{tab_3dgs_profile_diff_param_psnr_result_nerf_blender}
\end{table}


\end{frame}

\begin{frame}
    \frametitle{有限几张图的情况下GS效果}
    \begin{figure}
        \begin{subfigure}{0.18\linewidth}
            \includegraphics[width=\textwidth]{"fig_sparse_gs_train_0.png"}
        \end{subfigure}
        \begin{subfigure}{0.18\linewidth}
            \includegraphics[width=\textwidth]{"fig_sparse_gs_train_1.png"}
        \end{subfigure}
        \begin{subfigure}{0.18\linewidth}
            \includegraphics[width=\textwidth]{"fig_sparse_gs_train_2.png"}
        \end{subfigure}
        \begin{subfigure}{0.18\linewidth}
            \includegraphics[width=\textwidth]{"fig_sparse_gs_train_3.png"}
        \end{subfigure}
        \begin{subfigure}{0.18\linewidth}
            \includegraphics[width=\textwidth]{"fig_sparse_gs_train_4.png"}
        \end{subfigure}
        \caption{Gaussian Splatting on Sparse Image dataset}
    \end{figure}
    训练之后的结果如图

    \begin{figure}
        \begin{subfigure}{0.38\linewidth}
            \includegraphics[width=\textwidth]{"fig_sparse_gs_train_result_0.png"}
        \end{subfigure}
        \begin{subfigure}{0.38\linewidth}
            \includegraphics[width=\textwidth]{"fig_sparse_gs_train_result_1.png"}
        \end{subfigure}
        \caption{Gaussian Splatting Train Result}
    \end{figure}
\end{frame}

\begin{frame}
    \begin{quote}
        对于空间中的一个点$p=(p_1,p_2,p_3)$,我们可以写出它的齐次坐标
        $p_{hom}=(p_1,p_2,p_3,1)$,这时候相机坐标的(旋转、平移)变换可以表示为4阶变换矩阵$V_{4x4}$
        将空间坐标中的点变换为相机坐标中的点,进而通过投影变换$(x',y')=(x/z,y/z)$来投影到相机平面上。
    \end{quote}
    \begin{quote}
        如果我们已知同一个3D空间点在N个相机平面上的2D投影点坐标,就相当于我们能写出一2N个方程的线性方程组来
        求解3个未知量,我们可以用最小二乘法来求解其最接近的结果 $(X^TX)^{-1}X^TY$
    \end{quote}
\end{frame}

\begin{frame}
    \frametitle{用对极几何反解尝试}
    \begin{figure}
        \includegraphics[width=\textwidth]{fig_epi_iter_20240403.png}
        \caption{Epipolar Iteration}
    \end{figure}
    \begin{quote}
        使用pytorch的实现尝试,可以发现问题主要集中在很多不进行更新的点也被对极投影了,
        导致最终结果有很多噪点,但是几何信息是基本正确的。
    \end{quote}
\end{frame}


\begin{frame}
    \frametitle{用对极几何反解尝试}
    \begin{itemize}
        \item 使用它作为一个预训练方法,得到的比直接训练有增长13.32 -> 15.04, 19.12 -> 20.03 (NeRF Blender Lego Scene)
        \item 使用它作为一个中间方法和原本的prune混合在一起,增长结果不明显,可能是参数还没有调整好。
        \item 后续需要做更多的实验来验证,并与成熟方法对比。
    \end{itemize}
\end{frame}


\begin{frame}
    \frametitle{实现torch的projector,方便后续调试}
    \begin{figure}
        \includegraphics[width=0.5\textwidth]{"fig_inno_torch_20240403.png"}
        \caption{实现torch的projector,方便后续调试}
    \end{figure}
\end{frame}

\end{document}