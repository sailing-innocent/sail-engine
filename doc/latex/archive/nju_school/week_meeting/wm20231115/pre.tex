\documentclass{njupre/njupre}
\title[组会]{组会报告}
\subtitle{研究进展\today}
\author[朱子航]{\texorpdfstring{朱子航 \\ \smallskip \textit{522022150087@smail.nju.edu.cn}}{}}
\date[2023-11-15]{\texorpdfstring{2023-11-15 组会}{}}
\begin{document}
\begin{frame}
    \titlepage
\end{frame}
\begin{frame}
    \frametitle{目录}
    \tableofcontents
\end{frame}
\section{Gaussian Splatting实验部分}
\sectionframe{Gaussian Splatting 实验部分}
\subsection{NeRF Blender Synthetic 数据集}
\begin{frame}
\frametitle{NeRF Synthetic Dataset}
\begin{figure}[H]
    \includegraphics[width=0.9\linewidth]{demo_nerf_dataset.png}
    \caption[short]{NeRF Synthetic数据集}
\end{figure}
\begin{quote}
    该数据集由NeRF原文提出,主要是利用Blender软件
    构建了8个场景,然后通过随机选定相机视角,真实感渲染得到该视角下的渲染结果图片。
\end{quote}
\end{frame}

\begin{frame}
\frametitle{姿态和图片对}
\begin{figure}
    \centering
        \begin{subfigure}{0.48\linewidth}
            \includegraphics[width=\textwidth]{"fig_demo_lego_img_nerf_blender.png"}
            \caption{Lego Scene}
        \end{subfigure}
        \begin{subfigure}{0.48\linewidth}
            \includegraphics[width=\textwidth]{"fig_demo_lego_poses_nerf_blender.png"}
            \caption{Lego Poses}
        \end{subfigure}
\end{figure}
\end{frame}
\subsection{Mip360 数据集}
\begin{frame}
    \frametitle{COLMAP Dataset}
    \begin{figure}[H]
        \includegraphics[width=0.9\linewidth]{fig_demo_colmap_dataset.png}
        \caption[short]{COLMAP数据集}
    \end{figure}
    \begin{quote}
        COLMAP数据集是一类使用真实相机拍摄的数据集,通过经典的三维重建SfM方法进行前处理重建得到相机的位姿和对应的图片对。Mip360数据集就是采用这种方法得到了7个对应的高精度场景。
    \end{quote}
\end{frame}

\begin{figure}
    \centering
    \begin{subfigure}{0.48\linewidth}
        \includegraphics[width=\textwidth]{fig_demo_bicycle_poses_mip360.png}
        \caption{Mip360 Bicycle Poses}
    \end{subfigure}
    \begin{subfigure}{0.48\linewidth}
        \includegraphics[width=\textwidth]{fig_demo_bicycle_img_mip360.jpg}
        \caption{Mip360 Bicycle Images}
    \end{subfigure}
    \caption{Bicycle Scene}
    \label{mip360:dataset:bicycle}
\end{figure}

\subsection{Benchmark}
\begin{frame}
\frametitle{PSNR: Peak Signal Noise Ratio}

\begin{figure}
    \centering
    \begin{subfigure}{0.25\linewidth}
        \includegraphics[width=\textwidth]{"fig_gs_nerf_gt.png"}
        \caption{Ground Truth}
    \end{subfigure}
    \begin{subfigure}{0.25\linewidth}
        \includegraphics[width=\textwidth]{"fig_gs_nerf_res.png"}
        \caption{Result}
    \end{subfigure}
\end{figure}

\begin{quote}
    PSNR衡量了一张带有噪声的图片K和一张干净图片I之间的差异,单位是分贝(dB)
    $$MSE = \frac{1}{mn}\Sigma_{i=0}^{m-1}\Sigma_{j=0}^{n-1}[I(i,j) - K(i,j)]^2$$
    $$PSNR = 10 log_{10}(\frac{MAX_I^2}{MSE})$$ 
\end{quote}
\end{frame}

\begin{frame}
\frametitle{Gaussian复现结果}
\begin{table}
    \input{tab_gaussian_nerf}
\end{table}
\end{frame}

\section{其他进展}
\sectionframe{其他进展}
\subsection{与流体结合的动态场景}
\begin{frame}
\frametitle{与流体模拟相结合}
\begin{itemize}
    \item 流体模拟本身是基于拉格朗日视角的,所以自带有流体位置坐标
    \item 将每个流体位置作为Gaussian中心点,赋予Gaussian的协方差,透明度,颜色等参数就可以用Gaussian Splatting作为流体的实时后续渲染方法
    \item 整个过程是可微的
    \item 使用LuisaCompute框架重新实现了Gaussian Splatting,与之前的流体模拟框架结合
    \item demo见视频演示
\end{itemize}
\end{frame}

\subsection{添加光照效果的尝试}
\begin{frame}
\frametitle{增添光照效果}
\begin{itemize}
    \item 光源类型一般可以分为点光源,面光源,方向光源
    \item 光照和物体表面的交互过程一般被称为shading 
    \item 通常的shading方法需要考虑漫反射,镜面反射等因素,需要用辐射度量学来推导渲染公式,在反射点向各个方向的光源做积分
    \item Phong, Blin-Phone, Physically Based Rendering ..
\end{itemize}
\end{frame}

\begin{frame}
\frametitle{光照效果原型实现}
\begin{itemize}
    \item 因为光照和相机的对偶特性,我们可以先将所有的Gaussian点投影到面光源的表面上,得到每个高斯点在面光源表面上的权重
    \item 类似于通过权重来进行反向传播,把光源处的信息乘到Gaussian的固有色上
    \item 通过预计算改变Gaussian颜色之后,再接上原本的正向过程得到相机处渲染图片。
    \item 原型见视频,使用了一个40x40x40的立方体点云作为场景,让光源沿着物体周围做圆周运动并且照向物体
\end{itemize}
\end{frame}


\end{document}