\documentclass{njupre/njupre}
\title[组会报告]{ 组会报告 }
\author[朱子航]{\texorpdfstring{朱子航 \\ \smallskip \textit{522022150087@smail.nju.edu.cn}}{}}
\date[\today]{\texorpdfstring{2023-03-28}{}}
\begin{document}
\begin{frame}
    \titlepage
\end{frame}

\begin{frame}
    \frametitle{目录}
    \tableofcontents
\end{frame}

\begin{frame}[allowframebreaks]{Reference}
    \bibliography{ref}
    \bibliographystyle{plain}
\end{frame}

\begin{frame}
    \frametitle{简单MLP图像分类框架}
    重启了一个简单的使用MLP进行图像分类的实验,结果如下
    这是我使用多重感知机(Multi-Layer Perceptron)在图像分类问题中的应用报告。

\begin{figure}
    
\begin{tikzpicture}
	\begin{axis}[
		 title={Profile Img Classify MLP BS EP},
		 xlabel={x},
		 ylabel={y},
		 legend style={at={(0.97,0.5)},anchor=west},
		 legend entries={
simple mlp batchsize 32,
simple mlp batchsize 64
 },
		]
\addplot table {result_simple_mlp_batchsize_32_data.dat};
\addplot table {result_simple_mlp_batchsize_64_data.dat};
	\end{axis}
\end{tikzpicture}
        
    \caption{Simple MLP for Image Classification}
\end{figure}
\end{frame}

\begin{frame}
    \frametitle{对不同参数组合探索}
    重启了对不同参数组合的3DGS训练效果的实验
    对比不同的参数组合对训练效果的影响
\begin{table}
    \input{tab_3dgs_profile_psnr_result_nerf_blender}
\end{table}

% 经过观察,我们发现直接对点云进行位移优化效果似乎不显著,
% 所以我们尝试将position lr设置为0,只观察通过densify是否会让
% 最终重建的结果也达到几何标准。

% 在nerf blender数据集上进行训练的结果如图所示:

% \begin{table}
%     \input{tab_3dgs_profile_xyz_grad_psnr_result_nerf_blender}
% \end{table}

\end{frame}

\begin{frame}
    \frametitle{有限几张图的情况下GS效果}
    \begin{figure}
        \begin{subfigure}{0.18\linewidth}
            \includegraphics[width=\textwidth]{"fig_sparse_gs_train_0.png"}
        \end{subfigure}
        \begin{subfigure}{0.18\linewidth}
            \includegraphics[width=\textwidth]{"fig_sparse_gs_train_1.png"}
        \end{subfigure}
        \begin{subfigure}{0.18\linewidth}
            \includegraphics[width=\textwidth]{"fig_sparse_gs_train_2.png"}
        \end{subfigure}
        \begin{subfigure}{0.18\linewidth}
            \includegraphics[width=\textwidth]{"fig_sparse_gs_train_3.png"}
        \end{subfigure}
        \begin{subfigure}{0.18\linewidth}
            \includegraphics[width=\textwidth]{"fig_sparse_gs_train_4.png"}
        \end{subfigure}
        \caption{Gaussian Splatting on Sparse Image dataset}
    \end{figure}
    训练之后的结果如图

    \begin{figure}
        \begin{subfigure}{0.48\linewidth}
            \includegraphics[width=\textwidth]{"fig_sparse_gs_train_result_0.png"}
        \end{subfigure}
        \begin{subfigure}{0.48\linewidth}
            \includegraphics[width=\textwidth]{"fig_sparse_gs_train_result_1.png"}
        \end{subfigure}
        \caption{Gaussian Splatting Train Result}
    \end{figure}
\end{frame}

\begin{frame}
    \frametitle{用对极几何反解尝试,继续debug}
    \begin{figure}
        \begin{subfigure}{0.8\linewidth}
            \includegraphics[width=0.48\textwidth]{"fig_0_01_20240328.png"}
            \includegraphics[width=0.48\textwidth]{"fig_0_02_20240328.png"}
            \caption{Iter 1}
        \end{subfigure}
        \begin{subfigure}{0.8\linewidth}
            \includegraphics[width=0.48\textwidth]{"fig_4_01_20240328.png"}
            \includegraphics[width=0.48\textwidth]{"fig_4_02_20240328.png"}
            \caption{Iter 1}
        \end{subfigure}
        \begin{subfigure}{0.8\linewidth}
            \includegraphics[width=0.48\textwidth]{"fig_8_01_20240328.png"}
            \includegraphics[width=0.48\textwidth]{"fig_8_02_20240328.png"}
            \caption{Iter 1}
        \end{subfigure}
        \caption{Projector可以反解位置,但是因为权重问题不收敛}
    \end{figure}
\end{frame}

\begin{frame}
    \frametitle{实现torch的projector,方便后续调试}
    \begin{figure}
        \includegraphics[width=0.5\textwidth]{"fig_fig_torch_gs_20240328.png"}
        \caption{实现torch的projector,方便后续调试}
    \end{figure}
\end{frame}

\end{document}