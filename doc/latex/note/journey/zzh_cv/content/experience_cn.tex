\hspace{-2.4pt}\heading{\faSuitcase}{主要经历}
    \subcvevent{2023年12月}{2024年06月(预期)}{D5图形渲染实习生}{}{D5线上}{fig_d6_render.png}
    {2023年9月返回学校开题,12月后在D5的实习转为线上,参与新项目的开发,主要参与合并SakuraEngine和LuisaCompute开发下一代真实感渲染器,主要做了一些blender场景对接和粒子动画方面的设计探索。}
    \vspace{\itemspace}\\
    \subcvevent{2022年12月}{2023年08月}{D5图形AI实习生}{D5渲染器}{南京}{asset_icon_d5.png}
    {研一寒假和下学期在D5渲染器作为图形AI实习生,期间主要做了一些基于GAN和PCG的地形生成,基于Stable Diffusion的建筑生成等,帮助搭建d5.hi上线,后期转向可微渲染方向。}
    \vspace{\itemspace}\\
\cvevent{2022年09月}{2025年06月(预计)}{研究生(在读)}{南京大学}{南京}{asset_icon_nju_small.jpg}
{硕士期间开始全面转向图形学,和导师商量的研究主题为NeRF,个人开题为“基于可微渲染的新视角生成方法”,主要内容就是NeRF和Gaussian Splatting方法。}
\vspace{\itemspace}\\

\cvevent{2022年04月}{2022年07月}{前端工程师}{字节跳动}{杭州}{asset_icon_bytedance.png}
{自学前端之后进入字节跳动工作了一段时间,主要做数据可视化大屏中一些webgl相关的工作。所以也比较熟悉前端相关的技能。}
\vspace{\itemspace}\\
\cvevent{2017年09月}{2022年03月}{本科}{浙江大学}{杭州}{asset_icon_zju_small.png}
{竺可桢学院交叉创新平台智能机器人方向,控制和机械电子工程双学位,但是发现自己相比硬件更喜欢软件,相比系统原理更喜欢数学模型的实现,大三才因为GAMES发现居然有图形学这么一个学科,可惜有点迟了,所以到研究生期间才有空余精力转向。}
\vspace{0.3cm}\\