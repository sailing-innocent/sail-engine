高性能计算框架\textbf{Luisa Compute} \cite{zhengLuisaRenderHighPerformanceRendering2022}
具有非常多的特性,
本节主要概述LC框架的部分概念以及本项目中使用到的特性。

如图\ref{fig:lc_arch}所示,LC框架在前端提供了C++的Embedded DSL(Domain Specific
Language),用户可以在C++中直接编写Kernel,LC框架通过DSL生成AST(Abstract Syntax Tree),并在运行时将其编译为对应后端的Shader,这个方式通常被称为JIT(Just In Time)。

此外LC框架抽象了不同后端的差异,提供了一套统一的运行时API,用户可以使用同样的API对不同的后端进行操作,例如资源申请,内存拷贝。除了基础的运行时,LC还提供了光追加速结构的支持,用户可以在LC中使用统一的API进行光追加速结构的构建与使用,本项目的光追碰撞检测算法部分将会使用到这一特性。

得力于JIT,LC的运行时与Kernel编写能够产生“1+1>2”的耦合效果,这种耦合效果是通过\textbf{变量捕获}和\textbf{资源捕获}产生的。例如,用户可以在C++中定义一个变量,然后在DSL编写的Kernel中直接使用这个变量。这个变量随后将会被捕获并转化为Kernel中的一个常量。Kernel同样可以捕获Buffer、Image、BindlessArray,用户无需在调用对应Shader$^1$时手动传递资源的Handle,但二者机制有所差异。后面的小节中将会更深入的介绍\textbf{捕获}的内容。

LC框架提供了多个后端(例如DirectX、Vulkan、Metal等),用户可以将相同的Kernel运行在不同的后端上。在本项目中,我们使用了DirectX后端。
\begin{figure}[H]
	\centering
	\includegraphics[keepaspectratio,width=\linewidth]{fig_diagram_lc.jpg}
	\caption{\href{https://github.com/LuisaGroup/LuisaCompute/tree/next}{LC架构示意图}}
	\label{fig:lc_arch}
\end{figure}

注$^1$:本报告中Shader一般指编译完成的后端代码,Kernel一般指LC中的由用户直接编写的Kernel函数,此外还有Callable,指LC中的由用户编写的非内联展开的设备端函数。
\section{LC的函数}
LC中有三种函数:
\begin{enumerate}
	\item \textbf{Kernel}:用于编译成后端Shader的函数,是设备端的入口函数。
	\item \textbf{Callable}:可被Kernel调用的非内联展开函数。
	\item \textbf{Inline Function}:可被Kernel调用的内联展开函数,在形式上等价于C++原生函数,但在LC C++ DSL的意义下是将设备端代码段在Kernel对应位置展开。
\end{enumerate}

\section{捕获}
\textbf{变量捕获}:在Kernel内使用Host端的运行时对象,这个对象的\textbf{值}将会被捕获。对于基本类型(整型、浮点、向量、矩阵),捕获值为其运行时的值。LC在AST生成阶段将其作为Kernel中的常量。

\textbf{资源捕获}:在Kernel内使用Host端的Buffer、Image、BindlessArray等资源对象,捕获值为其\textbf{Handle}。LC将会在对应Shader调用时隐式传递这些资源的Handle,用户无需手动传递。

\section{多阶段代码生成}
多阶段代码生成是LC的重要特性,LC中Shader代码生成阶段主要有:
\begin{enumerate}
	\item C++ 预处理器宏展开阶段
	\item C++ 编译期模板元实例化阶段
	\item C++ 运行时AST生成阶段
	\item AST到Device端代码生成阶段
\end{enumerate}
其中第1、2阶段为常见的C++编程代码生成阶段,借助LC的C++ Embedded Domain Specific Language(DSL),我们可以利用第1、2阶段对Shader代码进行初期的代码生成工作,例如使用C++模板对不同的数据类型生成对应的函数。在第3阶段,由于运行时编译的特点,我们可以将Bindless Array以\textbf{资源捕获}的方式编译到对应的函数或者Shader中,通过LC的隐式资源Handle传递来完成代码封装。本项目中,\textbf{SPHerePackage}利用了此特性,用于提高代码的模块化程度。第4阶段,用户一般无法直接干预后端代码的生成。

\section{即时编译}
LC中各后端(例如DirectX)使用的Shader均为运行时编译(或从文件cache读取)。JIT方式能够大幅提高Shader的灵活性,从根本上解决“变体”问题。此外JIT方式能够充分利用程序运行时信息,减少不必要分支、优化寄存器分配等。本项目在XPBD求解器的约束求解结果归一化阶段充分利用了此特性,在保证代码高复用性的同时减少了Memory Barrier。

\section{命令重排}
LC运行时能够根据Stream中Command(例如Copy Command、Shader)对内存资源(例如Buffer、Image)的读写情况进行命令重排以减少Kernel之间的同步。本项目中所有API均遵循“Stream无知”的原则,将命令的发出权完全交还给用户以最大化命令重排的效果。

