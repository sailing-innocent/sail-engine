\documentclass[master,oneside,winfonts]{njuthesis/njuthesis}
\setstretch{1.375}%22磅
\titlea{我的毕业论文}
\titleb{2022到2025}
% 论文作者姓名
\author{朱子航}
% 论文作者联系电话
%\telphone{18888921390}
% 论文作者电子邮件地址
\email{522022150087@smail.nju.edu.cn}
\studentnum{522022150087}
% 论文作者入学年份(年级)
\grade{2022}
\supervisor{}
% 导师的联系电话
%\supervisortelphone{13851496821}
% 论文作者的学科与专业方向
\major{电子信息}
% 论文作者的研究方向
\researchfield{控制工程}
% 论文作者所在院系的中文名称
\department{工程管理学院}
% 论文作者所在学校或机构的名称。此属性可选,默认值为``南京大学''。
\institute{南京大学}
% 论文的提交日期,需设置年、月、日。
\submitdate{2023年5月13日}
% 论文的答辩日期,需设置年、月、日。
\defenddate{2023年5月13日}
% 论文的定稿日期,需设置年、月、日。此属性可选,默认值为最后一次编译时的日期,精确到日。
\date{2023年5月13日}

%%%%%%%%%%%%%%%%%%%%%%%%%%%%%%%%%%%%%%%%%%%%%%%%%%%%%%%%%%%%%%%%%%%%%%%%%%%%%%%
% 设置论文的英文封面

% 论文的英文标题,不可换行
\englishtitle{My Final Thesis}
% 论文作者姓名的拼音
\englishauthor{Zhu Zihang}
% 导师姓名职称的英文
\englishsupervisor{}
% 论文作者学科与专业的英文名
\englishmajor{Control Science and Engineering}
% 论文作者所在院系的英文名称
\englishdepartment{School of Management \& Engineering}
% 论文作者所在学校或机构的英文名称。此属性可选,默认值为``Nanjing University''。
\englishinstitute{Nanjing University}
% 论文完成日期的英文形式,它将出现在英文封面下方。需设置年、月、日。日期格式使用美国的日期
% 格式,即``Month day, year'',其中``Month''为月份的英文名全称,首字母大写;``day''为
% 该月中日期的阿拉伯数字表示;``year''为年份的四位阿拉伯数字表示。此属性可选,默认值为最后
% 一次编译时的日期。
\englishdate{May 13, 2023}

%%%%%%%%%%%%%%%%%%%%%%%%%%%%%%%%%%%%%%%%%%%%%%%%%%%%%%%%%%%%%%%%%%%%%%%%%%%%%%%
% 设置论文的中文摘要

% 设置中文摘要页面的论文标题及副标题的第一行。
% 此属性可选,其默认值为使用|\title|命令所设置的论文标题
\abstracttitlea{我过去24年的简单总结}
% 设置中文摘要页面的论文标题及副标题的第二行。
% 此属性可选,其默认值为空白
%\abstracttitleb{}

%%%%%%%%%%%%%%%%%%%%%%%%%%%%%%%%%%%%%%%%%%%%%%%%%%%%%%%%%%%%%%%%%%%%%%%%%%%%%%%
% 设置论文的英文摘要

% 设置英文摘要页面的论文标题及副标题的第一行。
% 此属性可选,其默认值为使用|\englishtitle|命令所设置的论文标题
\englishabstracttitlea{The Conclusion of my past 24 years}
% 设置英文摘要页面的论文标题及副标题的第二行。
% 此属性可选,其默认值为空白
%\englishabstracttitlea{Face recognition with occlusion based on sparse representation}
%\englishabstracttitleb{for Data Centers}

%%%%%%%%%%%%%%%%%%%%%%%%%%%%%%%%%%%%%%%%%%%%%%%%%%%%%%%%%%%%%%%%%%%%%%%%%%%%%%%

%%%%%%%%%%%%%%%%%%%%%%%%%%%%%%%%%%%%%%%%%%%%%%%%%%%%%%%%%%%%%%%%%%%%%%%%%%%%%%%
\begin{document}
%%%%%%%%%%%%%%%%%%%%%%%%%%%%%%%%%%%%%%%%%%%%%%%%%%%%%%%%%%%%%%%%%%%%%%%%%%%%%%%

% 制作国家图书馆封面(博士学位论文才需要)
%\makenlctitle
% 制作中文封面
\maketitle
% 制作英文封面
\makeenglishtitle


%%%%%%%%%%%%%%%%%%%%%%%%%%%%%%%%%%%%%%%%%%%%%%%%%%%%%%%%%%%%%%%%%%%%%%%%%%%%%%%
% 开始前言部分
%\frontmatter
%\makeoriginal%原创性声明
%%%%%%%%%%%%%%%%%%%%%%%%%%%%%%%%%%%%%%%%%%%%%%%%%%%%%%%%%%%%%%%%%%%%%%%%%%%%%%%
% 论文的中文摘要
\pagenumbering{Roman}
\begin{spacing}{1.25}
    \begin{abstract}
        回归(Regression)方法作为机器学习领域的一类基础模型,被广泛应用于数据的特征提取、分类、预测等任务中。然而,在大数据时代背景下,数据中含有诸多不确定性因素如高维、复杂噪声等,给传统的回归方法带来挑战。研究鲁棒回归模型对于实际情形中的数据分析具有重要意义.XXXXX


        % 中文关键词。关键词之间用中文全角分号隔开,末尾无标点符号。
        \keywords{XXX,XXX,XXX}
    \end{abstract}
\end{spacing}

%%%%%%%%%%%%%%%%%%%%%%%%%%%%%%%%%%%%%%%%%%%%%%%%%%%%%%%%%%%%%%%%%%%%%%%%%%%%%%%
% 论文的英文摘要
\begin{spacing}{1.15}
    \begin{englishabstract}
        As a fundamental model of machine learning, regression is widely used for feature extraction, classification, prediction and other tasks. However, in the big data era, there are a variety of uncertainties in data, such as high-dimensionality and complex noises, which pose a great challenge to traditional regression methods. It is vital for practical data analysis to develop robust regression models.


        % 英文关键词。关键词之间用英文半角逗号隔开,末尾无符号。
        \englishkeywords{XXX, XXX, XXX}
    \end{englishabstract}
\end{spacing}

%%%%%%%%%%%%%%%%%%%%%%%%%%%%%%%%%%%%%%%%%%%%%%%%%%%%%%%%%%%%%%%%%%%%%%%%%%%%%%%
% 生成论文目次
\begin{spacing}{1}
    \renewcommand*\contentsname{目录}
    \tableofcontents
\end{spacing}
%%%%%%%%%%%%%%%%%%%%%%%%%%%%%%%%%%%%%%%%%%%%%%%%%%%%%%%%%%%%%%%%%%%%%%%%%%%%%%%
% 生成插图清单。如无需插图清单则可注释掉下述语句。
%%\listoffigures

%%%%%%%%%%%%%%%%%%%%%%%%%%%%%%%%%%%%%%%%%%%%%%%%%%%%%%%%%%%%%%%%%%%%%%%%%%%%%%%
% 生成附表清单。如无需附表清单则可注释掉下述语句。
%%\listoftables

%%%%%%%%%%%%%%%%%%%%%%%%%%%%%%%%%%%%%%%%%%%%%%%%%%%%%%%%%%%%%%%%%%%%%%%%%%%%%%%
% 开始正文部分
\mainmatter
%%%%%%%%%%%%%%%%%%%%%%%%%%%%%%%%%%%%%%%%%%%%%%%%%%%%%%%%%%%%%%%%%%%%%%%%%%%%%%%
% 学位论文的正文应以《绪论》作为第一章
\pagenumbering{arabic}
\chapter{绪论}\label{chapter_introduction}
\section{研究背景与意义}
近年来,随着传感器、互联网和通信技术的飞速发展,数据的生成、传输和获取变得尤为便利。
来自金融、工业、医疗、科学研究等诸多领域的数据呈现出爆炸式增长,对社会发展产生了深远影响。
自2020年新冠疫情爆发以来,有关交通、通信和互联网等行业的海量数据为病毒感染者、密切接触者的锁定和追踪提供了宝贵信息,成为疫情防控的重要资源基础。现实中庞大的数据资源蕴含着巨大的知识价值。
如何有效挖掘数据的本质信息,探索其隐含知识和规律并将其应用于人类的生产生活中,是当前计算机领域的重要研究问题。

\section{基于回归模型的特征提取与分类研究概述}
\subsection{基于回归模型的特征提取方法}
特征提取和特征选择是维数约简的两类重要方法。
特征提取在原始特征上施加线性或非线性变换,以获取新的低维特征;
而特征选择不改变原有特征,仅从数据的原始特征中挑选一部分特征,
实现维数约简。线性回归学习从源数据到目标数据的线性变换,
是一种基本的统计和机器学习模型,被广泛应用于特征提取和特征选择。
Lasso回归\cite{arjovskyWassersteinGAN2017}通过对组合系数施加$\ell_1$惩罚,
使得大部分系数为0,从而实现重要特征的自适应挑选,是一种基本的特征选择方法。
Nie等提出了鲁棒特征选择(Robust Feature Selection, RFS)\cite{nie2010efficient}方法。
RFS采用$\ell_{2,1}$范数约束下的最小二乘回归模型来分析各个特征在分类中的重要度,实现特征选择。由于特征选择只能从原始特征中进行选择,算法效果严重依赖于原始特征质量。
特征提取通过对原始特征的组合,可以生成新的抽象特征。

%%%%%%%%%%%%%%%%%%%%%%%%%%%%%%%%%%%%%%%%%%%%%%%%%%%%%%%%%%%%%%%%%%%%%%%%%%%%%%%
% 开始前言部分
%\frontmatter
%\makeoriginal%原创性声明
%%%%%%%%%%%%%%%%%%%%%%%%%%%%%%%%%%%%%%%%%%%%%%%%%%%%%%%%%%%%%%%%%%%%%%%%%%%%%%%

%%%%%%%%%%%%%%%%%%%%%%%%%%%%%%%%%%%%%%%%%%%%%%%%%%%%%%%%%%%%%%%%%%%%%%%%%%%%%%%
% 致谢,应放在《结论》之后
\begin{acknowledgement}
    XXXXX
\end{acknowledgement}

% 参考文献。应放在\backmatter之前。
% 推荐使用BibTeX,若不使用BibTeX时注释掉下面一句。
%\usepackage{natbib}
%\setlength{\bibsep}{0.5ex}

\nocite{*}
\begin{spacing}{0.2}
    %\bibliographystyle{gbt7714-2005} %原版:gbt7714-2005,或IEEEtran
    \bibliography{sample}
\end{spacing}


%%%%%%%%%%%%%%%%%%%%%%%%%%%%%%%%%%%%%%%%%%%%%%%%%%%%%%%%%%%%%%%%%%%%%%%%%%%%%%%
% 书籍附件
%%\backmatter
%%%%%%%%%%%%%%%%%%%%%%%%%%%%%%%%%%%%%%%%%%%%%%%%%%%%%%%%%%%%%%%%%%%%%%%%%%%%%%%
% 作者简历与科研成果页,应放在backmatter之后
\begin{resume}
    % 论文作者身份简介,一句话即可。
    \begin{authorinfo}
        \noindent 朱子航,男,汉族,1999年4月出生,安徽合肥人。
    \end{authorinfo}
    % 论文作者教育经历列表,按日期从近到远排列,不包括将要申请的学位。
    \begin{education}
    \item[2007年9月 --- 2010年6月] 南京大学工程管理学院 \hfill 硕士
    \item[2017年9月 --- 2022年3月] 浙江大学控制学院 \hfill 本科
    \end{education}
    % 论文作者在攻读学位期间所发表的文章的列表,按发表日期从近到远排列。
    \begin{publications}
        \item XXXXX.
        \item XXXXX.
    \end{publications}
    % 论文作者在攻读学位期间参与的科研课题的列表,按照日期从近到远排列。
    %%\begin{projects}
    %%\item 国家自然科学基金面上项目``无线传感器网络在知识获取过程中的若干安全问题研究''(课题年限~2010年1月 --- 2012年12月),负责位置相关安全问题的研究。
    %%\end{projects}
\end{resume}


%%%%%%%%%%%%%%%%%%%%%%%%%%%%%%%%%%%%%%%%%%%%%%%%%%%%%%%%%%%%%%%%%%%%%%%%%%%%%%%
% 生成《学位论文出版授权书》页面,应放在最后一页
%\makelicense

%%%%%%%%%%%%%%%%%%%%%%%%%%%%%%%%%%%%%%%%%%%%%%%%%%%%%%%%%%%%%%%%%%%%%%%%%%%%%%%
\end{document}
