\documentclass[10pt, hyperref={colorlinks=true,linkcolor=blue},xcolor=dvipsnames]{beamer}
%----------------------------------------------------------------------------------------
%	FUNCTIONAL SETTINGS
%----------------------------------------------------------------------------------------
\usepackage{cite}
\usepackage{graphicx}
\usepackage{float}
\usepackage{amsmath}

% defines

%----------------------------------------------------------------------------------------
%	STYLES AND THEME
%----------------------------------------------------------------------------------------

\setbeamertemplate{bibliography item}[text]
\usepackage{booktabs}
\usetheme{Dresden}
\usecolortheme{dove}
\usefonttheme{structureitalicserif} 
\usepackage{mathptmx}
\usepackage{inconsolata}
\usepackage[default]{lato}
\useinnertheme{rectangles}

\useoutertheme{shadow}

%----------------------------------------------------------------------------------------
%	PRESENTATION INFORMATION
%----------------------------------------------------------------------------------------


\title[title]{ Recent Research Progress }

\subtitle{Since 20230407}

\author[Zhu Zihang]{\texorpdfstring{Zhu Zihang \\ \smallskip \textit{sailing-innocent@foxmail.com}}{}}

\date[\today]{\texorpdfstring{Weekly Meeting \\ 20230407}{}}

\begin{document}

\begin{frame}
    \titlepage
\end{frame}

\begin{frame}
    \frametitle{TOC}
    \tableofcontents
\end{frame}

\section{AIGC}

\subsection{GAN: Generative Adversarial Networks}

\begin{frame}
    \frametitle{Generative Adversarial Networks}
    \begin{itemize}
        \item Milestones
        \item Principle
        \item Application
    \end{itemize}
\end{frame}

\begin{frame}
    \frametitle{Principal}
    \begin{quote}
        Two MLPs were trained using back-propogation as
        \begin{itemize}
            \item Generative Model G
            \item Discriminative Model D
        \end{itemize}

        The Generative Model captures data distribution.
        The Discriminative Model estimates the probability that a sample is from trained data or not.

        \textbf{Ideal Result}:
        \begin{itemize}
            \item G could generate the data in trained set
            \item D equal to $\frac{1}{2}$ ( so that it cannot determine whether the generated data is from trained set or not)
        \end{itemize}
    \end{quote}
    \textbf{Loss}:
    $$\min_G\max_DV(D,G) = E_{x\sim p_{data}(x)}[logD(x)] + E_{z\sim p_{z}(z)}[log(1 - D(G(z)))]$$
\end{frame}

\begin{frame}
    \frametitle{Milestones}
    \begin{itemize}
        \item 2014: GAN\cite{goodfellowGenerativeAdversarialNetworks2014}
        \item 2014: condition GAN \cite{mirzaConditionalGenerativeAdversarial2014}
        \item 2017: Wasserstein GAN \cite{arjovskyWassersteinGAN2017}
        \item 2018: pix2pix \cite{isolaImagetoImageTranslationConditional2018}
        \item 2019: GauGAN \cite{parkSemanticImageSynthesis2019}
        \item 2020: cycleGAN \cite{zhuUnpairedImagetoImageTranslation2020}
        \item 2023: GigaGAN \cite{kangScalingGANsTexttoImage2023}
    \end{itemize}
\end{frame}

\begin{frame}
    \frametitle{Use pix2pix to generate Terrain}
    \begin{columns}[c] % The "c" option specifies centered vertical alignment while the "t" option is used for top vertical alignment
        \begin{column}{0.32\textwidth} % Left column width
            \begin{figure}
                \includegraphics[width=\linewidth]{fig_real_A_wm20230407.png}
                \caption{Real A}
            \end{figure}
        \end{column}
        \begin{column}{0.32\textwidth} % Right column width
            \begin{figure}
                \includegraphics[width=\linewidth]{fig_fake_B_wm20230407.png}
                \caption{Fake B}
            \end{figure}
        \end{column}
        \begin{column}{0.32\textwidth} % Right column width
            \begin{figure}
                \includegraphics[width=\linewidth]{fig_real_b_wm20230407.png}
                \caption{Real B}
            \end{figure}
        \end{column}
    \end{columns}
\end{frame}

\subsection{Diffusion Model}

\begin{frame}
    \frametitle{Diffusion Model}

    \begin{itemize}
        \item Milestones
        \item Principle
        \item Application (Stable Diffusion, LoRA, etc.)
    \end{itemize}

\end{frame}

\begin{frame}
    \frametitle{MileStones}
    \begin{itemize}
        \item 2015: Deep Unsupervised Learning Using Nonequilibrium Thermodynamics \cite{sohl-dicksteinDeepUnsupervisedLearning2015}
        \item 2020: Denoising Diffusion Probabilistic Model \cite{hoDenoisingDiffusionProbabilistic2020}
        \item 2022: Stable Diffusion \cite{rombachHighResolutionImageSynthesis2022}
        \item 2023: Control Net \cite{zhangAddingConditionalControl2023}
        \item 2023: A recent Survey by Pecking University \cite{yangDiffusionModelsComprehensive2023}
    \end{itemize}
\end{frame}

\begin{frame}
    \frametitle{Principal}
    TBD
\end{frame}

\begin{frame}
    \frametitle{Application}
    \begin{figure}
        \includegraphics[width=0.9\linewidth]{fig_sd_ui_wm20230407.png}
        \caption[short]{Stable Diffusion WebUI \href{https://github.com/AUTOMATIC1111/stable-diffusion-webui}{github}}
    \end{figure}
\end{frame}

\begin{frame}
    \frametitle{FineTuning: LoRA}
    \begin{columns}[c] % The "c" option specifies centered vertical alignment while the "t" option is used for top vertical alignment
        \begin{column}{0.3\textwidth} % Left column width
            \begin{figure}
                \includegraphics[width=0.8\linewidth]{fig_lora_principle.png}
                \caption{\href{http://arxiv.org/abs/2106.09685}{LoRA}}
            \end{figure}
        \end{column}
        \begin{column}{0.68\textwidth} % Right column width
            \begin{itemize}
                \item init B with 0, making $BAx = 0$
                \item $h = (W+BA)x$
            \end{itemize}
        \end{column}
    \end{columns}
\end{frame}

\begin{frame}
    \frametitle{Result with LoRA}

    \begin{columns}[c] % The "c" option specifies centered vertical alignment while the "t" option is used for top vertical alignment
        \begin{column}{0.48\textwidth} % Left column width
            \begin{figure}
                \includegraphics[width=\linewidth]{fig_lora_before_wm20230407.png}
                \caption{Without LoRA}
            \end{figure}
        \end{column}
        \begin{column}{0.48\textwidth} % Right column width
            \begin{figure}
                \includegraphics[width=\linewidth]{fig_lora_after_wm20230407.png}
                \caption{with LoRA trained with 50 sketch images 15000 steps}
            \end{figure}
        \end{column}
    \end{columns}
    \begin{quote}prompt: house, river, blue, sketch
    \end{quote}
\end{frame}

\begin{frame}
    \frametitle{ControlNet Hackathon}
    \begin{figure}
        \includegraphics[width=0.9\linewidth]{fig_train_your_control_net_wm20230407.png}
        \caption[short]{HuggingFace \href{https://mp.weixin.qq.com/s/6p2jwa9E6VYHBN38HlPRXQ}{ControlNet Hackathon}}
    \end{figure}
\end{frame}

\subsection{Transformer}

\begin{frame}
    \frametitle{SAM: Segment Anything Model}
    \framesubtitle{by Meta 2023}
    \begin{columns}[c] % The "c" option specifies centered vertical alignment while the "t" option is used for top vertical alignment
        \begin{column}{0.3\textwidth} % Left column width
            \begin{figure}
                \includegraphics[width=0.9\linewidth]{demo_sam_paper.png}
                \caption{\href{https://arxiv.org/abs/2304.02643}{Segment Anything}}
            \end{figure}
        \end{column}
        \begin{column}{0.68\textwidth} % Right column width
            \begin{enumerate}
                \item Demo: \href{https://segment-anything.com/demo}{segment-anything.com}
                \item Project: \href{https://github.com/facebookresearch/segment-anything}{Github Page}
            \end{enumerate}
            \begin{quote}
                In this work, our goal is to build a foundation model for image segmentation. That is, we seek to develop a promptable model and pre-train
                it on a broad dataset using a task that enables powerful generalization. some little change.
            \end{quote}
        \end{column}
    \end{columns}
\end{frame}


\begin{frame}
    \frametitle{Some Results on SAM}
    \begin{columns}[c] % The "c" option specifies centered vertical alignment while the "t" option is used for top vertical alignment
        \begin{column}{0.48\textwidth} % Left column width
            \begin{figure}
                \includegraphics[width=\linewidth]{demo_sam_city_raw.png}
                \caption{City}
            \end{figure}
        \end{column}
        \begin{column}{0.48\textwidth} % Right column width
            \begin{figure}
                \includegraphics[width=\linewidth]{demo_sam_city_result.png}
                \caption{City With Segs}
            \end{figure}
        \end{column}
    \end{columns}
\end{frame}


\begin{frame}
    \frametitle{Some Results on SAM}
    \begin{columns}[c] % The "c" option specifies centered vertical alignment while the "t" option is used for top vertical alignment
        \begin{column}{0.48\textwidth} % Left column width
            \begin{figure}
                \includegraphics[width=\linewidth]{demo_sam_jungle_raw.png}
                \caption{Jungle}
            \end{figure}
        \end{column}
        \begin{column}{0.48\textwidth} % Right column width
            \begin{figure}
                \includegraphics[width=\linewidth]{demo_sam_jungle_result.png}
                \caption{Jungle Result}
            \end{figure}
        \end{column}
    \end{columns}
\end{frame}


\begin{frame}
    \frametitle{Local Deployment}
    \begin{figure}
        \includegraphics[width=0.9\linewidth]{fig_local_wm20230407.png}
        \caption[short]{It can reach higher effect but with longer latency}
    \end{figure}
\end{frame}


\section{Fluid Simulation}

\begin{frame}
    \frametitle{CSIG 4th competition}
    \begin{figure}
        \includegraphics[width=0.9\linewidth]{fig_csig_wm20230407.png}
        \caption[short]{Recently I joined this competition. DDL May 15th and Final September}
    \end{figure}
\end{frame}


\begin{frame}
    \frametitle{An Overview on Fluid Simulation}
    \begin{itemize}
        \item Navier Stokes Equation
        \item Eulerian and Lagrangian
        \item SPH and MPM
        \item parallel computing with GPU
    \end{itemize}
\end{frame}

\begin{frame}
    \frametitle{Navier Stokes Equation}

    \begin{columns}[c] % The "c" option specifies centered vertical alignment while the "t" option is used for top vertical alignment
        \begin{column}{0.45\textwidth} % Left column width
            \begin{figure}
                \includegraphics[width=0.5\linewidth]{fig_mpm_ssf_demo_wm20230407.png}
                \caption[short]{Current Demo with MPM and SSF}
            \end{figure}
        \end{column}
        \begin{column}{0.5\textwidth} % Right column width
            \begin{equation}
                \begin{aligned}
                    \frac{\partial \vec{u}}{\partial t} + \vec{u}\cdot \nabla \vec{u} + \frac{1}{\rho}\nabla p & = \nu \nabla^2 \vec{u} + \vec{g} \\
                    \nabla \cdot \vec{u}                                                                       & = 0
                \end{aligned}
            \end{equation}
        \end{column}
    \end{columns}

\end{frame}

\begin{frame}
    \frametitle{Lagrangian and Eulerian }
    \begin{columns}[c] % The "c" option specifies centered vertical alignment while the "t" option is used for top vertical alignment
        \begin{column}{0.48\textwidth} % Left column width
            \begin{figure}
                \includegraphics[width=\linewidth]{fig_lagrangian_wm20230407.png}
                \caption{Lagrangian}
            \end{figure}
        \end{column}
        \begin{column}{0.48\textwidth} % Right column width
            \begin{figure}
                \includegraphics[width=\linewidth]{fig_eulerian_wm20230407.png}
                \caption{Eulerian}
            \end{figure}
        \end{column}
    \end{columns}
\end{frame}

\begin{frame}
    \frametitle{SPH \& MPM}
    \begin{itemize}
        \item SPH: Smoothed Particle Hydrodynamics (Origin\cite{priceSmoothedParticleHydrodynamics2012}, Tutorial\cite{koschierSmoothedParticleHydrodynamics2020})
        \item MPM: Material Point Method (Origin\cite{stomakhinMaterialPointMethod2013}, SIGGRAPH2016 Course Note \cite{jiangMaterialPointMethod2016})
    \end{itemize}
\end{frame}


\begin{frame}
    \frametitle{LuisaCompute}
    \begin{figure}
        \includegraphics[width=0.9\linewidth]{fig_luisa_compute_wm20230407.png}
        \caption[short]{\href{https://github.com/LuisaGroup/LuisaCompute}{LuisaCompute Github\cite{zhengLuisaRenderHighPerformanceRendering2022}}}
    \end{figure}
\end{frame}

\begin{frame}
    \frametitle{Huang's Work}
    \begin{figure}
        \includegraphics[width=0.9\linewidth]{fig_huang_sph_wm20230407.png}
        \caption[short]{Our Paper to implement\cite{huangGeneralNovelParallel2019}}
    \end{figure}
\end{frame}

\section{Reference}

\begin{frame}[allowframebreaks]{Reference}
    \bibliography{ref}
    \bibliographystyle{plain}
\end{frame}

\end{document}