这里主要介绍PBD-GS(Position Based Dynamic Aided Gaussian Splatting for better Kernel Reconstruction,我暂时拟定的名字)的想法,原理和初步实验结果。

一些前言介绍的部分直接写的英文,因为后续可能会修改为论文的introduction部分

\section{基于点的体渲染方法}
Point Cloud and Voxel are two primary techniques for representing data in 3D space. Volume-based methods fundamentally rely on back-mapping, which involves emitting rays from the camera towards the scene and accumulating color and opacity along the ray's path, a process exemplified by ray marching. 

Conversely, point-based methods utilize forward-mapping, akin to a rasterization process. This involves transforming points from world space to screen space, followed by 'splatting' onto the screen.


The most classic method for point-based rendering is splatting, which involves projecting points onto the screen and accumulating their color and opacity. This method is simple and efficient, but it has limitations in terms of anti-aliasing and filtering. To address these issues, researchers have proposed various improvements to the splatting method, such as EWA splatting, which uses elliptical weighted average filtering to improve the quality of the rendered image.

\paragraph{EWA Splatting}
\paragraph{EWA Splatting}

EWA Splatting \cite{zwickerEWAVolumeSplatting2001} \cite{zwickerEWASplatting2002} 
is a high-quality splatting method based on Gaussian kernels.
The idea is based on Heckbert's EWA (Elliptical Weighted Average) filter for texture mapping.

\subparagraph{Splatting Equation}



\input{volume_render_equation_brief_en}

Now we assume the extinction function is the weighted 
sum of coefficients and reconstruction kernels

\begin{equation}
g(x)=\sum\limits_{k}\limits^{} g_kr_k(\mathbf{x})
\label{eq:extinction_function}
\end{equation}

This reconstruction kernel reflects the position and shape for individual particles.

We substitude the equation (\ref{eq:extinction_function}) into the volume rendering equation (\ref{eq:volume_render_equation}):

$$I_{\lambda}(\hat{\mathbf{x}})=
\sum\limits_{k}\limits^{} 
(
    \int_0^L c_\lambda(\hat{\mathbf{x}},\xi)g_kr_k(\hat{\mathbf{x}},\xi)
    \prod\limits_{j }\limits^{} e^{
        -g_j \int_0^\xi r_j(\hat{\mathbf{x}},\mu)d\mu
    } d\xi
)$$

To compute this function mathematically, we often use assumptions

\begin{enumerate}
    \item use simplified reconstruction kernel, e.g. Gaussian Kernel.
    \item the local support of each reconstruction won't overlap with each other along the ray.
    \item the reconstruction kernels can be ordered front to back
    \item the emission coefficient is constant 
\end{enumerate}

Further more, we use the Taylor expansion for the exponential function, and we can get the final equation for Splatting:

\input{splatting_equation}

The coordinate $\hat{\mathbf{x}}=(x_0,x_1)$ is called the screen space coordinate, and we can say that $I_\lambda(\hat{\mathbf{x}})$ and $q_k(\hat{\mathbf{x}})$ are defined in \textit{screen space}

splatting is attractive because it only requires the precomputed 2D convolution kernels. In contrast, ray casting methods require the 3D convolution kernels.

The coordinate $\hat{\mathbf{x}}=(x_0,x_1)$ is called 
the screen space coordinate, and we can say that 
$I_\lambda(\hat{\mathbf{x}})$ and $q_k(\hat{\mathbf{x}})$
are defined in \textit{screen space}

splatting is attractive because it only requires the 
precomputed 2D convolution kernels. In contrast, ray
casting methods require the 3D convolution kernels.

\subparagraph{EWA Volome Resampling Filter}

The splatting equation (\ref{eq:splatting_equation}) represents
the output image as a continuous 2D function, 
but in practice, it will be sampled at discrete pixel locations.
So it has to be band-limited to respect to Nyquist frequency of the grid, 
to avoid aliasing.

By appling an appropriate low-pass filter $h(\hat{\mathbf{x}})$,
we have 

$$(I_\lambda \otimes h)(\hat{\mathbf{x}})=
    \int_{\mathbb{R}^2}\sum\limits_{k }\limits^{} 
    c_{\lambda k}(\mu)g_kq_k(\mu)\prod\limits_{j=0}\limits^{k-1} 
    (1-g_jq_j(\mu))h(\hat{\mathbf{x}}-\mu)d\mu $$

we make two more assumptions

\begin{enumerate}
    \item the emission coefficient is approximately constant in supporting area $c_{\lambda x}$
    \item ignore the effect of shading 
    \item the attenuation factor is approximately constant 
\end{enumerate}

Thus we have 

$$\rho_k(\hat{\mathbf{x}})=(r_k \otimes h)(\hat{\mathbf{x}})$$

The resampling filter is strongly space-variant.

Swan et al. present uniform low-pass filter.

But our method provides non-uniform scaling in these cases.

\subparagraph{Elliptical Gaussian Kernel}

We choose Elliptical Gaussian Kernel 

TODO: Gaussian Transform
% \cite{zwickerEWAVolumeSplatting2001} 
% \cite{zwickerEWASplatting2002} 

\section{新视角生成任务}

\paragraph{新视角(Novel View Synthesis, NVS)生成任务}

新视角生成任务表述如下:给定一系列相机的位姿和相机拍摄的图像,能否准确预测出一个新的相机位姿下对应的图像?

和传统的三维重建任务相似,一般的新视角生成方法都需要建立自己的场景表达数学模型(点云,网格,隐式表达等),但三维重建更关注重建场景的空间信息,几何信息,表面纹理特征,深度等重建是否准确,而新视角生成任务更关注场景的视觉效果是否准确,新视角生成任务是三维重建领域下更关注生成图像质量的子问题。

\paragraph{数据格式与性能指标}

以下是新视角生成任务的数据格式与性能指标

\paragraph{数据集}

以下是新视角生成任务中常用的数据集

\subparagraph{MipNeRF360}

在这个数据集上当前的SOTA指标

\begin{table}[H]
    \input{tab_psnr_mip360_sota}
\end{table}

\section{Gaussian Splatting方法}
\paragraph{Scene in Gaussian Splatting}
The 3D Gaussian Splatting method interprets a scene 
as a collection of $N$ 3D Gaussians, forming a characteristic point cloud. 
Each Gaussian is characterized by its mean $\mathbf{m}_i \in R^3$, 
covariance matrix $\mathbf{C}_i$, density $\rho_i\in R$, and color $\mathbf{c}_i\in R^3$.

The covariance matrix $\mathbf{C}_i$, a $3\times 3$ matrix, 
determines the Gaussian's shape. 
It can be computed from a standard Gaussian through a combination of scaling and rotation transformations. 
The scaling transformation, represented by $s_i\in R^3$, indicates the scaling factor along each axis. 
The rotation transformation, represented by a normalized quaternion $q_i\in R^4$, signifies the rotation.

In practice, the gaussian color is computed using Spherical Harmonics (SH) coefficients. 
This approach effectively captures the anisotropic properties of the radiance field across different directions.
For D degree SH, the color $\mathbf{c}_i$ is represented as a vector of $(D+1)^2$ coefficients.

\paragraph{Forward Pass}
The Forward Pass of Gaussian Splatting fetches 
the Gaussian Parameters $\{\mathcal{G}\}$ and Camera Parameters $\mathcal{P}$ 
from the scene as input, and renders the scene into a 2D image $\mathcal{I}_{w\times h}$ of resolution $w$ and $h$.

$$\mathcal{F}_{forward}(\mathcal{G},\mathcal{P})\rightarrow \mathcal{I}_{w\times h}$$



\paragraph{Backward Pass}
Given the loss function $\mathcal{L}$, 
the derivative of the loss function with respect to the 
output image is $\frac{\partial \mathcal{L}}{\partial \mathcal{I}}$.

When it comes to backward pass, 
we use the gradient of the output image 
with respect to the input Gaussian Parameters $\{\mathcal{G}\}$: 
$\frac{\partial \mathcal{I}}{\partial \mathcal{G}}$, 
then we can back-propagate the derivatives from the output image 
to the input Gaussian Parameters 

$$\frac{\partial \mathcal{L}}{\partial \mathcal{G}} = \frac{\partial \mathcal{L}}{\partial \mathcal{I}}\frac{\partial \mathcal{I}}{\partial \mathcal{G}}$$

And we can use this derivative to update the Gaussian Parameters using gradient descent in the optimization stage.

\paragraph{Optimization}
Here we introduce the overall optimization process of 
the Gaussian Splatting Algorithm. 

After initializing the Gaussian Parameters $\{\mathcal{G}\}$, repeatedly perform the following steps:

\begin{enumerate}
    \item randomly choose a view $\mathcal{P}$ from dataset and its corresponding ground truth image $\mathcal{I}_{gt}$
    \item Calculate the output image $\mathcal{I}_{w\times h}$ using the Forward Pass
    \item Calculate the loss function $\mathcal{L}$ between the output image $\mathcal{I}_{w\times h}$ and the ground truth image $\mathcal{I}_{gt}$
    \item Calculate the derivative of the loss function with respect to the output image $\frac{\partial \mathcal{L}}{\partial \mathcal{I}}$
    \item Calculate the derivative of the output image with respect to the input Gaussian Parameters $\{\mathcal{G}\}$: $\frac{\partial \mathcal{I}}{\partial \mathcal{G}}$ and back-propagate the derivatives from the output image to the input Gaussian Parameters $\{\mathcal{G}\}$: $\frac{\partial \mathcal{L}}{\partial \mathcal{G}} = \frac{\partial \mathcal{L}}{\partial \mathcal{I}}\frac{\partial \mathcal{I}}{\partial \mathcal{G}}$ using the Backward Pass
    \item Update the Gaussian Parameters $\{\mathcal{G}\}$ using gradient descent
\end{enumerate}

The optimization process is repeated until the Gaussian Parameters $\{\mathcal{G}\}$ converge to the optimal solution, 
where the output image $\mathcal{I}$ is close enough to the ground truth image $\mathcal{I}_{gt}$. 
We will save the optimal Gaussian Parameters $\hat{\{\mathcal{G}\}}$ for the next stage of evaluation.


\section{我的idea}

核心的想法就在于基于点云的体渲染方法中有一句: the local support of each reconstruction won’t overlap with each other along the ray,要求两个点之间的距离不能过近,否则信号重建的过程中会出现偏差。【WIP】(此处需要示意图)

于是,在优化过程中,如果引入点云内部的动力学,在反向过程中,对于相邻的两个高斯,引入一个新的Loss

$\mathcal{L}_{recon} = e^{-\frac{1}{2}\frac{d^2}{\sigma_z^2}}$

这个loss代表重建损失,其中$\sigma_z$代表在深度方向上的协方差矩阵分量,d代表相邻两个高斯的深度差。

通过这个loss反向传递的梯度,相当于点云彼此之间通过自身的协方差矩阵和自身位置定义了一个势场,将其他点“推开”,这也是为何要取名"Positional Based Dynamic Aided Gaussian Splatting"

\section{初步实验结果}

目前只是初步的训练结果,没有经过更仔细的调参,但是已经可以看出在很多场景下比很多方法,尤其相对于基础方法有提升。标0处的数据代表训练还没有完成,还没有仔细调整过参数,部分结果之所以没有数据是因为在20000次迭代后有奇怪的崩溃问题,可能跟机器显存有关,还在debug,但看loss下降曲线来看效果应该还可以。


多个场景下的数据表明,想法应该是work的,但复杂场景下没有出现质变的关键提升,还需要继续观察。

\paragraph{NeRF Synthetic}

以下是PBD-GS在NeRF Synthetic数据集上的实验结果

\begin{table}[H]
    \input{tab_psnr_ns_pbd_gs}
\end{table}

\paragraph{MipNeRF360}

以下是PBD-GS在MipNeRF360数据集上的实验结果。保持经典的实验设置,包括训练集、验证集和测试集的划分,训练集和验证集的训练细节,测试集的评价指标等。

\begin{table}[H]
    \input{tab_psnr_mip360_pbd_gs}
\end{table}

\begin{table}[H]
    \input{tab_ssim_mip360_pbd_gs}
\end{table}

表\ref{tab:ssim_mip360_pbd_gs}展现了SSIM的指标,表\ref{tab:psnr_mip360_pbd_gs}展现了PSNR的指标。


