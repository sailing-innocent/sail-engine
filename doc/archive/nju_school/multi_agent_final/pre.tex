\documentclass{njupre/njupre}
\title[title] {Text-Guided Synthesis of Crowd Animation}
\subtitle{SIGGRAPH 2024 Paper, final pre for MARL course}
\author[Zhu Zihang]{\texorpdfstring{Zhu Zihang \\ \smallskip \textit{522022150087@smail.nju.edu.cn}}{}}
\date[2024-05-23]{\textit{2024-05-23}}
\begin{document}
\begin{frame}
    \titlepage 
\end{frame}
\begin{frame}
    \frametitle{TOC}
    \tableofcontents 
\end{frame}

\begin{frame}
    \frametitle{Text-Guided Synthesis of Crowd Animation}
    \begin{figure}
        \includegraphics[width=\textwidth]{fig_teaser_tgsca.png}
        \caption{Generating realistic crowd animation given an environment map}
    \end{figure}
    \begin{quote}
        Researchers from HKU and Tencent find a technique \cite{jiTextGuidedSynthesisCrowd2024}
        of using LLM and Diffusion Model to automately generate realistic crowd animation.
    \end{quote}
\end{frame}

\section{Introduction}
\subsection{Crowd Animation}
\begin{frame}
    \frametitle{Crowd Animation}
    Task
    \begin{quote}
        Given a description of environment, generate a realistic crowd moving animation 
    \end{quote}
    We want:
    \begin{enumerate}
        \item Collision-free 
        \item Avoid unnatural effects
        \item Apply the right path
    \end{enumerate}
    But 
    \begin{enumerate}
        \item The environement has various semantic meanings
        \item The Interaction of crowd and environemt is multi-modal
        \item Physically-correct simulation is computational expensive
    \end{enumerate}
\end{frame}


\begin{frame}
    \frametitle{Main Methods for Crowd Animation}
    \begin{columns}[c]
        \begin{column}{0.3\textwidth}
            \begin{figure}
                \includegraphics[width=\textwidth]{fig_navigation_field.png}
                \caption{Flow-Based Method\cite{patilDirectingCrowdSimulations2011}}
            \end{figure}
        \end{column}
        \begin{column}{0.3\textwidth}
            \begin{figure}
                \includegraphics[width=\textwidth]{fig_maslow_hierarchy_of_needs.png}
                \caption{Entity-Based Method}
            \end{figure}
        \end{column}
        \begin{column}{0.3\textwidth}
            \begin{figure}
                \includegraphics[width=\textwidth]{fig_agent_based_crowd.png}
                \caption{Agent-Based Method \cite{tanakaGuidanceFieldVector2016}}
            \end{figure}
        \end{column}
    \end{columns}
\end{frame}


\subsection{LLM}
\begin{frame}
    \frametitle{The Large Language Model}
    \begin{figure}
        \includegraphics[width=\textwidth]{fig_llm_history.png}
        \caption{Overview of LLM \cite{naveedComprehensiveOverviewLarge2024}}
    \end{figure}
\end{frame}

\begin{frame}
    \frametitle{GPT-4o}
    \begin{figure}
        \includegraphics[width=\textwidth]{fig_gpt4o.png}
        \caption{The New Multi-Model LLM GPT-4o (Omni)}
    \end{figure}
\end{frame}
\subsection{Diffusion Model}
\begin{frame}
    \frametitle{The Diffusion Model}
    \begin{figure}
        \includegraphics[width=\textwidth]{fig_ddpm.png}
        \caption{DDPM \cite{hoDenoisingDiffusionProbabilistic2020}}
    \end{figure}
\end{frame}

\begin{frame}
    \frametitle{The Latent Diffusion Model}
    \begin{figure}
        \includegraphics[width=\textwidth]{fig_ldm.png}
        \caption{Latent Diffusion Model \cite{rombachHighResolutionImageSynthesis2022}}
    \end{figure}
\end{frame}

\begin{frame}
    \frametitle{MileStones}
    \begin{itemize}
        \item 2015: Deep Unsupervised Learning Using Nonequilibrium Thermodynamics \cite{sohl-dicksteinDeepUnsupervisedLearning2015}
        \item 2020: Denoising Diffusion Probabilistic Model \cite{hoDenoisingDiffusionProbabilistic2020}
        \item 2022: Stable Diffusion \cite{rombachHighResolutionImageSynthesis2022}
        \item 2023: Control Net \cite{zhangAddingConditionalControl2023}
    \end{itemize}
\end{frame}



\section{Method}
\subsection{Overview}
\section{First Year Overview}
从2022年9月入学之后,我意识到自己之前耗费了巨大的代价
才看清选定了方向,拥有极大的侥幸才得到了一个安定的环境。
于是研究生阶段核心任务自然就是入门图形学,
同时需要保证主线的研究生毕业过程不发生大的翻车,
最好能够有更加丰富的行业实习经历和论文科研经历。

2023年的11月我在LuisaCompute上通过自己提出idea,
设计方法,克服困难实现了将Gaussian Splatting方法和SPH流体模拟结合起来之后,
我意识到自我探索的过程即将结束,
后续我已经开始有能力面对外界的挑战,并且在战斗中逐渐成长。

\begin{enumerate}
    \item 系统入门了计算机图形学,视觉三维重建,人工智能与深度学习相关的领域,自学并熟练掌握了C++,python等编程语言,
    \item 在实习和科研项目中获得了比较丰富的项目经验,组队参加CSIG2023年全国比赛并获得AMD流体粒子仿真赛道的亚军,在团队中处于较为核心的位置
    \item 在持续相关领域的论文阅读中跟进了有关“神经辐射场”(Neural Radiance Fields, NeRF)相关领域的前沿,复现了若干经典和前沿的项目。并在此基础上提出了自己的创新。
\end{enumerate}

研究生阶段的第一年主要是图形学和AI的入门,以及基础笔记和代码库的构建。回过头去看,搭起这样一个代码和研究框架简直匪夷所思,虽然自己有时候还是觉得不够满足,但是长远客观的视角看来,这就是我当时水平下所能做到最多的事情了。

我难以想象自己当时究竟是以一种什么样的毅力在坚持着,也许是不甘心吧……如果重来一次,面对自己当时那么薄弱的基础,如果知道自己未来将会面对怎样挑战的话,当时还敢all in这个领域吗?

无论如何,我侥幸走过来了,事实也能证明,很多时候我们只不过是高估了自己一周内的产出,却低估了自己在一年内的变化。坚持不懈,久久为功,总还是能有所进益。

但是在取得成绩的同时,我也意识到自己的不足:
\begin{enumerate}
    \item 整体过于偏向工程实现的能力积累,以至于理论和实验环节有所缺失。
    \item 数学和物理的基础相对不足,对于复杂问题的推演能力偏弱等等。
    \item 方向较宽泛,不够聚焦和明确,部分选题超过自身能力,导致长期难以出结果
\end{enumerate}

未来一年我计划6月之前留在学校内,主要完成
\begin{enumerate}
    \item 完成剩余4学分的课业
    \item 未来的一年会更偏向理论的积累,利用自己之前在各个领域积累的工程和实现能力,扎实推进理论的研究。
    \item 提高设计实验验证理论的能力,提高执行力,提高实现框架和推进实验的效率
\end{enumerate}

\subsection{研究笔记库的构建}

这是我继承之前用obsidian等工具构建的笔记系统,使用dendron构建的研究笔记。

\paragraph{研究笔记框架}

这个项目追溯起来可能来自我本科时候才开始熟悉编程时候想要完成的“大一统数据管理系统”。
因为对于笔记美观的追求才开始学习前端。经过很多年的尝试,分辨之后终于在研究生阶段逐渐形成了一套比较熟悉的工具链和稳定的框架。
经过大半年的尝试运行之后基本可以很好地协助完成我的各种需求。

\subsection{代码库的构建}

\paragraph{研究代码框架}

作为图形学的研究生,对于代码框架的要求多且繁杂。一方面需要大量的C/C++基础库造轮子,另一方面需要python和深度学习库来做训练。甚至还需要承担一部分报告论文的排版工作。

整个研究代码框架的搭建是我在南大的研究生第一年的主要成果。现如今基本覆盖了我绝大部分的自我研究需求。

整个框架主要包括

\begin{itemize}
    \item 纯C++的基础库sail
    \item 基于底层图形API的功能库ing 
    \item 基于luisa-compute的功能库inno 
    \item 基于python的综合库cent 
\end{itemize}

\paragraph{基础库sail}

最开始学C++建立的基础库,是所有后续代码框架的根基。在2023年一月熟悉xmake之后重构了一次,后续又有若干次重构,从纯头文件库变成了基础功能库。

\paragraph{功能库ing}

最开始学习图形API(OpenGL, Vulkan, CUDA)过程中建立的库,在2023年一月根据xmake重构了一次,在刚开始鲸吞GaussianSplatting的过程中又进行了一些重构。是很多算法功能的基础开发库,利用稳定的API来实现基础的构思。

\paragraph{功能库inno}

功能库inno的实现最晚,主要是在2023年的三月到六月开始使用LuisaCompute实现一个基于SPH算法的流体模拟仿真项目。暂时依然沿用那一套package框架方案,只是拓展了开发可微渲染器的功能,支持python binding。主要利用LuisaCompute的开发效率和社区环境进行深入参与维护,并合并各个子模块的功能。

\paragraph{综合库cent}

cent是与研究的信息管理系统交互的主要媒介,其doc子文件夹用xmake构建了一个编译项目框架,可以支持多种笔记,报告,论文的排版产出,无缝与信息管理系统vault同步。同时也是各种初步实现,原型的集散地。也是底层功能库封装的调用地点。cent使用pytest来整体管理多输出的python项目。

2023年八月中旬之后,从我D5离开,
从高楼门的小住房搬离之后就开始了研究生第二年的修行。

主要是入门了可微渲染,夯实了专业和代码基础,
构建了自己的专业主线。之后开始着手进行SPH项目的收尾
以及Gaussian Splatting项目的推进。后者最终成为我的开题。

\subsection{Semantic Inference}
\begin{frame}
    \frametitle{Canonical Semantic}
    \begin{figure}
        \includegraphics[width=\textwidth]{fig_semantic_struct_tgsca.png}
        \caption{The Canonical Semantic Structure}
    \end{figure}
\end{frame}


\subsection{Diffusion Training}
\begin{frame}
    \frametitle{Images for Diffusion Model}
    for each group i, we will predict its:
    \begin{enumerate}
        \item $I_s^i$ the starting point probability distribution
        \item $I_g^i$ the goal point probability distribution
        \item $I_v^i$ the velocity field (navigation field)
    \end{enumerate}
\end{frame}

\begin{frame}
    \frametitle{LDM Principle}
    \begin{enumerate}
        \item data distribution $d_0\sim q(d_0)$
        \item gradually inject noise $d_{0:T}\sim q(d_{1:T}|d_0)q(d_0)$
        \item a learnable model is trained to reverse 
        \item reparameterize to predict the noise potential $\epsilon_\theta(d^t,t,h)$
        \item the prediction of image $d_{t-1}\sim \mathcal{N}(\frac{1}{\sqrt{\alpha_t}}(d_t-\frac{\beta_t}{\sqrt{1-\tilde{\alpha_t}}}\epsilon_\theta(d_t,t,h)), \sigma_t I)$
        \item h is the latent signal that can be encoded from CLIP or other images
    \end{enumerate}
\end{frame}

\begin{frame}
    \frametitle{Jointly Prediction}
    for each group j:
    $$h^j=CLIP(s^j)$$
    $$I^j_{s,g,t-1}= \frac{1}{\sqrt{\alpha_t}}(d_t-\frac{\beta_t}{\sqrt{1-\tilde{\alpha_t}}}\epsilon_\theta(I^j_{s,g,t},I_e,t,h^j))$$
    $$I^j_{v,t-1}= \frac{1}{\sqrt{\alpha_t}}(d_t-\frac{\beta_t}{\sqrt{1-\tilde{\alpha_t}}}\epsilon_\theta(I^j_{v,t},I_e,I_s,I_g,t,h^j))$$

    \begin{quote}
        Two Diffusion Models trained seperately, on 4 RTX 4090 with 192h, 50 epoch for start and goal diffusion and 200 epoch for velocity field diffusion
    \end{quote}
\end{frame}


\subsection{Dataset Generation}
\begin{frame}
    \frametitle{Dataset Generation}
    \begin{enumerate}
        \item Randomly sample environment map for start and goal pairs and semantic entities
        \item Sample Navigation Path 
        \item generate groud truth canonical semantics 
    \end{enumerate}
\end{frame}

\begin{frame}
    \frametitle{Veloicity Field Adjustment}
    \begin{figure}
        \includegraphics[width=0.8\textwidth]{fig_vel_adjust_tgsca.png}
        \caption{Veloicity Field Adjustment with RVO\cite{vandenbergReciprocalVelocityObstacles2008}}
    \end{figure}
\end{frame}


\section{Evaluation}
\subsection{Qualitative Evaluation}
\begin{frame}
    \frametitle{Result 1}
    \begin{figure}
        \includegraphics[width=\textwidth]{fig_res_1_tgsca.png}
        \caption{The Scenario with Large Group}
    \end{figure}
\end{frame}

\begin{frame}
    \frametitle{Result 2}
    \begin{figure}
        \includegraphics[width=\textwidth]{fig_res_2_tgsca.png}
        \caption{The Scenraio with 6 Different groups}
    \end{figure}
\end{frame}
\subsection{Quantitative Evaluation}
\begin{frame}
    \frametitle{Quantitative Result}
    \begin{figure}
        \includegraphics[width=\textwidth]{fig_quant_res_tgsca.png}
        \caption{Quantative Result}
    \end{figure}
    \begin{enumerate}
        \item ATD: Agent Trajectory Distance
        \item SSR: Strict Success Rate (80\% agents 70\% in trajectory calls a group successful)
        \item RSR: Relax Successful Rate (70\% agents 50\% in trajectory calls a group successful)
    \end{enumerate}
\end{frame}
\subsection{User Study}
\begin{frame}
    \frametitle{User Study}
    \begin{quote}
        Each participant is asked to assign a score for each scenario, 
        scaling from 1-5, where a higher score indicates better realism. 
        A total of 12 participants were recruited. 
    \end{quote}
\end{frame}

\section{Conclusion}
本文提出了一个可迭代的发包策略在解决车间调度问题中的应用。该策略通过学习之前调度过程中的错误来不断地调整发包策略,使得车间调度问题的求解过程逐渐收敛到最优。实验结果证明可迭代的发包策略确实可以在解决车间调度问题中取得更好的效果,同时对可迭代发包策略的深入分析(初始化,克服局部最优等)也能进一步提高效果。

\begin{frame}[allowframebreaks]{Reference}
    \bibliography{ref}
    \bibliographystyle{plain}
\end{frame}
\begin{frame}
    \frametitle{Thank You}
    Thanks for watching
\end{frame}
\end{document}