\documentclass{njupre/njupre}
\title[title] {Text-Guided Synthesis of Crowd Animation}
\subtitle{SIGGRAPH 2024 Paper, final pre for MARL course}
\author[Zhu Zihang]{\texorpdfstring{Zhu Zihang \\ \smallskip \textit{522022150087@smail.nju.edu.cn}}{}}
\date[2024-05-23]{\textit{2024-05-23}}
\begin{document}
\begin{frame}
    \titlepage 
\end{frame}
\begin{frame}
    \frametitle{TOC}
    \tableofcontents 
\end{frame}

\begin{frame}
    \frametitle{Text-Guided Synthesis of Crowd Animation}
    \begin{figure}
        \includegraphics[width=\textwidth]{fig_teaser_tgsca.png}
        \caption{Generating realistic crowd animation given an environment map}
    \end{figure}
    \begin{quote}
        Researchers from HKU and Tencent find a technique \cite{jiTextGuidedSynthesisCrowd2024}
        of using LLM and Diffusion Model to automately generate realistic crowd animation.
    \end{quote}
\end{frame}

\section{Introduction}
\subsection{Crowd Animation}
\begin{frame}
    \frametitle{Crowd Animation}
    Task
    \begin{quote}
        Given a description of environment, generate a realistic crowd moving animation 
    \end{quote}
    We want:
    \begin{enumerate}
        \item Collision-free 
        \item Avoid unnatural effects
        \item Apply the right path
    \end{enumerate}
    But 
    \begin{enumerate}
        \item The environement has various semantic meanings
        \item The Interaction of crowd and environemt is multi-modal
        \item Physically-correct simulation is computational expensive
    \end{enumerate}
\end{frame}


\begin{frame}
    \frametitle{Main Methods for Crowd Animation}
    \begin{columns}[c]
        \begin{column}{0.3\textwidth}
            \begin{figure}
                \includegraphics[width=\textwidth]{fig_navigation_field.png}
                \caption{Flow-Based Method\cite{patilDirectingCrowdSimulations2011}}
            \end{figure}
        \end{column}
        \begin{column}{0.3\textwidth}
            \begin{figure}
                \includegraphics[width=\textwidth]{fig_maslow_hierarchy_of_needs.png}
                \caption{Entity-Based Method}
            \end{figure}
        \end{column}
        \begin{column}{0.3\textwidth}
            \begin{figure}
                \includegraphics[width=\textwidth]{fig_agent_based_crowd.png}
                \caption{Agent-Based Method \cite{tanakaGuidanceFieldVector2016}}
            \end{figure}
        \end{column}
    \end{columns}
\end{frame}


\subsection{LLM}
\begin{frame}
    \frametitle{The Large Language Model}
    \begin{figure}
        \includegraphics[width=\textwidth]{fig_llm_history.png}
        \caption{Overview of LLM \cite{naveedComprehensiveOverviewLarge2024}}
    \end{figure}
\end{frame}

\begin{frame}
    \frametitle{GPT-4o}
    \begin{figure}
        \includegraphics[width=\textwidth]{fig_gpt4o.png}
        \caption{The New Multi-Model LLM GPT-4o (Omni)}
    \end{figure}
\end{frame}
\subsection{Diffusion Model}
\begin{frame}
    \frametitle{The Diffusion Model}
    \begin{figure}
        \includegraphics[width=\textwidth]{fig_ddpm.png}
        \caption{DDPM \cite{hoDenoisingDiffusionProbabilistic2020}}
    \end{figure}
\end{frame}

\begin{frame}
    \frametitle{The Latent Diffusion Model}
    \begin{figure}
        \includegraphics[width=\textwidth]{fig_ldm.png}
        \caption{Latent Diffusion Model \cite{rombachHighResolutionImageSynthesis2022}}
    \end{figure}
\end{frame}

\begin{frame}
    \frametitle{MileStones}
    \begin{itemize}
        \item 2015: Deep Unsupervised Learning Using Nonequilibrium Thermodynamics \cite{sohl-dicksteinDeepUnsupervisedLearning2015}
        \item 2020: Denoising Diffusion Probabilistic Model \cite{hoDenoisingDiffusionProbabilistic2020}
        \item 2022: Stable Diffusion \cite{rombachHighResolutionImageSynthesis2022}
        \item 2023: Control Net \cite{zhangAddingConditionalControl2023}
    \end{itemize}
\end{frame}



\section{Method}
\subsection{Overview}
\begin{frame}
    \frametitle{The Overall Pipeline}
    \begin{figure}
        \includegraphics[width=\textwidth]{fig_pipeline_tgsca.png}
        \caption{The Overall Pipeline}
    \end{figure}
\end{frame}
\begin{frame}
    \frametitle{Input}
    \begin{enumerate}
        \item a given environment map with one-hot encoding $I_e = C\times H\times W$ image, where C is the number of semantic types
        \item a rough text-based description $T$ (optionally including number of agents, groups..)
    \end{enumerate}
    \begin{quote}
        Many people are entering from the entrance located 
        at the bottom right of the map, 
        passing through the right access pathway 
        and getting into the top right door. 
        The crowd in another direction that 
        wants to leave the subway passes through the left passage 
        from the top left of the map and leaves at the bottom left.
    \end{quote}
\end{frame}
\begin{frame}
    \frametitle{Semantic Types}
    \begin{figure}
        \includegraphics[width=\textwidth]{fig_semantic_type_tgsca.png}        
        \caption{The Semantic Types}
    \end{figure}
\end{frame}
\begin{frame}
    \frametitle{Objective}
    \begin{enumerate}
        \item the agent-controler pair: $$\mathcal{A}=\{x^0,\pi(x)\rightarrow \dot{x}\}$$
        \item the agent's position at t time: $x^t$
        \item moving direction is calculated by $\dot{x^t}=\pi(x^t)$
        \item runtime agent simulation with CARLA \cite{dosovitskiyCARLAOpenUrban2017}
    \end{enumerate}
\end{frame}
\subsection{Semantic Inference}
\begin{frame}
    \frametitle{Canonical Semantic}
    \begin{figure}
        \includegraphics[width=\textwidth]{fig_semantic_struct_tgsca.png}
        \caption{The Canonical Semantic Structure}
    \end{figure}
\end{frame}


\subsection{Diffusion Training}
\begin{frame}
    \frametitle{Images for Diffusion Model}
    for each group i, we will predict its:
    \begin{enumerate}
        \item $I_s^i$ the starting point probability distribution
        \item $I_g^i$ the goal point probability distribution
        \item $I_v^i$ the velocity field (navigation field)
    \end{enumerate}
\end{frame}

\begin{frame}
    \frametitle{LDM Principle}
    \begin{enumerate}
        \item data distribution $d_0\sim q(d_0)$
        \item gradually inject noise $d_{0:T}\sim q(d_{1:T}|d_0)q(d_0)$
        \item a learnable model is trained to reverse 
        \item reparameterize to predict the noise potential $\epsilon_\theta(d^t,t,h)$
        \item the prediction of image $d_{t-1}\sim \mathcal{N}(\frac{1}{\sqrt{\alpha_t}}(d_t-\frac{\beta_t}{\sqrt{1-\tilde{\alpha_t}}}\epsilon_\theta(d_t,t,h)), \sigma_t I)$
        \item h is the latent signal that can be encoded from CLIP or other images
    \end{enumerate}
\end{frame}

\begin{frame}
    \frametitle{Jointly Prediction}
    for each group j:
    $$h^j=CLIP(s^j)$$
    $$I^j_{s,g,t-1}= \frac{1}{\sqrt{\alpha_t}}(d_t-\frac{\beta_t}{\sqrt{1-\tilde{\alpha_t}}}\epsilon_\theta(I^j_{s,g,t},I_e,t,h^j))$$
    $$I^j_{v,t-1}= \frac{1}{\sqrt{\alpha_t}}(d_t-\frac{\beta_t}{\sqrt{1-\tilde{\alpha_t}}}\epsilon_\theta(I^j_{v,t},I_e,I_s,I_g,t,h^j))$$

    \begin{quote}
        Two Diffusion Models trained seperately, on 4 RTX 4090 with 192h, 50 epoch for start and goal diffusion and 200 epoch for velocity field diffusion
    \end{quote}
\end{frame}


\subsection{Dataset Generation}
\begin{frame}
    \frametitle{Dataset Generation}
    \begin{enumerate}
        \item Randomly sample environment map for start and goal pairs and semantic entities
        \item Sample Navigation Path 
        \item generate groud truth canonical semantics 
    \end{enumerate}
\end{frame}

\begin{frame}
    \frametitle{Veloicity Field Adjustment}
    \begin{figure}
        \includegraphics[width=0.8\textwidth]{fig_vel_adjust_tgsca.png}
        \caption{Veloicity Field Adjustment with RVO\cite{vandenbergReciprocalVelocityObstacles2008}}
    \end{figure}
\end{frame}


\section{Evaluation}
\subsection{Qualitative Evaluation}
\begin{frame}
    \frametitle{Result 1}
    \begin{figure}
        \includegraphics[width=\textwidth]{fig_res_1_tgsca.png}
        \caption{The Scenario with Large Group}
    \end{figure}
\end{frame}

\begin{frame}
    \frametitle{Result 2}
    \begin{figure}
        \includegraphics[width=\textwidth]{fig_res_2_tgsca.png}
        \caption{The Scenraio with 6 Different groups}
    \end{figure}
\end{frame}
\subsection{Quantitative Evaluation}
\begin{frame}
    \frametitle{Quantitative Result}
    \begin{figure}
        \includegraphics[width=\textwidth]{fig_quant_res_tgsca.png}
        \caption{Quantative Result}
    \end{figure}
    \begin{enumerate}
        \item ATD: Agent Trajectory Distance
        \item SSR: Strict Success Rate (80\% agents 70\% in trajectory calls a group successful)
        \item RSR: Relax Successful Rate (70\% agents 50\% in trajectory calls a group successful)
    \end{enumerate}
\end{frame}
\subsection{User Study}
\begin{frame}
    \frametitle{User Study}
    \begin{quote}
        Each participant is asked to assign a score for each scenario, 
        scaling from 1-5, where a higher score indicates better realism. 
        A total of 12 participants were recruited. 
    \end{quote}
\end{frame}

\section{Conclusion}
本文提出了一个可迭代的发包策略在解决车间调度问题中的应用。该策略通过学习之前调度过程中的错误来不断地调整发包策略,使得车间调度问题的求解过程逐渐收敛到最优。实验结果证明可迭代的发包策略确实可以在解决车间调度问题中取得更好的效果,同时对可迭代发包策略的深入分析(初始化,克服局部最优等)也能进一步提高效果。

\begin{frame}[allowframebreaks]{Reference}
    \bibliography{ref}
    \bibliographystyle{plain}
\end{frame}
\begin{frame}
    \frametitle{Thank You}
    Thanks for watching
\end{frame}
\end{document}