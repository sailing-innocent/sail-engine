\begin{figure}[H]
    \includegraphics[width=\textwidth]{fig_pipeline_tgsca.png}
    \caption{The Overall Pipeline}
    \label{fig:pipeline}
\end{figure}

该论文的整体方法流程如图\ref{fig:pipeline}所示

\begin{figure}[H]
    \includegraphics[width=\textwidth]{fig_semantic_type_tgsca.png}        
    \caption{The Semantic Types}
    \label{fig:semantic_type}
\end{figure}

我们首先得到如图\ref{fig:semantic_type}所示的语义类型,包括:障碍类型,通道类型,出口-入口等。并输入一段文本描述,如:
\begin{quote}
    很多人在从地图右下角的出口处进入,然后经过地图中央偏右的通道,向左上角的入口处移动。另一处人群从地图左上角的出口处进入,然后经过地图中央偏左的通道,向右下角的入口处移动。
\end{quote}

这样的文本描述当然是不精确的,我们需要借用GPT-4将其转换为一个标准化的语义结构$\{s^j\}$然后对每一个语义结构进行CLIP编码,得到训练每一个扩散模型的潜空间信号$h^j$。最后通过两重扩散模型生成速度场$I_v^j$,起始点分布概率图$I_s^j$和目标点分布概率图$I_g^j$,并通过局部避障算法RVO进行速度场修正,最终得到一个合理的人群行为模拟。

\subsection{文本格式化}

\begin{figure}[H]
    \includegraphics[width=\textwidth]{fig_semantic_struct_tgsca.png}
    \caption{The Canonical Semantic Structure}
    \label{fig:semantic_struct}
\end{figure}

如图\ref{fig:semantic_struct}所示,我们首先将自然语言描述转化为一个标准化的语义结构,这个结构包含了若干个关键元素,包括:
场景描述,人群信息,出口-入口,规划路径上的特殊位置。

\subsection{扩散模型训练}

对于每一组人群i,我们需要生成如下三张地图:

\begin{enumerate}
    \item 起始点分布概率图$I_s^i$
    \item 目标点分布概率图$I_g^i$
    \item 速度场$I_v^i$
\end{enumerate}

但是经过前期测试,发现直接用一个扩散模型生成上述图像的效果并不好,所以研究者采取了两重扩散模型来解决这个问题,首先用一个扩散模型生成$I_s^i$和$I_g^i$,将生成好的起始-目标点分布作为约束,监督另一个扩散模型生成$I_v^i$。

潜空间扩散模型的原理如下:
\begin{enumerate}
    \item 给定一张真实图片,通过逐步加噪声的方式可以将其转化为一张噪声图片,这个过程可以看作是一个逐步扩散的过程,因此被称为扩散模型。
    \item 模型会在每一步加噪声的过程中,学习一个逆向去噪的变换,从而拥有了从噪声中逐步恢复出真实的图片的能力。
    \item 通过参数替换,每一个去噪变换最终可以转换为学习一个特殊的噪声分布$\epsilon_\theta(d^t,t,h)$
    \item 预测的图片分布:$d_{t-1}\sim \mathcal{N}(\frac{1}{\sqrt{\alpha_t}}(d_t-\frac{\beta_t}{\sqrt{1-\tilde{\alpha_t}}}\epsilon_\theta(d_t,t,h)), \sigma_t I)$
    \item 此处h是潜空间信号,可以用CLIP或者其他图片编码得到
\end{enumerate}

对于每一个组j,我们首先用CLIP嵌入得到$h^j$,然后用两个扩散模型分别训练起始-目标点分布和速度场,训练过程如下:
$$h^j=CLIP(s^j)$$

$$I^j_{s,g,t-1}= \frac{1}{\sqrt{\alpha_t}}(d_t-\frac{\beta_t}{\sqrt{1-\tilde{\alpha_t}}}\epsilon_\theta(I^j_{s,g,t},I_e,t,h^j))$$

$$I^j_{v,t-1}= \frac{1}{\sqrt{\alpha_t}}(d_t-\frac{\beta_t}{\sqrt{1-\tilde{\alpha_t}}}\epsilon_\theta(I^j_{v,t},I_e,I_s,I_g,t,h^j))$$

研究者在4块RTX 4090上训练了两个扩散模型,每个模型训练了50个epoch,速度场模型训练了200个epoch,总共训练了192小时。

\subsection{数据集构造}

为了构建训练的数据集,我们需要

\begin{enumerate}
    \item 利用语义元素变换生成若干语义地图,并通过人工挑选出合适的若干场景
    \item 随机自动在地图边缘生成出口-入口对
    \item 对于任意一个出口-入口对,生成一个规划路径,排除无法生成合理路径的场景
    \item 对规划可行路径生成标准化的语句描述
\end{enumerate}


\begin{figure}[H]
    \includegraphics[width=0.8\textwidth]{fig_vel_adjust_tgsca.png}
    \caption{Veloicity Field Adjustment}
    \label{fig:vel_adjust}
\end{figure}


在生成合理路径图之后,我们可以直接得到一个速度场,但是这个速度场不能避免智能体的局部碰撞,因此我们需要局部避障算法RVO\cite{vandenbergReciprocalVelocityObstacles2008}来进行速度场修正,如图\ref{fig:vel_adjust}所示。

