本文主要提出了一个基于学习的文本提示生成人群动画的方法,充分利用了先进的语言大模型和扩散模型技术,实现了对于人群动画生成的文本提示自动生成,提高了人群动画生成的效率和质量。

但与此同时,文章还存在一些不足之处,比如语义环境不能太复杂,构建训练的数据集需要大量的标注成本,生成的动画也无法处理更加复杂的交互动作。未来的工作可以考虑进一步提高生成的动画质量,增加动画的多样性,提高生成的效率,以及提高生成的动画的交互性。

最后有一个开放性的问题,现在我们已经有了GPT-4o,一个多模态输入输出的大语言模型,我们是否可以直接从这个模型中获取到所需的起点/终点/速度图呢?