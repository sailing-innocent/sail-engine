\begin{frame}
    \frametitle{方法}
    \begin{enumerate}
        \item 迭代发包策略
        \item 遗传编码计算IDR
        \item 训练集的构建
        \item 参数和训练细节
    \end{enumerate}
\end{frame}

\subsection{迭代发包策略}
研究者在这部分给出了迭代发包策略的定义。

一个传统的发包策略可以描述为一个优先级函数$\Sigma(\mathcal{J,M})$,根据一些已知的先验条件(比如工序的加工时间,截止日期,机器当前任务的完成时间等)分配了每一个工序在一系列机器上的优先级。当优先级确定之后,最高优先级的工序会被安排在下一个执行(等待直到满足可以执行的条件并执行)。这一类策略的局限性在于它只能知道决定阶段之前的条件,无法在后续随着情况变化而更新。

不同于传统的发包策略,可迭代发包策略可以描述为$\Sigma^I(\mathcal{M, J, R})$,此处$\mathcal{R}$为已经发生的调度结果的记录。整个算法如图\ref{fig:algorithm_idr}所示

\begin{figure}[H]
    \includegraphics[width=0.6\textwidth]{fig_algorithm_1_idr.png}
    \caption{The Iterative Dispatching Rule}
    \label{fig:algorithm_idr}
\end{figure}

在算法中$o_{j,l}$指的是对工序j的第l个操作,$L_k$是第k机器的就绪时间,$p(\sigma)$是每一个操作$\sigma$耗费的事件,$r(\sigma)$指的是该操作的就绪时间,$m(\sigma)$是操作$\sigma$需要执行的机器序号,$next(\sigma)$指的是在本序列中下一个操作,如果是最后一个操作,则为null。

在算法的最开始,记录为$\mathcal{R^0}$,4-12步重复计划了一个无延迟的规划,在每一个规划确定之后,都会与上一个目标函数$Obj*$对比,目标函数初始值为正无穷大,如果下一个目标函数值比上一个小,则更新目标函数值。重复构建新的目标策略,知道最终目标函数值没有变化。

迭代发包策略的时间复杂度小于$\left[\frac{Obj_I-LB}{\epsilon}\right]$,此处$Obj_I$是初始策略的目标函数,LB是问题的理想下界,$\epsilon$是最小的可能更新。明显这个迭代的发包策略的构建可以在有限时间内结束。

\subsection{遗传编码计算IDR}
\begin{frame}
    \frametitle{终端符号表}
    \begin{table}[H]
    \caption{Terminal Table}
    \centering
    \begin{tabular}{|c|c|}
        \hline
        RJ & Operation Ready Time \\
        RO & Number of remaining opeartions of the job \\
        RT & Remaining process time of the job \\
        PR & Operation processing time \\
        W  & Weight of the job \\
        DD & Due date of the job \\
        RM & Machine Ready Time \\
        \#  & Random Number from 0 to 1 \\
        RET & Recorded finish time \\
        RWT & Recorded operation waiting time \\
        RNWT & Recorded waiting time for next operation \\
        \hline      
    \end{tabular}
    \label{tab:terminal}
\end{table}
    \begin{quote}
        程序的名词表如表\ref{tab:terminal}所示,提供了从外部条件中获得的常量信息。
    \end{quote}
\end{frame}

\begin{frame}
    \begin{enumerate}
        \item 一个GP策略被表达为$\Delta^I(\mathcal{J,M,R})$,分配了每一个工序在序列中的优先级
        \item 运行在一系列实例$\mathbb{I}=\{I_1,\dots,I_T\}$上作为训练集
        \item 因为不同场景结果之间实际意义相差很大吗,我们会对其进行一定的归一化
            $$dev(\Delta^I,I_n)=\frac{Obj(\Delta^I,I_n)-Ref(I_n)}{Ref(I_n)}$$
        \item 其中$Obj(\Delta^I,I_n)$是目标函数值,$Ref(I_n)$是当前实例上的参考目标值。这样我们就能得到一个GP系统的适应性指标
        \item $$dev_{avg}(\Delta^I)=\frac{\sum\limits_{I_n\in \mathbb{I}}\limits^{} dev(\Delta^I, I_n)}{|\mathbb{I}|}$$
    \end{enumerate}
\end{frame}

\begin{frame}
    \frametitle{遗传编码更新算法}
    \begin{figure}[H]
        \includegraphics[width=0.6\textwidth]{fig_algorithm_2_idr.png}
        \caption{GP Algorithmn to envolve IDRs}
        \label{fig:algorithm_gp_idr}
    \end{figure}
    整体算法如图\ref{fig:algorithm_gp_idr}所示,更多对GP符号组合的探讨会在后文展开。
\end{frame}
\subsection{训练细节}
\begin{frame}
    \frametitle{训练集的构建}
    \begin{figure}[H]
        \includegraphics[width=0.6\textwidth]{fig_dataset_idr.png}
        \caption{Used Dataset}
        \label{fig:dataset_idr}
    \end{figure}
    \begin{quote}
    图\label{fig:dataset_idr}展示了研究者用到的数据集。研究者按照奇数偶均分训练集和测试集。这样可以保证训练集和测试集的分布尽可能相似。
    \end{quote}    
\end{frame}
\begin{frame}
    \frametitle{初始化参数}
    \begin{table}[H]
    \caption{The HyperParameters of GP}
    \begin{tabular}{|c|c|}
        \hline 
        Population Size & 1000 \\
        Crossover Rate & 90\% \\
        Mutation Rate & 5\% \\
        Reproduction Rate & 5\% \\
        Generations & 50 \\
        Max Depth & 8 \\
        Operator Set & $if,+,-,\times,\%,min,max,abs$ \\
        \hline 
    \end{tabular}
    \label{tab:hyper_param}
\end{table}
    \begin{quote}
        为了让模型达到最优的效果,需要搜索最合适的参数组合。表\ref{tab:hyper_param}展示了初始化的超参数。
        
        研究者采用了一种随机对半分的策略来初始化遗传编程种群,具体来说,就是将一半数目的种群进行“全初始化”,达到语法树的最大深度并包含全部的算子。另一半则没有这一层要求,倾向于让其“生长”。
    \end{quote}    
\end{frame}


\begin{frame}
    \frametitle{遗传编程操作}
    \begin{figure}[H]
        \includegraphics[width=0.4\textwidth]{fig_gp_subtree_idr.png}
        \caption{遗传编程操作}
        \label{fig:gp_subtree_idr}
    \end{figure}    
\end{frame}