\begin{frame}
    \frametitle{终端符号表}
    \begin{table}[H]
    \caption{Terminal Table}
    \centering
    \begin{tabular}{|c|c|}
        \hline
        RJ & Operation Ready Time \\
        RO & Number of remaining opeartions of the job \\
        RT & Remaining process time of the job \\
        PR & Operation processing time \\
        W  & Weight of the job \\
        DD & Due date of the job \\
        RM & Machine Ready Time \\
        \#  & Random Number from 0 to 1 \\
        RET & Recorded finish time \\
        RWT & Recorded operation waiting time \\
        RNWT & Recorded waiting time for next operation \\
        \hline      
    \end{tabular}
    \label{tab:terminal}
\end{table}
    \begin{quote}
        程序的名词表如表\ref{tab:terminal}所示,提供了从外部条件中获得的常量信息。
    \end{quote}
\end{frame}

\begin{frame}
    \begin{enumerate}
        \item 一个GP策略被表达为$\Delta^I(\mathcal{J,M,R})$,分配了每一个工序在序列中的优先级
        \item 运行在一系列实例$\mathbb{I}=\{I_1,\dots,I_T\}$上作为训练集
        \item 因为不同场景结果之间实际意义相差很大吗,我们会对其进行一定的归一化
            $$dev(\Delta^I,I_n)=\frac{Obj(\Delta^I,I_n)-Ref(I_n)}{Ref(I_n)}$$
        \item 其中$Obj(\Delta^I,I_n)$是目标函数值,$Ref(I_n)$是当前实例上的参考目标值。这样我们就能得到一个GP系统的适应性指标
        \item $$dev_{avg}(\Delta^I)=\frac{\sum\limits_{I_n\in \mathbb{I}}\limits^{} dev(\Delta^I, I_n)}{|\mathbb{I}|}$$
    \end{enumerate}
\end{frame}

\begin{frame}
    \frametitle{遗传编码更新算法}
    \begin{figure}[H]
        \includegraphics[width=0.6\textwidth]{fig_algorithm_2_idr.png}
        \caption{GP Algorithmn to envolve IDRs}
        \label{fig:algorithm_gp_idr}
    \end{figure}
    整体算法如图\ref{fig:algorithm_gp_idr}所示,更多对GP符号组合的探讨会在后文展开。
\end{frame}