\begin{frame}
    \frametitle{迭代发包策略}
    \begin{enumerate}
        \item 一个传统的发包策略可以描述为一个优先级函数$\Sigma(\mathcal{J,M})$
        \item 不同于传统的发包策略,可迭代发包策略可以描述为$\Sigma^I(\mathcal{M, J, R})$,此处$\mathcal{R}$为已经发生的调度结果的记录。整个算法如图\ref{fig:algorithm_idr}所示
        \item $o_{j,l}$指的是对工序j的第l个操作
        \item $L_k$是第k机器的就绪时间
        \item $p(\sigma)$是每一个操作$\sigma$耗费的事件\
        \item $r(\sigma)$指的是该操作的就绪时间
        \item $m(\sigma)$是操作$\sigma$需要执行的机器序号
        \item $next(\sigma)$指的是在本序列中下一个操作,如果是最后一个操作,则为null
    \end{enumerate}
\end{frame}

\begin{frame}
    \frametitle{迭代发包策略}
    \begin{figure}[H]
        \includegraphics[width=0.6\textwidth]{fig_algorithm_1_idr.png}
        \caption{The Iterative Dispatching Rule}
        \label{fig:algorithm_idr}
    \end{figure}
\end{frame}

\begin{frame}
    \frametitle{算法分析}
    \begin{enumerate}
        \item 在算法的最开始,记录为$\mathcal{R^0}$,4-12步重复计划了一个无延迟的规划,在每一个规划确定之后,都会与上一个目标函数$Obj*$对比,目标函数初始值为正无穷大,如果下一个目标函数值比上一个小,则更新目标函数值。重复构建新的目标策略,知道最终目标函数值没有变化。

        \item 迭代发包策略的时间复杂度小于$\left[\frac{Obj_I-LB}{\epsilon}\right]$,此处$Obj_I$是初始策略的目标函数,LB是问题的理想下界,$\epsilon$是最小的可能更新。明显这个迭代的发包策略的构建可以在有限时间内结束。
    \end{enumerate}
\end{frame}