\begin{frame}
    \frametitle{算法分析}
    \begin{enumerate}
        \item 结果分析
        \item 进一步优化
    \end{enumerate}
\end{frame}

\subsection{结果分析}
\begin{frame}
    \frametitle{$J_m|C_{max}$结果}
    \begin{figure}
        \includegraphics[width=0.6\textwidth]{fig_result_1_idr.png}
        \caption{$J_W|C_{max}$结果}
        \label{fig:result_1}
    \end{figure}
\end{frame}

\begin{frame}
    \frametitle{$J_W|\sum \omega_jT_j$结果}
    \begin{figure}
        \includegraphics[width=0.6\textwidth]{fig_result_2_idr.png}
        \caption{$J_W|\sum \omega_jT_j$结果}
        \label{fig:result_2}
    \end{figure}    
\end{frame}

\subsection{进一步优化}

\begin{frame}
    \frametitle{优化方向}
    \begin{enumerate}
        \item IDR-P:引入假节点的IDR    
        \begin{quote}
            为了避免不合适的初始化导致最终的结果变差。在之前的讨论中研究者采用了一种启发式的方式来构建$\mathcal{R}^0$,但同时我们也可以将其看做一个常量函数,作为语法树中的一个节点加入整体的遗传编程迭代过程中。
        \end{quote}
        \item IDR-VNS:引入邻域参数搜索的IDR
    \end{enumerate}

\end{frame}
\begin{frame}
    \frametitle{IDR的参数邻域搜索}
    为避免陷入局部最优化,研究者引入参数邻域搜索(Variable Neighbor Search, VNC)算法如图\ref{fig:vns_idr}所示
    \begin{figure}[H]
        \centering
        \includegraphics[width=0.6\textwidth]{fig_algorithm_3_idr.png}
        \caption{带邻域搜索的IDR策略}
        \label{fig:vns_idr}
    \end{figure}
\end{frame}
\begin{frame}
    \frametitle{优化IDR结果对比}

    \begin{figure}[H]
        \centering
        \includegraphics[width=0.6\textwidth]{fig_result_3_1_idr.png}
        \caption{$J_W|C_{max}$结果}
        \label{fig:result_enhanced_1}
    \end{figure}
\end{frame}

\begin{frame}
    \frametitle{优化IDR结果对比}
    \begin{figure}[H]
        \centering
        \includegraphics[width=0.6\textwidth]{fig_result_3_2_idr.png}
        \caption{$J_W|\sum \omega_jT_j$结果}
        \label{fig:result_enhanced_2}
    \end{figure}
\end{frame}