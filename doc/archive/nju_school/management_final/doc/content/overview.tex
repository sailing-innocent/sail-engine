对于发包策略(dispatching rule)的研究开始很早,一般来说,发包策略指的是一个优先级函数,给每一个工序调度映射一个值来表明其优先程度,从而在调度中优先。发包策略在车间调度问题中有广泛的应用,但是有两个关键的不足

\begin{enumerate}
    \item 成功的发包策略函数非常复杂,且需要很多关于工序的先验知识
    \item 发包策略的效果泛化性上往往比不上使用优化方法得到的策略
\end{enumerate}

对于第一点,很多机器学习方法可以自动化设计这个过程,其中遗传编程(Genetic Programming)因为其简单,灵活的特点,逐渐成为最流行的方法。方法证明基于遗传编程的发包策略可以比以往方法更加优秀。

但是,现在得到的策略往往无法进行泛化,往往只能针对静态的,理想的情景而无法扩展到真实场景。有一些基于元启发式的算法,如遗传算法(Genetic Algorithm, GA)等可以帮助得到更好的发包策略,虽然起到了很好的效果,但这些方法选出的发包策略往往没有直接设计的方法直观,且计算也更加复杂。

这篇研究中提出了迭代发包策略(iterative dispatching rule, IDR),不同于以往针对一个问题固定的发包策略,迭代发包策略会随着调度进程不断调整自身策略,从之前的失败调度中逐渐调整优化到最终靠近最优的发包策略。但是因为需要采集和依赖的信息变多,设计这一类迭代发包策略也变得复杂,本研究核心贡献如下:

\begin{enumerate}
    \item 设计了一个基于遗传编程的迭代发包策略生成算法
    \item 对比了迭代发包策略生成算法和之前固定策略的优劣
    \item 分析了迭代发包策略算法,并提升了他们的表现
\end{enumerate}

本篇论文主要是采用遗传编码算法学习了一个可迭代的发包策略,从而高效解决了车间调度问题。下面第\ref{chap:method}章会用来介绍具体的算法细节和实现过程,第\ref{chap:analysis}章会用来介绍对于算法结果的分析和优化技巧,最终在第\ref{chap:conclusion}章进行一个简单的总结与展望。