研究者在这部分给出了迭代发包策略的定义。

一个传统的发包策略可以描述为一个优先级函数$\Sigma(\mathcal{J,M})$,根据一些已知的先验条件(比如工序的加工时间,截止日期,机器当前任务的完成时间等)分配了每一个工序在一系列机器上的优先级。当优先级确定之后,最高优先级的工序会被安排在下一个执行(等待直到满足可以执行的条件并执行)。这一类策略的局限性在于它只能知道决定阶段之前的条件,无法在后续随着情况变化而更新。

不同于传统的发包策略,可迭代发包策略可以描述为$\Sigma^I(\mathcal{M, J, R})$,此处$\mathcal{R}$为已经发生的调度结果的记录。整个算法如图\ref{fig:algorithm_idr}所示

\begin{figure}[H]
    \includegraphics[width=0.6\textwidth]{fig_algorithm_1_idr.png}
    \caption{The Iterative Dispatching Rule}
    \label{fig:algorithm_idr}
\end{figure}

在算法中$o_{j,l}$指的是对工序j的第l个操作,$L_k$是第k机器的就绪时间,$p(\sigma)$是每一个操作$\sigma$耗费的事件,$r(\sigma)$指的是该操作的就绪时间,$m(\sigma)$是操作$\sigma$需要执行的机器序号,$next(\sigma)$指的是在本序列中下一个操作,如果是最后一个操作,则为null。

在算法的最开始,记录为$\mathcal{R^0}$,4-12步重复计划了一个无延迟的规划,在每一个规划确定之后,都会与上一个目标函数$Obj*$对比,目标函数初始值为正无穷大,如果下一个目标函数值比上一个小,则更新目标函数值。重复构建新的目标策略,知道最终目标函数值没有变化。

迭代发包策略的时间复杂度小于$\left[\frac{Obj_I-LB}{\epsilon}\right]$,此处$Obj_I$是初始策略的目标函数,LB是问题的理想下界,$\epsilon$是最小的可能更新。明显这个迭代的发包策略的构建可以在有限时间内结束。
