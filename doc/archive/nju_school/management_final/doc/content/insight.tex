在算法的基础上,作者尝试进一步分析如何最优化可迭代的发包策略。与过往只实现单独策略$\pi$不同,一个可迭代的发包策略最终会产生一系列的发包策略序列 $\pi^0,\pi^1,\pi^2,\dots,\pi^n$

\subsection{IDR的初始化}

为了避免不合适的初始化导致最终的结果变差。在之前的讨论中研究者采用了一种启发式的方式来构建$\mathcal{R}^0$,但同时我们也可以将其看做一个常量函数,作为语法树中的一个节点加入整体的遗传编程迭代过程中。

\subsection{IDR的参数邻域搜索}

陷入局部最优化是车间调度问题的常见现象,过去算法克服这一难点的主要方式是引入邻域搜索算法如模拟退火,禁忌搜索等。但是这些方法在迭代发包策略中的应用较为困难。

为此,研究者引入参数邻域搜索(Variable Neighbor Search, VNC)算法如图\ref{fig:vns_idr}所示

\begin{figure}[H]
    \centering
    \includegraphics[width=0.6\textwidth]{fig_algorithm_3_idr.png}
    \caption{带邻域搜索的IDR策略}
    \label{fig:vns_idr}
\end{figure}

\subsection{优化后的IDR结果}

经过上述两个优化最终得到如图\ref{fig:result_enhanced_1}和\ref{fig:result_enhanced_2}所示的结果

\begin{enumerate}
    \item IDR-P:引入假节点的IDR
    \item IDR-VNS:引入邻域参数搜索的IDR
\end{enumerate}

\begin{figure}[H]
    \centering
    \includegraphics[width=0.6\textwidth]{fig_result_3_1_idr.png}
    \caption{$J_W|C_{max}$结果}
    \label{fig:result_enhanced_1}
\end{figure}

\begin{figure}[H]
    \centering
    \includegraphics[width=0.6\textwidth]{fig_result_3_2_idr.png}
    \caption{$J_W|\sum \omega_jT_j$结果}
    \label{fig:result_enhanced_2}
\end{figure}

\subsection{前瞻策略}

前瞻策略(Look Ahead Strategy)换成动态的生成