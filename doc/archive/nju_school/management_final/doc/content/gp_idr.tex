上文所述可迭代的发包策略并不容易直接计算得到,所以作者提出了一种基于遗传编码的算法来计算这个可迭代的发包策略。

在这个遗传编码策略中,支持四种简单的二元算子$+,-,\times,\%$和一些常用的函数算子$abs,min,max$。引入条件算子$if$来构建相对复杂的程序。程序的名词表如表\ref{tab:terminal}所示,提供了从外部条件中获得的常量信息。

\begin{table}[H]
    \caption{Terminal Table}
    \centering
    \begin{tabular}{|c|c|}
        \hline
        RJ & Operation Ready Time \\
        RO & Number of remaining opeartions of the job \\
        RT & Remaining process time of the job \\
        PR & Operation processing time \\
        W  & Weight of the job \\
        DD & Due date of the job \\
        RM & Machine Ready Time \\
        \#  & Random Number from 0 to 1 \\
        RET & Recorded finish time \\
        RWT & Recorded operation waiting time \\
        RNWT & Recorded waiting time for next operation \\
        \hline      
    \end{tabular}
    \label{tab:terminal}
\end{table}

在这个系统中,一个GP策略被表达为$\Delta^I(\mathcal{J,M,R})$,分配了每一个工序在序列中的优先级。最开始的时候,在没有任何记录情况下初始化一个GP,RFT的初始值是当前策略的总执行时间,RWT和RNWT的初始值则设定为过半机器完成的时间。

为了验证生成策略的有效性,需要运行在一系列实例$\mathbb{I}=\{I_1,\dots,I_T\}$上作为训练集,记录所有运行结果。因为不同工序结果之间实际意义相差很大吗,我们会对其进行一定的归一化

$$dev(\Delta^I,I_n)=\frac{Obj(\Delta^I,I_n)-Ref(I_n)}{Ref(I_n)}$$

其中$Obj(\Delta^I,I_n)$是目标函数值,$Ref(I_n)$是当前实例上的参考目标值。这样我们就能得到一个GP系统的适应性指标

$$dev_{avg}(\Delta^I)=\frac{\sum\limits_{I_n\in \mathbb{I}}\limits^{} dev(\Delta^I, I_n)}{|\mathbb{I}|}$$

在目标函数为$J_m||C_{max}$时候,参考目标为其他方法所获得的目标值下界。因为取到了下界,所以此处的dev永远非负。如果适应指标接近0,则以为着GP可以迭代产生近似最优的解法。对于$J_m||\sum \omega_jT_J$ 情况,下界不是总能取得,我们会采用EDD方法得到的参考目标。

\begin{figure}[H]
    \includegraphics[width=0.6\textwidth]{fig_algorithm_2_idr.png}
    \caption{GP Algorithmn to envolve IDRs}
    \label{fig:algorithm_gp_idr}
\end{figure}

整体算法如图\ref{fig:algorithm_gp_idr}所示,更多对GP符号组合的探讨会在后文展开。