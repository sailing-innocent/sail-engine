\documentclass{article}

\usepackage{arxiv/arxiv}

\usepackage{amsmath}
\usepackage[utf8]{inputenc} % allow utf-8 input
\usepackage[T1]{fontenc}    % use 8-bit T1 fonts
\usepackage{hyperref}       % hyperlinks
\usepackage{url}            % simple URL typesetting
\usepackage{booktabs}       % professional-quality tables
\usepackage{amsfonts}       % blackboard math symbols
\usepackage{nicefrac}       % compact symbols for 1/2, etc.
\usepackage{microtype}      % microtypography
\usepackage{cleveref}       % smart cross-referencing
\usepackage{lipsum}         % Can be removed after putting your text content
\usepackage{graphicx}
\usepackage{natbib}
\usepackage{doi}

\title{Deep Reinforcement Learning Homework 01}

\author{
    \hspace{1mm}Zhu Zihang \\
    Nanjing University\\
    Nanjing, Jiangsu Province\\
    \texttt{522022150087@smail.nju.edu.cn}
}

\hypersetup{
    pdftitle={drl_homework_01},
    pdfsubject={},
    pdfauthor={Zihang Zhu},
    pdfkeywords={Deep Reinforcement Learning, Homework 01},
}

\begin{document}

\maketitle

\paragraph{Homework 01: Policy Gradient }

Policy Gradient is a method 
that optimize the policy function $\pi(a,s)$ directly.
In this section we are going to use Policy Gradient Method to 
solve two problems 

\begin{enumerate}
    \item the point maze navigation problem
    \item the MuJoco HalfCheet running problem
\end{enumerate}

We are going to implement 

\begin{enumerate}
    \item Vanilla Policy Gradient -- REINFORCE
    \item Natural Gradient Policy
    \item Trust-Region Policy Gradient -- TRPO 
    \item Proximal Policy Gradient -- PPO
\end{enumerate}

\paragraph{Policy Gradient}

The goal of reinforcement learning 

$p_{\theta}(\tau)=p_{\theta}(s_1,a_1,\dots, s_T,a_T)=p(s_1)\prod_{t=1}^T\pi_{\theta}(a_t,s_t)p(s_{t+1}|s_t,a_t)$

$\theta^{*}=argmax_{\theta}E_{\tau\sim p_\theta(\tau)}[\Sigma_t r(s_t, a_t)]$

$\tau$ means a trajectory, 
our goal is to make the expectation of reward maximize, 
but we have no idea what the reward may be. 

So we can use "Sampling" method, 
replay for multiple times and evaluate 
the expectation of reward according to the sample's reward.

e.g. replay N times

$J(\theta)=E_{\tau\sim p_{\theta}(\tau)}[\Sigma_t{r(s_t,a_t)}]\approx \frac{1}{N}\Sigma_{i=1}^N\Sigma_t r(s_{i,t},a_{i,t})$

direct policy differentiation

$J(\theta)=\int p_\theta(\tau)r(\tau)d\tau$

given $r_{\tau}=\Sigma_{t=1}^T r(s_t,a_t)$ is constant for $\theta$

$\nabla_{\theta}J(\theta)=\int\nabla_{\theta}p_{\theta}(\tau)r(\tau)d\tau=\int p_{\theta}(\tau)\nabla_{\theta}\log{p_{\theta}(\tau)}r(\tau)d\tau=E_{\tau\sim p_{\theta}(\tau)}[\nabla_{\theta}\log{p_\theta(\tau)r(\tau)}]$

Then we want to calculate $\log{p_\theta(\tau)}$

given $p_{\theta}(\tau)=p(s_1)\prod_{t=1}^T\pi_{\theta}(a_t|s_t)p(s_{t+1}|s_t,a_t)$

Then the log could convert the prod to sum

$\log{p_{\theta}(\tau)} = \log{p(s_1)}+\Sigma_{t=1}^T\log{\pi_\theta(a_t|s_t)+\log{p(s_{t+1}|s_t,a_t)}}$


now we may find that, the $\log{p(s_1)}$ and $\log{p(s_{t+1}|s_t,a_t)}$ is independent on $\theta$, so now we can transform 

$\nabla_{\theta}J(\theta)=E_{\tau\sim p_{\theta}(\tau)}[(\Sigma_{t=1}^T\nabla_{\theta} \log{\pi_{\theta}(a_t|s_t)})(\Sigma_{t=1}^T r(s_t,a_t))]$

Evaluate the policy gradient

recall we have the approximation 

$J(\theta)\approx \frac{1}{N}\Sigma_i\Sigma_t r(s_{i,t},a_{i,t})$

Thus we can approximate the gradient 

$\nabla_\theta J(\theta) \approx \frac{1}{N}\Sigma_{i=1}^N(\Sigma_{t=1}^T\nabla_\theta\log{\pi_{\theta}(a_{i,t}|s_{i,t})})(\Sigma_{t}r(s_t, a_t))$

Then $\theta = \theta + \alpha \nabla_{\theta}J(\theta)$

\subparagraph{REINFORCE algorithm}

\begin{enumerate}
\item sample $\{\tau^i\}$ from $\pi_{\theta}(a_t, s_t)$ (run the policy)
\item $\nabla_{\theta}J(\theta)\approx \Sigma_i((\Sigma_t \nabla_\theta \log{\pi_\theta(a_t^i,s_t^i)})(\Sigma_t r(s_t^i,a_t^i)))$
\item $\theta\leftarrow \theta + \alpha\nabla_\theta J(\theta)$
\end{enumerate}

What is $\Sigma_t \nabla_\theta \log{\pi_\theta(a_{i,t},s_{i,t})}$?

Comparison to maximum likelihood

$\pi_\theta(a_t|s_t)$ could be something like neural network. 
For example, Gaussian Policies will use a neural network 
to generate the mean of distribution of policy

$\pi_{\theta}(a_t|s_t)=\mathcal{N}(f_{nn}(s_t);\Sigma)$

$\log{\pi_\theta(a_t|s_t)}=-\frac{1}{2}|f(s_t)-a_t|_{\Sigma}^2+ const$

$\nabla_{\theta}\log{\pi_\theta(a_t|s_t)}=-\frac{1}{2}\Sigma^{-1}(f(s_t)-a_t)\frac{df}{d\theta}$

But we may see, that the $\alpha$ is hard to set for vanilla policy gradient 
and the matrix is prone to ill-cases and hard to solve. 

Thus, we will introduce Natural Gradient Method.

\subparagraph{Natural Policy Gradient}

We can easily calculate the distance between two points in a Euclidean Coordinate, 
but how to calculate the distance between two probability distribution $p(x)$ and $q(x)$?

$$D_{KL}(p(x)||q(x))=\Sigma_i{p(x_i)log\frac{p(x_i)}{q(x_i)}}$$ 

and for Discrete Distribution

$$D_{KL}(p(x)||q(x))=\int_x p(x)\log{\frac{p(x)}{q(x)}}dx$$

The KL divergence of 0 indicates the two distributions are identical.

Suppose two Gaussian Distribution:

$p(x)=\mathcal{N}(\mu_1,\sigma_1^2)$ and $q(x)=\mathcal{N}(\mu_2,\sigma_2^2)$



The KL divergence of p and q is the Hessian of KL, we can estimate it with Fisher information matrix (FIM)

\begin{itemize}
\item Hessian: A square matrix of second-order partial derivatives
\item Fisher information: a way of measuring the amount of infromation than an observable random variable X carries about an unknown parmter $\theta$ upon which the probability of X depends
\end{itemize}

$F(\theta)=E_{x\sim \pi_{\theta}}(\nabla_\theta(\log{\pi_\theta(x)})\nabla_{\theta}\log{\pi_{\theta}(x)^T})$
\begin{itemize}
\item input: intial policy parameters $\theta_0$
\item for k = 0,1, 2...
    \begin{itemize}
        \item  collect trajectory $\mathcal{D}_k$ on policy $\pi_k=\pi(\theta_k)$
        \item estimate advantages $\hat{A_t^{\pi_k}}$ using any advantage estimation algorithm
        \item form sample estimates for policy gradient g and KL-divergence Hessian / Fisher Information Matrix $\hat{H}_k$
    \end{itemize}
\item Compute Natural Policy Gradient update: $\theta_{k+1}=\theta + \sqrt{\frac{2\epsilon}{\hat{g}_k^T\hat{H}_k\hat{g}_k}}\hat{H}_k^{-1}\hat{g}_k$
\end{itemize}

\subparagraph{Fisher Informantion Matrix}


Suppose we have a model parameterized by parameter vector $\theta$ that models a distribution $p(x|\theta)$. In freqentist statistics, the way we learn $\theta$ is to maximize the likelihood $p(x|\theta)$ wrt. parameter $\theta$

$s(\theta)=\nabla_{\theta}\log{p(x|\theta)}$

That is, score function is the gradient of log likelihood function.

The expected value of score wrt. our model is zero:

\begin{equation}
    \begin{aligned}
        \mathop{\mathbb{E}}\limits_{p(x|\theta)}[s(\theta)] 
        & = \mathop{\mathbb{E}}\limits_{p(x|\theta)}[\nabla{\log{p(x|\theta)}}] \\
        & = \int{\nabla{\log{p(x|\theta)}p(x|\theta)dx}} \\
        & = \int{\frac{\nabla p(x|\theta)}{p(x|\theta)}p(x|\theta)dx} \\ 
        & = \int{\nabla p(x|\theta) dx} \\ 
        & = \nabla 1 = 0
    \end{aligned}
\end{equation}

But how certain are we to our estimate? we can define an uncertainty measure around the expected estimates

$\mathop{\mathbb{E}}\limits_{p(x|\theta)}[ (s(\theta) - 0)(s(\theta) - 0)^T ]$

The convariance of score function above is the definition of Fisher Information. As we assume the Fisher Information is in a matrix form, called Fisher Information Matrix:

$F = \mathop{\mathbb{E}}\limits_{p(x|\theta)}[ \nabla\log p(x_i|\theta) \nabla\log (x_i|\theta)^T ]$

However, usually our likelihood si complicated and computing the expectation is intractable, we can approximate the expectation of F using empirical distribution $\hat{q}(x)$, which is given by our training data $X=\{x_1, x_2, \dots, x_N\}$.

In this form, Fisher information matrix is called Empirical Fisher:

$F=\frac{1}{N}\sum\limits_{i=1}\limits^{N}\nabla p(x_i|\theta)\nabla \log{p(x_i|\theta)^T}$

We may find that the Fisher Information Matrix is the interpretation of negative expected Hessian of our model's log likelihood.

The negative expected Hessian of log likelihood is equal to the Fisher Information Matrix F:

Supppose the gradient of log likelihood funciton : $g(x|\theta)=\frac{\nabla p(x|\theta)}{p(x|\theta)}$

[[Hessian|notes.math.elementary.hessian]] is given by the [[Jacobian|notes.math.elementary.jacobian]] of its gradient:

$H_{\log p(x|\theta)}=J(g(x|\theta))$

given quotient rule of derivative $\frac{\partial (\frac{\nabla_{\theta}p}{p})_j}{\partial x_i}=\frac{\frac{\partial (\nabla_\theta p)_j}{\partial x_i} - \frac{\partial p}{\partial x_i}(\nabla_\theta p)_j}{p^2}$

So $J(\frac{\nabla_\theta p}{p})=\frac{J(\nabla_\theta p)p-(\nabla_\theta p)(\nabla_\theta p)^T}{p(x|\theta)^2}$

Thus $H_{\log p(x|\theta)}=\frac{H_{p(x|\theta)}p(x|\theta)}{p(x|\theta)p(x|\theta)}-\frac{(\nabla_\theta p)(\nabla_\theta p)^T}{p(x|\theta)p(x|\theta)}=\frac{H_{p(x|\theta)}}{p(x|\theta)}-(\frac{\nabla_\theta p}{p(x|\theta)})(\frac{\nabla_\theta p}{p(x|\theta)})^T$

where we have 

\begin{equation}
    \begin{aligned}
    \mathop{\mathbb{E}}\limits_{p(x|\theta)}[ H_{\log{p(x|\theta)}}]
    & = \mathop{\mathbb{E}}\limits_{p(x|\theta)}[ \frac{H_{p(x|\theta)}}{p(x|\theta)} ] - \mathop{\mathbb{E}}\limits_{p(x|\theta)}[ (\frac{\nabla_\theta p}{p(x|\theta)})(\frac{\nabla_\theta p}{p(x|\theta)})^T ] \\
    & = \int \frac{H_{p(x|\theta)}}{p(x|\theta)} p(x|\theta)dx - \mathop{\mathbb{E}}\limits_{p(x|\theta)}[ \nabla_\theta log(p(x|\theta)) \nabla_\theta log(p(x|\theta))^T] \\ 
    & = H_{\int p(x|\theta)dx} - F \\ 
    & = H_1 - F = -F
    \end{aligned}
\end{equation}

Thus we have 

$F = -\mathop{\mathbb{E}}\limits_{p(x|\theta)}[ H_{\log{p(x|\theta)}} ]$

\paragraph{Experiment}

Here we implement our method based on pytorch and conduct our experiments based on gym.

\subparagraph{Point Navigation}

\begin{itemize}
\item inherited from gym.Environment
\item action space: $[-0.1, 0.1]^2$ the movement of point
\item observation with target position $[-0.5,0.5]^2$ and agent position $[-0.5,0.5]^2$
\end{itemize}

\subparagraph{MuJoco HalfCheet}

This environment is part of the Mujoco envrionmetns which contains general information about the environment

\begin{itemize}
\item Action Space: Box(-1.0, 1.0, (6,), float32)
    \begin{itemize}
    \item dim 6
    \item 0 torque applied on the back thigh rotor 
    \item back shin rotor
    \item back foot rotor
    \item front thigh rotor
    \item front shin rotor
    \item front foot rotor
    \end{itemize}
\item Observatoion Space Box(-inf, inf, (17,), float64), positional values of different body parts of the cheetah
    \begin{itemize}
        \item by default, observations to not include x-coordinate of the cheetah's center of mass, it may be included by passing 
            exclude\_current\_positions\_from\_observation=False, then the observation dim becomes 18
        \item first 8 are position
        \item last 9 are velocity
        \item episode end: truncates when the episode length is greater than 1000
    \end{itemize}
\item import \textit{gymnasium.make("HalfCheetah-v4")}
\item A Cat-Like Robot Real-Time Learning to Run
\item 2-dim robot consisting of 9 links and 8 joints connecting them
\item goal: apply the torque on the joints to make the cheetah run forward as fast as possible
\item reward: the distance, positive for moving forward and negative for moving backward
    \begin{itemize}
        \item forward\_reward: A reward of moving forward which is measured as \textit{forward\_reward\_weight * (X-coordinate-before - x-coordinate-after)/dt}
        \item ctrl\_cost: penalising the cheetah if it takes actions that are too large
    \end{itemize}
\item the torso and head of the cheetah are fixed
\item torque can only be applied on the other 6 joints over the front and back thighs
\end{itemize}

\paragraph{Code Framework}

our result is reserved on this public git repository: \url{https://github.com/sailing-innocent/drl_homework}

\paragraph{Result}

\subparagraph{Policy Gradient + Navigation 2D} with 500 episods and learning rate -0.0005

We can see the reward is apprximate to 0

\subparagraph{Policy Gradient + HalfCheetah} with 5000 episodes and learning rate -0.0005

The reward gradually move to positive (means the robot walks!) and has an average to 1500.

\subparagraph{Natural Policy Gradient + Nav2D} with 500 episods and learning rate -0.0005

as you can see, for the same parameter it indeed get converged rapidly.

The training for our NGP on HalfCheetah has some trouble with
 the ill-posed matrix computing eigen value, 
 currently I don't have any idea how to save it.


\end{document}