
\newcommand\EqRef[1]{(\ref{#1})}
\newcommand\SepLine{~\\}

\begin{frame}{XPBD Solver}
XPBD is a simple deformable simulation method, which is:
\begin{itemize}
    \item fast(matrix-free),
    \item easy to implement,
    \item constraint-based ,
    \item stable.
\end{itemize}
can be used to simulate:
\begin{itemize}
    \item cloth : mass-spring, strain based ...
    \item hair/rod: rod dynamics ...
    \item soft body: strain based ...
    \item rigid body: shape matching ...
    \item fluid
\end{itemize}
\end{frame}

\begin{frame}
XPBD starts with Newton's equation of motion given by:
\begin{equation}
    \mathbf{M} \ddot{\mathbf{x}}=-\nabla U\left(\mathbf{x}\right)^T + \mathbf{f}_{ext},\label{xpbd-newton}
\end{equation}

\begin{table}[]
\begin{tabular}{ll}
    $\mathbf{M}$ &  the lumped mass matrix \\
    $\mathbf{x}$ & $\left({\mathbf{x}_0}^T,{\mathbf{x}_1}^T,{\mathbf{x}_2}^T,...\right)^T$  \\
    $U$          & the elastic potential \\
    $\mathbf{f}_{ext}$ & the stacked external force vector
\end{tabular}
\end{table}

Apply the implicit Euler discretization to Eq. \ref{xpbd-newton}:
\begin{equation}
    \mathbf{M}\frac{\mathbf{x}-\hat{\mathbf{x}}}{\Delta t^2}=-\nabla U(\mathbf{x})^T, 
    \label{ipg}
\end{equation}
where $\mathbf{x}:= \mathbf{x}(t + \Delta t)$ and $\hat{\mathbf{x}}= 
2 \cdot \mathbf{x}(t)
- \mathbf{x}(t - \Delta t)
+ \mathbf{M}^{-1} \cdot \mathbf{f}_{ext} \cdot {\Delta t}^2$.
\end{frame}

\begin{frame}


All potentials in the XPBD framework are formulated as squared constraints:
\begin{equation}
    U\left(\mathbf{x}\right)=\frac{1}{2} \mathbf{C}\left(\mathbf{x}\right)^{T} \boldsymbol{\alpha}^{-1} \mathbf{C}\left(\mathbf{x}\right), \nonumber
\end{equation}
where $\alpha$ is the diagonal compliance matrix describing the inverse stiffness of all constraints. 

\SepLine

By taking the negative gradient of the potential term, we obtain the elastic force:

\begin{equation}
    \mathbf{f}_{\text {elastic}}=-\nabla U^T=-\nabla \mathbf{C}^T \boldsymbol{\alpha}^{-1} \mathbf{C} = \frac{1}{{\Delta t}^2} \nabla \mathbf{C}^T \boldsymbol{\lambda},  \nonumber
\end{equation}

where 
$\boldsymbol{\lambda}=-\boldsymbol{\tilde{\alpha}}^{-1} \mathbf{C}\left(\mathbf{x}\right)$
with $\boldsymbol{\tilde{\alpha}}=\frac{\boldsymbol{\alpha}}{\Delta t^2}$.

\end{frame}

\begin{frame}
Thus, we obtain two equations available for iteration:
\begin{align}
    \mathbf{M}\left(\mathbf{x}-\tilde{\mathbf{x}}\right)-\nabla \mathbf{C}\left(\mathbf{x}\right)^T \boldsymbol{\lambda}&=\mathbf{0}, \label{g}\\
    \mathbf{C}\left(\mathbf{x}\right)+\tilde{\boldsymbol{\alpha}} \boldsymbol{\lambda}&=\mathbf{0}.
    \label{h}
\end{align}
We rewrite Eqs. \ref{g} and \ref{h} as
$\mathbf{g}\left(\mathbf{x}, \boldsymbol{\lambda}\right) =\mathbf{0}$ and
    $\mathbf{h}\left(\mathbf{x}, \boldsymbol{\lambda}\right) =\mathbf{0}$, respectively,  
and obtain the following equation by linearizing Eqs. \ref{g} and \ref{h}:
\begin{equation}
    \left[\begin{array}{cc}
        \mathbf{K} & -\nabla \mathbf{C}^T\left(\mathbf{x}^i\right) \\
        \nabla \mathbf{C}\left(\mathbf{x}^i\right) & \tilde{\boldsymbol{\alpha}}
        \end{array}\right]\left[\begin{array}{l}
        \Delta \mathbf{x} \\
        \Delta \boldsymbol{\lambda}
        \end{array}\right]=-\left[\begin{array}{l}
        \mathbf{g}\left(\mathbf{x}^i, \boldsymbol{\lambda}^i\right) \\
        \mathbf{h}\left(\mathbf{x}^i, \boldsymbol{\lambda}^i\right)
    \end{array}\right],
    \label{linearization}
\end{equation}
where $\mathbf{x}^{i}$ denotes the result of the $i$-th iteration, and
\begin{equation}
    \mathbf{K}=\frac{\partial \mathbf{g}}{\partial \mathbf{x}} = \mathbf{M} - \frac{\partial^2 C^T}{\partial \mathbf{x}^2} \cdot \boldsymbol{\lambda},  \nonumber
\end{equation}
\end{frame}

\begin{frame}
By plugging the approximations of XPBD, 
    $\mathbf{K} \approx \mathbf{M}$ and 
    $\mathbf{g} \approx \mathbf{0}$,
    into Eq. \ref{linearization}, we get the following equations to describe the iteration process of XPBD:
\begin{equation}
    \left[\nabla \mathbf{C}\left(\mathbf{x}^i\right) \mathbf{M}^{-1} \nabla \mathbf{C}\left(\mathbf{x}^i\right)^T+\tilde{\boldsymbol{\alpha}}\right] \Delta \boldsymbol{\lambda}=-\mathbf{C}\left(\mathbf{x}^i\right)-\tilde{\boldsymbol{\alpha}} \boldsymbol{\lambda}^i,
    \label{iter_lambda}
\end{equation}
\begin{equation}
    \Delta \mathbf{x}=\mathbf{M}^{-1} \nabla \mathbf{C}\left(\mathbf{x}^i\right)^T \Delta \boldsymbol{\lambda}.
    \label{iter_x}
\end{equation}

\end{frame}


\begin{frame}
Since $\mathbf{M}$ and $\tilde{\boldsymbol{\alpha}}$ are diagonal matrices, Eqs. \ref{iter_lambda} and \ref{iter_x} can be expressed in a matrix-free style. Using $j$ to represent the $j$-th constraint, and $k$ to represent the index of the particle affected by the $j$-th constraint, we have
\begin{align}
    \Delta \lambda_j &=\frac{-C_j -\tilde{\alpha}_j \lambda^i_j}{\nabla C_j {\mathbf{M}_j}^{-1} \nabla C_j^T+\tilde{\alpha}_j}, \label{delta_lambda}\\
    \Delta \mathbf{x}_k &= m_k^{-1} {\frac{\partial C_j}{\partial \mathbf{x}_k}}^T \Delta \lambda_j, k \in \mathcal{O}_j, \label{delta_x}
\end{align}

where $\mathcal{O}_j$ is a set containing the indices of all particles affected by the $j$-th constraint. $\mathbf{M}_j$ is the lumped mass matrix of all particles in $\mathcal{O}_j$. For instance, if the $j$-th length constraint expresses the length relationship between particle $m$ and particle $n$, then $\mathcal{O}_j=\{m, n\}$.
\end{frame}

\begin{frame}
    \frametitle{XPBD Spring Example}
    \begin{figure}[H]
        \centering
        \includegraphics[width=\linewidth,keepaspectratio]{fig_spring_csig2023.png}
    \end{figure}
\end{frame}

\begin{frame}
    \frametitle{Our Constraints}
    \begin{itemize}
        \item [$\circ$] Length Constraint $C(\mathbf{x}_0, \mathbf{x}_1) = \| \mathbf{x}_0 - \mathbf{x}_1\| - l_0 = 0$
        \item [$\circ$] Triangle Strain Constraint $C\left(\mathbf{x}_{0}, \mathbf{x}_{1}, \mathbf{x}_{2}\right)=\operatorname{det}(\mathbf{F})-1=0$.
        \item [$\circ$] External Collision Constraint $C\left(\mathbf{x}_{0}, \mathbf{x}_{1}, \mathbf{x}_{2}\right)=-\left(d\left(\mathbf{x}_{0}, \mathbf{x}_{1}, \mathbf{x}_{2}\right)-d_{\text{thickness}}\right) \leq 0$.
        \item [$\circ$] Fluid Cloth Collision Constraint $C\left(\mathbf{x}_{\text{fluid}},\mathbf{x}_{0}, \mathbf{x}_{1}, \mathbf{x}_{2}\right)=-\left(d\left(\mathbf{x}_{\text{fluid}},\mathbf{x}_{0}, \mathbf{x}_{1}, \mathbf{x}_{2}\right)-d_{\text{thickness}}-r_{\text{fluid}}\right) \leq 0$.
    \end{itemize}
\end{frame}

\begin{frame}
    \frametitle{Solver Implementation}
\begin{itemize}
	\item \textbf{XPBD Solver}: Core Solver
	      \begin{itemize}
		      \item[$\circ$] Constraint Solver:Solve all kinds of constraints
		      \item[$\circ$] Integrator:Integrate particles in time
		      \item[$\circ$] Particle Manager:Manage particle attributes, manage the results of various constraint solutions and apply a specific normalization strategy
		      \item[$\circ$] Topology Manager:Maintain the topology of particles, such as recording specific information on edges, triangles, and quadrilaterals.
	      \end{itemize}
	\item \textbf{Builder}: Tools for generating solvable objects from assets
	      \begin{itemize}
		      \item[$\circ$] ClothBuilder:Cloth asset generation, generate basic grid cloth (Grid), generate cloth from .obj file, etc.
		      \item[$\circ$] ClothPatch:Representation of cloth patches, used to initialize cloth, download current cloth solution results from GPU, etc.
		      \item[$\circ$] ClothMaterial Cloth material.
	      \end{itemize}
\end{itemize}
\end{frame}

\begin{frame}
    \frametitle{Solver Implementation}
    \begin{figure}[H]
        \centering
        \includegraphics[width=\linewidth,keepaspectratio]{fig_xpbd_solver_csig2023.png}
        \caption{XPBD Solver}
    \end{figure}
\end{frame}