\section{项目简介}
本项目为基于高性能计算框架\textbf{Luisa Compute}\cite{Zheng2022LuisaRender}实时物理求解器\textbf{SPHere}。
目前为止本项目实现的求解器组件有:
\begin{itemize}
	\item SPH (Smooth Particle Hydrodynamics)流体仿真求解器:支持弱可压缩、不可压缩流体仿真
	\item XPBD (Extended Position Based Dynamics) 布料仿真求解器:支持基于弹簧质点系统与基于应变的布料模型。
	\item XPBD-SPH耦合器:支持流体与布料的耦合作用,如碰撞
	\item 边界后处理器:支持离散SDF(Signed Distance Field)、解析有向SDF等。
\end{itemize}

其中流体布料碰撞宽检测使用硬件光追进行加速,使用的算法来自文献\cite{rtCollision2022};SPH邻域搜索使用空间哈希以便充分利用LDS(Local Data Share)进行加速,使用的算法来自文献\cite{huangSPH2019}。

在\textbf{SPHere}的整体设计中,我们充分发挥\textbf{Luisa Compute}框架(以下简称LC)的优势,设计并实现了一套适用于实时编译Compute Shader(JIT-Compute-Shader)的模块化包管理系统\textbf{SPHerePackage}。本项目的所有求解器组件及所需的基础设施(如并行原语库等)均建立在\textbf{SPHerePackage}之上。

\section{团队成员与项目名称}
团队成员:
\begin{itemize}
	\item 卢子璇~中国科学院大学
	\item 陆昕瑜~哈尔滨工业大学
	\item 罗旭锟~中国科学院大学
	\item 朱子航~南京大学
	\item 黄可蒙~香港大学
\end{itemize}

SPHere = SPH(光滑粒子动力学)+ Here(这里), 同时还兼有球形(Sphere)的含义。这个提议来自团队成员朱子航。

\section{功能清单}
本项目包含以下功能特性(features)
\begin{itemize}
	\item \textbf{模块化设计}: 模块化与包管理系统\textbf{SPHerePackage}
	\item \textbf{物理仿真}:
	      \begin{itemize}
		      \item[$\circ$]SPH(PCISPH、WCSPH模型)
		      \item[$\circ$]task-based通用实时粒子邻域搜索算法
		      \item[$\circ$]通用硬件光追碰撞宽检测
		      \item[$\circ$]XPBD布料仿真
		      \item[$\circ$]XPBD布料-SPH水体弱耦合
		      \item[$\circ$]SDF静态边界交互
	      \end{itemize}
	\item \textbf{渲染与可视化}:实时相机、实时流体布料渲染
	\item \textbf{并行原语}: Reduce/Scan
	\item \textbf{文件IO}: abc/obj/sdf/smesh
	\item \textbf{资产生成}: cloth builder/fluid builder
	\item \textbf{用户界面}: QT-GUI
	\item \textbf{构建系统}: Xmake
\end{itemize}