\section{边界}
\subsection{静态边界交互}
SDF(signed distance field) 作为一种优秀的隐式模型表示,primitive可以向其快速地查询距离与法向。选取SDF模型作为场景中的静态障碍(obstacles)/边界(boundary),由于primitive向其查询距离与法向的便捷性,碰撞生成过程十分简单。
\subsection{原理}
SDF边界约束分为两步:
\begin{enumerate}
	\item \textbf{碰撞生成}:primitive向SDF询问最近距离,若该距离指示碰撞(例如在SDF模型内侧或者距离小于threshold),则生成该碰撞。
	      \begin{itemize}
		      \item 对于primitive为粒子形式,粒子/顶点直接在SDF中查询距离与法向,使用三重线性插值格式。
		      \item 对于primitive为三角网格形式,计算三角形到SDF的最近距离,生成最近点。(使用文献 \cite{macklin_local_2020} 中的Frank-Wolfe算法),具体来说,使用Frank-Wolfe方法求解以下非线性约束优化问题:
		            \begin{equation}
			            \mathbf x_i =	\arg\min_{\alpha,\beta,\gamma}\phi(\mathbf x_i),\\ \quad s.t.,\quad \mathbf x_i =\alpha \mathbf p_i + \beta\mathbf q_i + \gamma\mathbf r_i, \quad \alpha +\beta  + \gamma = 1.
		            \end{equation}

	      \end{itemize}
	      \item\textbf{ 碰撞处理}:
	      \begin{itemize}
		      \item 对于粒子/顶点,若生成碰撞,则将该粒子/顶点沿着外法向方向移动,直到约束至SDF外侧 /threshold处,同时粒子速度按照法向/切向摩擦系数进行衰减。
		      \item 对于三角网格,若生成碰撞,则生成XPBD外部碰撞约束,在Solver中统一求解。
	      \end{itemize}
\end{enumerate}
上述的原理可由下图展示,同时这也是本仿真求解器的全流程。如下图所示,当与SPH Solver交互时,静态SDF边界模块作为一个较为独立的可视化前后处理模块,这种策略与直接投影法相似;对于带有拓扑结构的三角网格,直接约束顶点会造成著名的“shape feature missing” 问题,因此需要显示求解碰撞点,而后使用碰撞处理与动力耦合的策略在XPBD Solver中求解接触问题。

\begin{figure}[H]
	\centering
	\includegraphics[width=0.9\linewidth,keepaspectratio]{fig_sdf_csig2023.png}
\end{figure}

\begin{figure}[H]
	\centering
	\includegraphics[width=0.6\linewidth,keepaspectratio]{fig_overview_csig2023.png}
\end{figure}

\section{模块架构}

\begin{figure}[H]
	\centering
	\includegraphics[width=0.5\linewidth,keepaspectratio]{fig_boundary_csig2023.png}
\end{figure}

上图展示了静态边界模块的架构。该模块采用三层次设计:
\begin{itemize}
	\item VolumeBoundary: 管理BoundaryConstraints子模块,拥有全局统一的约束容器,协调调度子模块的“约束”方法,最终提交约束。
	\item BoundaryConstraints: 调度子模块SDF,可以使用SDF提供的距离查询函数来生成碰撞,并执行对粒子的碰撞约束过程。
	\item SDF:SDF子模块,拥有SDF资产,管理SDF查询方法,拥有导入资产的接口。
\end{itemize}
下图展示了当流体与静态SDF边界模块交互,该模块以后处理的策略执行\hl{constrain()}的数据流与函数调用。
\begin{figure}[H]
	\centering
	\includegraphics[width=0.7 \linewidth,keepaspectratio]{fig_volume_boundary_package_csig2023.eps}
\end{figure}
