\section{并行原语}
为了支持LC中不同后端(例如DirectX)的Reduce/Scan并行原语,本项目Scan部分实现了文献 \cite{harris2007parallel} 中的\textit{A Work-Efficient Parallel Sca}n算法,同时Reduce部分也基于该算法实现。

\subsection{Scan}
算法主要是按Block划分序列元素,在同一个Block采用传统的Up Sweep/Down Sweep操作,将当前层的每个Block求和结果作为下一层的元素。每层元素数量个数将不断递减,直到只剩至多一个Block元素个数。

处理完最后一个Block后,目前每个block内元素均为其内部前缀和结果,因此还需要加上该block左端点之前的前缀和值,而该值实际上就是下一层的前一个元素(前一步已经计算完毕)。如图中蓝箭头所示,其起始端点0格子内值应当累加至其终端点该block所有格子值。

同时其中会产生辅助数组,用于存放除第一层外的其他层的所有元素,长度为$\frac{n}{256}+\frac{n}{256^2}+...\approx\frac{n}{255}$,其中256为BlockSize。
\begin{figure}[H]
	\centering
	\includegraphics[width=0.7\linewidth,keepaspectratio]{fig_parallel_scan_csig2023.png}
\end{figure}
另外为了避免存取LDS发生Bank Conflict,采用memory padding的方法,即扩宽LDS,避免在同一个Block里操作同一个Bank。编号转换如下,其中额外LDS长度为$BlockSize / BankSize$.
\begin{lstlisting}[language=C++]
index = index + (index) >> LOG_NUM_BANKS;
\end{lstlisting}

\subsection{Reduce}
Reduce实际上可以理解为去掉Down Sweep操作的Scan,因此本项目参考上面Scan算法,去掉Down Sweep操作后,实现了支持求和、取最小值、取最大值的Reduce算法。