
新视角生成(Novel View Synthesis)是模式识别与智能感知领域下,三维重建任务中的重要子任务之一。主要目的是利用若干离散的相机拍摄图片和相机位姿,预测任意位姿下的图像。近年来,随着科研,生产和娱乐产业中信息化,智能化的发展需求,对于实际物体进行三维数字重建成为重要的信息存储和交流媒介。三维信息在影视游戏,机器人,虚拟现实/增强现实,建筑设计,智能生产,数字孪生等领域有广泛的应用。

但与此同时,传统三维建模的方法需要设计者利用高精度扫描仪器进行扫描,利用设计软件手动调整数十万的三角面,微调真实感材质,设置光照,最终使用真实感渲染器进行渲染,才能得到近似于真实三维世界的效果。这个过程的代价十分昂贵。一方面高度依赖建模人员的经验,一方面渲染无法实时完成,应用范围受限。

基于可微渲染的新视角生成方法可以使用手机拍摄等廉价手段拍摄若干照片,利用深度学习或者机器学习进行优化,进而直接进行三维信息重建并生成真实感图片。该方法解决了当前的迫切需求,对于低成本生成高质量三维数字模型具有重要意义。
